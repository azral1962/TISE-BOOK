% Options for packages loaded elsewhere
% Options for packages loaded elsewhere
\PassOptionsToPackage{unicode}{hyperref}
\PassOptionsToPackage{hyphens}{url}
\PassOptionsToPackage{dvipsnames,svgnames,x11names}{xcolor}
%
\documentclass[
  letterpaper,
  DIV=11,
  numbers=noendperiod]{scrreprt}
\usepackage{xcolor}
\usepackage{amsmath,amssymb}
\setcounter{secnumdepth}{5}
\usepackage{iftex}
\ifPDFTeX
  \usepackage[T1]{fontenc}
  \usepackage[utf8]{inputenc}
  \usepackage{textcomp} % provide euro and other symbols
\else % if luatex or xetex
  \usepackage{unicode-math} % this also loads fontspec
  \defaultfontfeatures{Scale=MatchLowercase}
  \defaultfontfeatures[\rmfamily]{Ligatures=TeX,Scale=1}
\fi
\usepackage{lmodern}
\ifPDFTeX\else
  % xetex/luatex font selection
\fi
% Use upquote if available, for straight quotes in verbatim environments
\IfFileExists{upquote.sty}{\usepackage{upquote}}{}
\IfFileExists{microtype.sty}{% use microtype if available
  \usepackage[]{microtype}
  \UseMicrotypeSet[protrusion]{basicmath} % disable protrusion for tt fonts
}{}
\makeatletter
\@ifundefined{KOMAClassName}{% if non-KOMA class
  \IfFileExists{parskip.sty}{%
    \usepackage{parskip}
  }{% else
    \setlength{\parindent}{0pt}
    \setlength{\parskip}{6pt plus 2pt minus 1pt}}
}{% if KOMA class
  \KOMAoptions{parskip=half}}
\makeatother
% Make \paragraph and \subparagraph free-standing
\makeatletter
\ifx\paragraph\undefined\else
  \let\oldparagraph\paragraph
  \renewcommand{\paragraph}{
    \@ifstar
      \xxxParagraphStar
      \xxxParagraphNoStar
  }
  \newcommand{\xxxParagraphStar}[1]{\oldparagraph*{#1}\mbox{}}
  \newcommand{\xxxParagraphNoStar}[1]{\oldparagraph{#1}\mbox{}}
\fi
\ifx\subparagraph\undefined\else
  \let\oldsubparagraph\subparagraph
  \renewcommand{\subparagraph}{
    \@ifstar
      \xxxSubParagraphStar
      \xxxSubParagraphNoStar
  }
  \newcommand{\xxxSubParagraphStar}[1]{\oldsubparagraph*{#1}\mbox{}}
  \newcommand{\xxxSubParagraphNoStar}[1]{\oldsubparagraph{#1}\mbox{}}
\fi
\makeatother


\usepackage{longtable,booktabs,array}
\usepackage{calc} % for calculating minipage widths
% Correct order of tables after \paragraph or \subparagraph
\usepackage{etoolbox}
\makeatletter
\patchcmd\longtable{\par}{\if@noskipsec\mbox{}\fi\par}{}{}
\makeatother
% Allow footnotes in longtable head/foot
\IfFileExists{footnotehyper.sty}{\usepackage{footnotehyper}}{\usepackage{footnote}}
\makesavenoteenv{longtable}
\usepackage{graphicx}
\makeatletter
\newsavebox\pandoc@box
\newcommand*\pandocbounded[1]{% scales image to fit in text height/width
  \sbox\pandoc@box{#1}%
  \Gscale@div\@tempa{\textheight}{\dimexpr\ht\pandoc@box+\dp\pandoc@box\relax}%
  \Gscale@div\@tempb{\linewidth}{\wd\pandoc@box}%
  \ifdim\@tempb\p@<\@tempa\p@\let\@tempa\@tempb\fi% select the smaller of both
  \ifdim\@tempa\p@<\p@\scalebox{\@tempa}{\usebox\pandoc@box}%
  \else\usebox{\pandoc@box}%
  \fi%
}
% Set default figure placement to htbp
\def\fps@figure{htbp}
\makeatother


% definitions for citeproc citations
\NewDocumentCommand\citeproctext{}{}
\NewDocumentCommand\citeproc{mm}{%
  \begingroup\def\citeproctext{#2}\cite{#1}\endgroup}
\makeatletter
 % allow citations to break across lines
 \let\@cite@ofmt\@firstofone
 % avoid brackets around text for \cite:
 \def\@biblabel#1{}
 \def\@cite#1#2{{#1\if@tempswa , #2\fi}}
\makeatother
\newlength{\cslhangindent}
\setlength{\cslhangindent}{1.5em}
\newlength{\csllabelwidth}
\setlength{\csllabelwidth}{3em}
\newenvironment{CSLReferences}[2] % #1 hanging-indent, #2 entry-spacing
 {\begin{list}{}{%
  \setlength{\itemindent}{0pt}
  \setlength{\leftmargin}{0pt}
  \setlength{\parsep}{0pt}
  % turn on hanging indent if param 1 is 1
  \ifodd #1
   \setlength{\leftmargin}{\cslhangindent}
   \setlength{\itemindent}{-1\cslhangindent}
  \fi
  % set entry spacing
  \setlength{\itemsep}{#2\baselineskip}}}
 {\end{list}}
\usepackage{calc}
\newcommand{\CSLBlock}[1]{\hfill\break\parbox[t]{\linewidth}{\strut\ignorespaces#1\strut}}
\newcommand{\CSLLeftMargin}[1]{\parbox[t]{\csllabelwidth}{\strut#1\strut}}
\newcommand{\CSLRightInline}[1]{\parbox[t]{\linewidth - \csllabelwidth}{\strut#1\strut}}
\newcommand{\CSLIndent}[1]{\hspace{\cslhangindent}#1}



\setlength{\emergencystretch}{3em} % prevent overfull lines

\providecommand{\tightlist}{%
  \setlength{\itemsep}{0pt}\setlength{\parskip}{0pt}}



 


\KOMAoption{captions}{tableheading}
\makeatletter
\@ifpackageloaded{bookmark}{}{\usepackage{bookmark}}
\makeatother
\makeatletter
\@ifpackageloaded{caption}{}{\usepackage{caption}}
\AtBeginDocument{%
\ifdefined\contentsname
  \renewcommand*\contentsname{Table of contents}
\else
  \newcommand\contentsname{Table of contents}
\fi
\ifdefined\listfigurename
  \renewcommand*\listfigurename{List of Figures}
\else
  \newcommand\listfigurename{List of Figures}
\fi
\ifdefined\listtablename
  \renewcommand*\listtablename{List of Tables}
\else
  \newcommand\listtablename{List of Tables}
\fi
\ifdefined\figurename
  \renewcommand*\figurename{Figure}
\else
  \newcommand\figurename{Figure}
\fi
\ifdefined\tablename
  \renewcommand*\tablename{Table}
\else
  \newcommand\tablename{Table}
\fi
}
\@ifpackageloaded{float}{}{\usepackage{float}}
\floatstyle{ruled}
\@ifundefined{c@chapter}{\newfloat{codelisting}{h}{lop}}{\newfloat{codelisting}{h}{lop}[chapter]}
\floatname{codelisting}{Listing}
\newcommand*\listoflistings{\listof{codelisting}{List of Listings}}
\makeatother
\makeatletter
\makeatother
\makeatletter
\@ifpackageloaded{caption}{}{\usepackage{caption}}
\@ifpackageloaded{subcaption}{}{\usepackage{subcaption}}
\makeatother
\usepackage{bookmark}
\IfFileExists{xurl.sty}{\usepackage{xurl}}{} % add URL line breaks if available
\urlstyle{same}
\hypersetup{
  pdftitle={TISE Rekayasa Cerdas Berbasis Triune Intelligence:},
  pdfauthor={Armein Z. R. Langi},
  colorlinks=true,
  linkcolor={blue},
  filecolor={Maroon},
  citecolor={Blue},
  urlcolor={Blue},
  pdfcreator={LaTeX via pandoc}}


\title{TISE Rekayasa Cerdas Berbasis Triune Intelligence:}
\usepackage{etoolbox}
\makeatletter
\providecommand{\subtitle}[1]{% add subtitle to \maketitle
  \apptocmd{\@title}{\par {\large #1 \par}}{}{}
}
\makeatother
\subtitle{Panduan Metodologi, Penulisan Disertasi, dan Alat Bantu untuk
Era AI}
\author{Armein Z. R. Langi}
\date{2025-09-18}
\begin{document}
\maketitle

\renewcommand*\contentsname{Table of contents}
{
\hypersetup{linkcolor=}
\setcounter{tocdepth}{2}
\tableofcontents
}

\bookmarksetup{startatroot}

\chapter*{Preface}\label{preface}
\addcontentsline{toc}{chapter}{Preface}

\markboth{Preface}{Preface}

Tentu, berikut adalah draf kata pengantar yang menarik pembaca untuk
menyelami buku Anda hingga tuntas, dengan mengacu pada informasi dari
sumber-sumber yang telah diberikan dan riwayat percakapan kita:

\begin{center}\rule{0.5\linewidth}{0.5pt}\end{center}

\textbf{Kata Pengantar}

Selamat datang, para insinyur dan peneliti abad ke-21. Anda berdiri di
garis depan sebuah era yang penuh dengan tantangan sosio-teknis yang
kompleksitasnya belum pernah terjadi sebelumnya. Dari perubahan iklim
dan keberlanjutan kota hingga sistem kesehatan cerdas dan tata kelola
data yang etis, solusi-solusi tradisional yang terkotak-kotak tak lagi
memadai. Dunia kini menuntut pendekatan rekayasa yang lebih holistik,
sadar nilai, dan cerdas secara fundamental. Ilmu rekayasa tradisional,
meskipun berhasil mengembangkan infrastruktur dan produk, belum cukup
untuk menjawab kompleksitas persoalan kehidupan manusia di abad ke-21
yang membutuhkan transformasi energi, material, informasi, otomasi untuk
mengurangi beban kognitif, kecerdasan berbasis pengetahuan, kapabilitas
massal, dan ruang imersif yang cerdas, kreatif, aman, sehat, serta
berkelanjutan.

Buku ini hadir untuk memperkenalkan Anda pada
\textbf{Triune-Intelligence Smart-Engineering (TISE)}, sebuah paradigma
rekayasa revolusioner yang dirancang khusus untuk menghadapi
tantangan-tantangan tersebut. TISE bukan sekadar serangkaian metode atau
alat; ia adalah sebuah cara pandang transformatif yang memposisikan
manusia sebagai pusat, subjek, dan tujuan akhir rekayasa cerdas, bukan
sebagai objek yang dieliminasi demi efisiensi. Filosofi intinya
melampaui pemecahan masalah teknis semata, bergerak menuju tujuan yang
lebih luhur: \textbf{membangun ``teater kehidupan yang megah'' (splendid
theaters of life) bagi umat manusia}. Ini adalah pergeseran fundamental
dari sekadar \emph{problem-solving} menuju \emph{capability-building},
memberdayakan setiap individu dengan kapabilitas untuk menjalani
kehidupan yang bermakna, kreatif, dan sejahtera secara holistik.

Inti dari TISE adalah sinergi dinamis antara tiga pilar kecerdasan:
\textbf{Kecerdasan Alami (Natural Intelligence - NI)} dari manusia,
\textbf{Kecerdasan Budaya (Cultural Intelligence - CI)} dari institusi
pengetahuan dan nilai kolektif, dan \textbf{Kecerdasan Buatan
(Artificial Intelligence - AI)} dari mesin. Bayangkan sebuah dunia di
mana teknologi tidak menggantikan, melainkan menjadi mitra kolaboratif
yang memperkuat dan memperluas kapabilitas kita. TISE mengintegrasikan
ketiga kecerdasan ini secara eksplisit melalui bahasa, model, dan
ontologi formal, memastikan nilai-nilai etis manusia dan prinsip-prinsip
ekologi dapat dikodekan dan ditindaklanjuti oleh sistem AI. Ontology,
misalnya, dapat menciptakan konseptualisasi bersama yang krusial untuk
interoperabilitas dan semantik umum antar sistem, bahkan mengatasi
masalah di tingkat semantik yang sering menyebabkan kegagalan dalam
\emph{System of Systems} (SoS).

Bagi Anda, para mahasiswa pascasarjana, baik yang sedang menyusun tesis
maupun disertasi, perjalanan ini adalah kesempatan monumental untuk
memberikan kontribusi orisinal pada pengetahuan manusia. Namun,
perjalanan ini seringkali terasa menakutkan dan tidak terstruktur.
\textbf{Buku ini adalah peta jalan Anda.} Di dalamnya, Anda akan
menemukan fondasi teoretis TISE, metodologi penelitian yang sistematis,
dan panduan praktis untuk menerapkan paradigma ini dalam setiap tahap
riset Anda -- mulai dari menemukan topik hingga berhasil dalam ujian
akhir. Anda akan dibekali dengan strategi untuk mempublikasikan temuan
Anda di jurnal-jurnal bereputasi, termasuk menggunakan kerangka kerja
PICOC berlapis untuk membangun argumen yang kuat dan koheren dari riset
fundamental hingga dampak pada pengguna.

Kami akan menunjukkan bagaimana Anda dapat memanfaatkan alat bantu
modern seperti \textbf{Python} untuk implementasi AI dan simulasi,
\textbf{Ontologi} dan \textbf{Prolog} untuk penalaran logis dan
representasi pengetahuan yang menangkap konteks budaya dan nilai
manusia, serta \textbf{Quarto} dan \textbf{Mermaid} untuk dokumentasi
dan visualisasi yang komprehensif (informasi tentang Quarto dan Mermaid
ini adalah inferensi berdasarkan praktik terbaik penulisan teknis). Anda
akan melihat bagaimana Python dapat menjadi ``perekat integrasi'' antara
komponen AI dan pengetahuan, dan bagaimana kombinasi
Ontology-Prolog-Python memungkinkan simulasi kompleks seperti alternatif
transportasi dan dinamika \emph{P2P Lending}. Bahkan, Python dengan
pustaka seperti SymPy dapat digunakan untuk tutorial matematika dan
fisika.

Memahami dan menerapkan TISE akan membekali Anda dengan kemampuan untuk
tidak hanya menjadi seorang insinyur yang kompeten secara teknis, tetapi
juga seorang arsitek masa depan yang bijaksana -- yang mampu
menyinergikan kecerdasan manusia, kecerdasan buatan, dan kecerdasan alam
untuk menciptakan solusi yang benar-benar cerdas, berkelanjutan, dan
manusiawi. Anda akan menjadi \textbf{``Vokator''}---seseorang yang
menyuarakan kebenaran yang ditemukan melalui riset yang bertanggung
jawab dan membangun kehidupan istimewa di tengah disrupsi abad ke-21.

Mari kita bersama-sama menjelajahi paradigma yang akan membentuk masa
depan rekayasa dan masyarakat kita. Selamat membaca dan berinovasi!

Banudng 18 September 2025

Armein Z. R. Langi

\begin{center}\rule{0.5\linewidth}{0.5pt}\end{center}

This is a Quarto book.

To learn more about Quarto books visit
\url{https://quarto.org/docs/books}. Tentu, ini adalah \emph{outline}
komprehensif untuk buku Anda mengenai peran paradigma
Triune-Intelligence Smart-Engineering (TISE) dalam metode rekayasa dan
penulisan disertasi, serta peran alat bantu seperti Quarto, Mermaid,
Python, dan Triune Intelligence, dengan mengacu pada semua sumber yang
diberikan:

\begin{center}\rule{0.5\linewidth}{0.5pt}\end{center}

\textbf{Judul Buku (Tentatif): Rekayasa Cerdas Berbasis Triune
Intelligence: Panduan Metodologi, Penulisan Disertasi, dan Alat Bantu
untuk Era AI}

\textbf{Ringkasan Buku:} Buku ini menyajikan paradigma
\textbf{Triune-Intelligence Smart-Engineering (TISE)} sebagai pendekatan
transformatif dalam rekayasa sistem cerdas di era Kecerdasan Buatan
(AI). TISE memposisikan manusia sebagai pusat inovasi, memberdayakan
pemangku kepentingan untuk mencapai tujuan mereka melalui interaksi
sinergis antara Kecerdasan Alami (NI), Kecerdasan Budaya (CI), dan
Kecerdasan Buatan (AI). Buku ini akan menguraikan fondasi filosofis
TISE, metodologi rekayasanya yang komprehensif, serta panduan praktis
untuk penulisan disertasi dengan kerangka kerja TISE, termasuk
pemanfaatan alat bantu modern seperti Python untuk implementasi AI dan
simulasi, serta Quarto dan Mermaid untuk dokumentasi dan visualisasi.

\begin{center}\rule{0.5\linewidth}{0.5pt}\end{center}

\subsection*{\texorpdfstring{\textbf{Outline
Buku}}{Outline Buku}}\label{outline-buku}
\addcontentsline{toc}{subsection}{\textbf{Outline Buku}}

\textbf{Bagian 1: Memahami Paradigma Triune-Intelligence
Smart-Engineering (TISE)}

\textbf{Bab 1: Pengantar Rekayasa Cerdas di Era AI dan Kebutuhan TISE} *
\textbf{Tantangan rekayasa di abad ke-21}: Kompleksitas masalah
sosio-teknis yang belum pernah terjadi sebelumnya, termasuk perubahan
iklim, keberlanjutan kota, sistem kesehatan cerdas, dan tata kelola data
etis. Ilmu rekayasa tradisional tidak cukup untuk menjawab kompleksitas
persoalan kehidupan manusia di abad ke-21. * \textbf{Pergeseran
paradigma dari rekayasa tradisional ke Rekayasa Cerdas (Smart
Engineering)}: Menuju pendekatan yang lebih holistik, sadar nilai, dan
cerdas secara fundamental. * \textbf{Memperkenalkan Triune-Intelligence
Smart-Engineering (TISE)} sebagai filosofi rekayasa yang transformatif
dan memberdayakan manusia. * Prinsip inti TISE: \textbf{``Engineers
empower humans''}. Manusia adalah pusat, subjek, dan tujuan akhir
rekayasa cerdas, bukan objek yang dieliminasi demi efisiensi. *
Pergeseran dari \emph{problem-solving} menjadi
\emph{capability-building}. Solusi TISE membekali pemangku kepentingan
dengan alat dan kapabilitas untuk menyelesaikan masalah mereka sendiri.
* Tiga pilar fondasi TISE: Teori \textbf{Sistem Sosio-Teknikal (STS)}
yang menekankan interaksi manusia-teknologi; \textbf{Model Kolaborasi
Manusia-AI (HAIC)} untuk memetakan dinamika interaksi; dan
\textbf{Desain Berpusat pada Manusia (HCD) serta Desain Peka Nilai
(VSD)} sebagai metodologi implementasi.

\textbf{Bab 2: Fondasi Triune Intelligence (TI)} * \textbf{Definisi
Triune Intelligence (TI)}: Interplay antara Natural Intelligence (NI),
Cultural Intelligence (CI), dan Artificial Intelligence (AI) yang secara
bersama-sama menciptakan intelijen holistik dan \emph{emergent}. TI
lebih dari sekadar penjumlahan NI+AI+CI; ia adalah \textbf{sistem
intelijen hibrida dinamis} yang bersinergi. * \textbf{Peran Utama Setiap
Kecerdasan dalam TI}: * \textbf{Cultural Intelligence (CI)}: Memberikan
\textbf{``WHY''} (persepsi, pemahaman, tujuan, nilai kolektif,
kearifan). Berfokus pada kebijakan dan tata kelola. * \textbf{Natural
Intelligence (NI)}: Memberikan \textbf{``WHAT''} (pengambilan keputusan,
kontribusi orisinal, penalaran). Berfokus pada keterampilan
sensorimotorik, afeksi, dan kesadaran situasional manusia. *
\textbf{Artificial Intelligence (AI)}: Memberikan \textbf{``HOW''}
(bertindak, eksekusi, komputasi). Berfokus pada kain data, \emph{digital
twin}, dan orkestrator keputusan dengan API \emph{explainability}. * TI
sebagai sistem intelijen hibrida dinamis yang bersinergi, bukan sekadar
penjumlahan. * \textbf{Pemetaan TI ke Model Kolaborasi Manusia-AI
(HAIC)}: TISE menyediakan kerangka kerja yang fleksibel untuk berbagai
mode kolaborasi: Human-Centric (manusia memimpin, AI membantu),
Symbiotic (kemitraan seimbang), dan AI-Centric (AI memimpin, manusia
mengawasi). * \textbf{Metafora dasar}: Otak triune MacLean dan
id-ego-superego Freud sebagai dasar naratif untuk menyeimbangkan
dorongan, pertimbangan, dan norma. NI dianalogikan dengan \emph{id} dan
kompleks reptilian/limbik, CI dengan \emph{superego} dan neokorteks,
serta AI dengan \emph{ego} dan limbik/neokorteks.

\textbf{Bab 3: Karakteristik Artefak Cerdas dalam Paradigma TISE} *
\textbf{Enam Karakteristik Distingtif Artefak Cerdas TISE}: Setiap
artefak rekayasa TISE harus memiliki enam karakteristik yang saling
terkait sebagai kriteria desain dan tolok ukur keberhasilan. *
\textbf{Strong (Kuat)}: Kekuatan inheren melalui \textbf{Core Engine}
(daya fundamental dan kemampuan konversi kerja). AI mengoptimalkan dan
mengontrolnya, NI berkontribusi pada desain dan keamanan, CI
menginformasikan standar keselamatan dan relevansinya. * \textbf{Smart
(Cerdas)}: Adaptabilitas dan intelijen melalui \textbf{Triune-PUDAL
Cognitive Engine} (persepsi, pemahaman, pengambilan keputusan,
tindakan/respons, pembelajaran/evaluasi). Kolaborasi NI-AI, dengan CI
yang menyediakan konteks dan norma. * \textbf{Extended Range (Jangkauan
Luas)}: Penciptaan nilai holistik melalui \textbf{PSKVE Value Engine}
(Produk, Layanan, Pengetahuan, Nilai, Lingkungan). CI mendefinisikan
portofolio nilai, AI mengoptimalkan konversi, NI menyuntikkan
kreativitas dan tujuan. * \textbf{Realistic (Realistis)}: Performa yang
dapat diverifikasi melalui \textbf{PICOC Systematic} (Populasi,
Intervensi, Kontrol, Hasil, Konteks). Seluruh tiga kecerdasan terlibat
untuk validasi berbasis bukti. * \textbf{Doable (Dapat Dilaksanakan)}:
Pengembangan yang terstruktur dan layak melalui \textbf{Triune-ASTF
Four-Layered Architecture} (Aplikasi, Sistem, Teknologi, Riset
Fundamental). Mendekomposisi R\&D untuk NI, CI, dan AI ke dalam
lapisan-lapisan yang dapat dikelola. * \textbf{Methodic (Metodis)}:
Proses pengembangan yang sistematis melalui \textbf{V-Method}. Mengatur
siklus hidup, dari persyaratan berbasis CI hingga sistem NI-CI-AI yang
terintegrasi.

\textbf{Bagian 2: Metodologi Rekayasa dengan Paradigma TISE}

\textbf{Bab 4: Arsitektur Empat Lapisan TISE (ASTF)} * Penjelasan detail
\textbf{Arsitektur ASTF (Application, System, Technology, Fundamental
Research)} sebagai kerangka kerja dekomposisi kompleksitas dan panduan
inovasi. * \textbf{Fokus dan Peran Setiap Lapisan dalam TISE}: *
\textbf{Lapisan Aplikasi (A)}: Didorong oleh \textbf{CI}, fokus pada
``WHY'' (tujuan, nilai, dan masalah pemangku kepentingan di dunia
nyata). Mengidentifikasi masalah dan mendefinisikan solusi yang
diinginkan. * \textbf{Lapisan Sistem (S)}: Integrasi NI-CI-AI (desain
arsitektur sistem keseluruhan, antarmuka manusia-mesin, protokol
kolaborasi). * \textbf{Lapisan Teknologi (T)}: Mesin Kecerdasan
(pengembangan modul teknologi inti, algoritma AI, antarmuka NI, platform
CI). Ini adalah tempat PUDAL Engine dan Core Engine diciptakan. *
\textbf{Lapisan Riset Fundamental (F)}: Fondasi Ilmiah (pengetahuan
dasar, teori baru, prinsip-prinsip yang mendukung setiap kecerdasan). *
ASTF sebagai peta jalan riset, membantu mendefinisikan ruang lingkup dan
mengidentifikasi kontribusi berlapis.

\textbf{Bab 5: Siklus Kognitif PUDAL Engine} * \textbf{PUDAL Engine}
sebagai inti kognitif artefak cerdas dan siklus operasional lima fase:
Perceive, Understand, Decision-making \& Planning, Act-Response, dan
Learning-evaluating. PUDAL dibedakan dari kerangka keputusan lain
seperti OODA oleh fase \emph{Learning-evaluating} yang eksplisit,
menjadikannya arsitektur kognitif sejati. * \textbf{Re-arsitektur PUDAL
sebagai Kolaborasi TI}: * \textbf{Perceive (P)}: AI sebagai penggerak
utama, diarahkan oleh NI/CI. AI meningkatkan kemampuan ini melalui
Computer Vision dan Natural Language Processing (NLP). *
\textbf{Understand (U)}: Kolaborasi AI + CI untuk interpretasi data
kontekstual berdasarkan nilai. CI menyediakan konteks, nilai, dan
kearifan kolektif untuk memahami ``mengapa itu penting''. *
\textbf{Decision-making \& Planning (D)}: Didominasi oleh NI,
diinformasikan oleh AI/CI. NI memanfaatkan kapasitas uniknya untuk
penilaian, kreativitas, pertimbangan etis, dan akuntabilitas. *
\textbf{Act-Response (A)}: AI melaksanakan keputusan NI dengan presisi,
kecepatan, dan keandalan. * \textbf{Learning-evaluating (L)}: Lingkaran
penuh TI (NI + CI + AI) untuk penilaian hasil dan pembaruan
pengetahuan/model. AI mengukur hasil teknis, CI mengevaluasi terhadap
tujuan/nilai, dan NI merefleksikan keputusannya. * \textbf{PUDAL sebagai
``Meta-Prompt Engine''}: Berinteraksi dengan Core Engine melalui
``PROMPTS''. * Arsitektur \textbf{Human-in-the-Loop (HITL)} untuk
implementasi PUDAL: Pre-processing, In-the-loop (Blocking),
Post-processing, Parallel Feedback (Non-blocking).

\textbf{Bab 6: PSKVE Engine: Menciptakan Nilai Holistik dan Berjangkauan
Luas} * \textbf{Konsep PSKVE Energy}: Multi-dimensi nilai yang dikelola
dan dihasilkan oleh artefak cerdas (Product, Service, Knowledge, Value,
Environmental). * \textbf{Peran PSKVE Engine}: Memperluas jangkauan
artefak dan memberikan nilai holistik. PSKVE Engine adalah kerangka
kerja untuk optimasi multi-objektif dalam sistem sosio-teknis yang
kompleks, mengatasi \emph{trade-off} antar tujuan. * \textbf{Dampak
Triune Intelligence pada PSKVE}: CI mendefinisikan portofolio nilai
multi-dimensi, AI mengoptimalkan konversi dan \emph{trade-off} antar
dimensi, NI menyuntikkan kreativitas dan tujuan untuk penciptaan nilai.

\textbf{Bab 7: Metodologi Validasi PICOC Sistematis} * \textbf{PICOC
Systematic}: Kerangka kerja terstruktur untuk mendefinisikan pertanyaan
penelitian dan merancang studi empiris (Population, Intervention,
Control, Outcome, Context). * \textbf{Penerapan PICOC di Setiap Lapisan
ASTF}: Kontribusi metodologis paling orisinal TISE adalah penerapan
PICOC secara sistematis di setiap lapisan ASTF. Ini mengubah PICOC dari
alat tunggal menjadi kerangka validasi multi-lapis yang komprehensif. *
PICOC(A) untuk Lapisan Aplikasi (misalnya, pemangku kepentingan, solusi
baru, kinerja yang dirasakan). * PICOC(S) untuk Lapisan Sistem
(misalnya, \emph{testbed} simulasi, arsitektur sistem, kinerja sistem).
* PICOC(T) untuk Lapisan Teknologi (misalnya, bahan mentah, dapur
otomatis berbasis AI, efisiensi konversi). * PICOC(F) untuk Lapisan
Riset Fundamental (misalnya, fenomena alokasi sumber daya, model
optimasi baru, pengetahuan baru). * Peran TI dalam memperkuat ketelitian
PICOC: NI (penilaian ahli) untuk merumuskan pertanyaan dan menafsirkan
hasil; CI untuk memastikan konteks sosial dan lingkungan; AI untuk
mengumpulkan data, menjalankan simulasi, dan menganalisis hasil.

\textbf{Bab 8: V-Method untuk Rekayasa Cerdas} * \textbf{Prinsip
V-Method}: Model siklus hidup pengembangan sistem berbentuk V yang
menghubungkan fase pengembangan dengan fase pengujian, verifikasi, dan
validasi. V-Method memastikan \textbf{keterlacakan}. * \textbf{Integrasi
V-Method dengan ASTF dan PICOC}: * \textbf{Kaki Kiri (Decomposition and
Definition - ASTF Flow)}: Dari Lapisan Aplikasi ke Riset Fundamental,
dengan PICOC mendefinisikan persyaratan dan kriteria keberhasilan. *
\textbf{Kaki Kanan (Realization, Integration, and Validation - FTSA
Flow)}: Dari Riset Fundamental ke Lapisan Aplikasi, dengan validasi dan
verifikasi di setiap tahap. * \textbf{W-Model} sebagai pengembangan dari
V-Model, menambahkan ``kaki dalam'' yang merepresentasikan proses
internal desain dan validasi berkelanjutan. Kaki kiri dalam untuk
validasi berbasis PICOC. * Peran Triune Intelligence dalam
V-Method/W-Model: AI mendukung pemodelan dan simulasi; NI memandu
kreativitas dan pengambilan keputusan; CI memengaruhi keputusan desain
agar selaras dengan nilai-nilai budaya dan etika.

\textbf{Bagian 3: Penulisan Disertasi dengan Paradigma TISE dan Alat
Bantu}

\textbf{Bab 9: Struktur Disertasi dan Publikasi Ilmiah Berbasis TISE} *
\textbf{Prinsip Penulisan Disertasi/Paper Ilmiah}: Mengajukan cara
pandang/pengetahuan baru berdasarkan hasil riset yang valid. *
\textbf{Jenis-jenis Hipotesis dalam Riset Rekayasa}: Solutif, aplikatif,
dan teoritis. * \textbf{Rekomendasi Struktur Disertasi/Makalah IEEE
dengan PICOC Berlapis}: * \textbf{Judul}: Ringkas, informatif,
mencerminkan kontribusi inti. * \textbf{Abstrak}: Ringkasan 6 poin
(Konteks/Motivasi, Masalah/Kesenjangan, Intervensi/Kontribusi,
Metodologi, Hasil/Luaran Utama, Kesimpulan/Dampak). *
\textbf{Pendahuluan}: Pernyataan masalah (Cx(A)), situasi saat ini
(C(A)), pentingnya, solusi yang diusulkan (I(A)), tujuan dan kontribusi,
garis besar makalah. * \textbf{Karya Terdahulu/Studi Terkait}:
Pendalaman solusi lama, identifikasi kesenjangan di semua lapisan ASTF,
landasan ilmiah untuk solusi baru. * \textbf{Metodologi/Pendekatan yang
Diusulkan}: Detail PICOC untuk setiap lapisan ASTF (P, I, C, O, Cx)
secara rinci, memastikan reproduktifitas. * \textbf{Hasil dan
Pembahasan}: Penyajian dan interpretasi hasil (O) untuk setiap lapisan
ASTF (dari F ke A), koneksi antar-lapisan, keterbatasan, ancaman
terhadap validitas. * \textbf{Kesimpulan dan Saran Pengembangan
Selanjutnya}: Ringkasan kontribusi, jawaban pertanyaan penelitian,
dampak lebih luas, arah penelitian masa depan. * Mengantisipasi
pertanyaan \emph{reviewer} menggunakan kerangka TISE untuk menjelaskan
kebaruan, validasi, dan signifikansi praktis. * Panduan proses riset
disertasi TISE melalui 8 \emph{milestone}, mulai dari menemukan topik
hingga ujian akhir.

\textbf{Bab 10: Peran Alat Bantu: Python, Quarto, Mermaid, dan Triune
Intelligence} * \textbf{Python}: * Bahasa pemrograman utama untuk
\textbf{implementasi AI, simulasi, dan integrasi} dalam TISE. * Peran
dalam PUDAL (komputasi, pembelajaran AI) dan PSKVE (optimasi dimensi
nilai). * Mendukung \textbf{evaluasi PICOC} dengan analisis statistik. *
Bertindak sebagai ``perekat integrasi'' (\emph{integration glue}) antara
komponen AI dan pengetahuan. * Pustaka seperti SymPy dapat digunakan
untuk tutorial matematika dan fisika. * \textbf{Ontologi dan Prolog
(Sebagai Implementasi Triune Intelligence)}: * \textbf{Ontologi}:
Menciptakan \textbf{konseptualisasi bersama} dan \textbf{semantik umum}
antar sistem, yang krusial untuk interoperabilitas. Menangkap
pengetahuan manusia (NI) dan konteks budaya (CI) secara eksplisit,
mengurangi bias, dan memastikan keselarasan semantik. Dapat memformalkan
konsep-konsep inti kerangka TISE (PUDAL, PSKVE, ASTF, PICOC). *
\textbf{Prolog}: Menyediakan representasi pengetahuan deklaratif dan
kemampuan \textbf{penalaran logis}. Menggabungkan NI/CI ke dalam AI
dengan mengkodekan aturan dan preseden logis. * Sinergi
Ontology-Prolog-Python memungkinkan implementasi TI yang kuat: komputasi
AI (Python), penalaran logis (Prolog/NI), dan nilai-nilai (Ontology/CI).
* \textbf{Quarto dan Mermaid}: * \textbf{Quarto}: Sebagai sistem
publikasi ilmiah yang memungkinkan pembuatan buku, disertasi, dan
artikel dengan sintaks Markdown, mendukung integrasi kode dan
\emph{output} secara mulus. (Informasi ini tidak secara langsung
disebutkan dalam sumber yang diberikan, tetapi merupakan praktik umum
yang sangat relevan untuk penulisan buku dan disertasi dengan konten
teknis). * \textbf{Mermaid}: Untuk membuat diagram dan grafik berbasis
teks (seperti diagram alur, diagram urutan, bagan Gantt) yang dapat
diintegrasikan dengan mudah ke dalam dokumen. Meningkatkan kejelasan
visual arsitektur TISE, siklus PUDAL, dan struktur sistem. (Informasi
ini tidak secara langsung disebutkan dalam sumber yang diberikan, tetapi
merupakan praktik umum yang sangat relevan untuk penulisan buku dan
disertasi dengan konten teknis).

\textbf{Bab 11: Implementasi Praktis dan Studi Kasus TISE} *
\textbf{Kerangka Implementasi TISE secara Sistematis}: Tabel yang
menguraikan Lapisan TISE, Tujuan, Prinsip/Teori, Aktivitas Kunci,
Artefak yang Dihasilkan, dan Metrik Validasi. * \textbf{Studi Kasus 1:
Sistem Komuter Cerdas Jakarta-Bandung}: Demonstrasi dekomposisi masalah
kompleks menggunakan ASTF. * Aplikasi (A): Model bisnis ``berbagi
kamar-makanan-perjalanan'' terintegrasi. * Sistem (S): Kapsul tidur
komuter, layanan makanan siap saji, integrasi transportasi. * Teknologi
(T): Mesin listrik, dapur otomatis, uang digital, platform pembiayaan
digital. * Riset Fundamental (F): Optimasi jadwal waktu, alokasi sumber
daya, teori konversi nilai PSKVE. * \textbf{Studi Kasus 2: Analisis
Sistem Rekomendasi Makanan Sehat}: Contoh penerapan TISE pada masalah
sosio-teknis yang kompleks, menunjukkan bagaimana TI, PUDAL, dan PSKVE
bekerja sama. MSRS (Multi-Stakeholder Recommendation System) sebagai
komponen AI. * \textbf{Pemanfaatan Alat Bantu dalam Studi Kasus}: *
\textbf{Simulasi Alternatif Transportasi (Bandung-Jakarta 2030)}:
Ontologi \& Prolog mendefinisikan konsep kendaraan, rute, konsumsi
energi. Python mengeksekusi logika simulasi, menghitung biaya, emisi,
waktu tempuh. * \textbf{Simulasi Dinamika P2P Lending}: Ontologi
komprehensif (pemangku kepentingan, siklus pinjaman, faktor
makroekonomi) dalam Prolog. Python mensimulasikan perilaku agen
berdasarkan \emph{multiplier} yang diinformasikan ontologi di berbagai
skenario ekonomi.

\textbf{Bagian 4: Masa Depan dan Implikasi TISE}

\textbf{Bab 12: Tantangan dan Arah Penelitian Masa Depan dalam TISE} *
Mengatasi tantangan dalam mengukur metrik CI yang ``lunak''
(\emph{soft}) dan memastikan pembobotan yang adil antar-kutub
kecerdasan. * Pengembangan taksonomi yang komprehensif untuk
\emph{System of Autonomous Systems} (SoAS) yang mempertimbangkan otonomi
manajerial dan operasional. * Perlunya kerangka kerja formal untuk
penutupan sistem (fungsional, informasional). * Peningkatan
interoperabilitas alat dan standarisasi bahasa pemodelan untuk
penggunaan kembali pola.

\textbf{Bab 13: Etika dan Tanggung Jawab dalam Rekayasa TISE} * Filosofi
TISE: Memberdayakan manusia dan memastikan teknologi selaras dengan
tubuh dan budaya. * Implikasi sosial dan etika dari integrasi AI, NI,
dan CI, serta perlunya desain yang peka nilai (Value-Sensitive Design).
* Memastikan sistem cerdas bertanggung jawab dan berpusat pada manusia.

\textbf{Bab 14: Kesimpulan: Visi Rekayasa Simbiotik dan Pemberdayaan
Manusia} * Mengulang kembali visi TISE sebagai paradigma rekayasa yang
\textbf{seimbang, partisipatif, dan selaras nilai} untuk era AI. *
Menekankan bagaimana TISE mentransformasi rekayasa menjadi usaha
sosio-teknikal yang kokoh dan berpusat pada manusia. * Peran TISE dalam
menciptakan sistem yang tidak hanya canggih secara teknologi tetapi juga
berharga secara holistik, berdasar etika, dan diterapkan secara
bertanggung jawab.

\begin{center}\rule{0.5\linewidth}{0.5pt}\end{center}

\bookmarksetup{startatroot}

\chapter{Introduction}\label{introduction}

This is a book created from markdown and executable code.

See Knuth (1984) for additional discussion of literate programming.

\bookmarksetup{startatroot}

\chapter{\texorpdfstring{\textbf{Bab 1: Pengantar Rekayasa Cerdas di Era
AI dan Kebutuhan
TISE}}{Bab 1: Pengantar Rekayasa Cerdas di Era AI dan Kebutuhan TISE}}\label{bab-1-pengantar-rekayasa-cerdas-di-era-ai-dan-kebutuhan-tise}

Tentu, berikut adalah draf Bab 1 dari buku Anda:

\begin{center}\rule{0.5\linewidth}{0.5pt}\end{center}

Bab ini memperkenalkan filosofi inti \textbf{Triune-Intelligence
Smart-Engineering (TISE)} dan membedahnya dari pendekatan rekayasa
tradisional. Tujuannya adalah untuk membangun pemahaman mengapa
paradigma baru seperti TISE diperlukan untuk mengatasi
tantangan-tantangan modern.

\section{\texorpdfstring{\textbf{1.1 Definisi dan Filosofi Rekayasa
dalam
TISE}}{1.1 Definisi dan Filosofi Rekayasa dalam TISE}}\label{definisi-dan-filosofi-rekayasa-dalam-tise}

Permasalahan kemanusiaan di abad ke-21 adalah bagaimana manusia dapat
menjalani kehidupan yang istimewa melalui kontribusi maksimal yang
didasarkan pada pertumbuhan potensi sepenuhnya, dalam ruang hidup yang
cerdas, kreatif, aman, sehat, dan berkelanjutan. Ilmu rekayasa
tradisional, meskipun telah berhasil mengembangkan infrastruktur,
produk, dan aplikasi, tidak lagi cukup untuk menjawab kompleksitas
persoalan kehidupan manusia di abad ke-21. Kebutuhan rekayasa saat ini
mencakup transformasi energi, material, dan informasi untuk menggantikan
tenaga manusia; otomasi untuk mengurangi beban kognitif dan kelelahan
mental; kecerdasan berbasis pengetahuan untuk fungsi yang sulit
dilakukan manusia; kebutuhan massal untuk kompetisi pasar dan tujuan
ekonomi; serta kebutuhan ruang imersif untuk lingkungan kehidupan yang
cerdas, kreatif, aman, sehat, dan berkelanjutan.

Dalam kerangka TISE, rekayasa didefinisikan sebagai \textbf{aplikasi
kreatif dari pengetahuan ilmiah untuk menghasilkan artefak (atau
``Rancang-Bangun'' - RB) yang dapat memanfaatkan kekuatan alam secara
aman dan terkendali untuk memecahkan masalah-masalah penting manusia}.
Definisi ini mengandung beberapa elemen kunci: 1. \textbf{``Aplikasi
kreatif''} menekankan bahwa rekayasa bukanlah sekadar penerapan rumus,
melainkan tindakan inovasi. 2. \textbf{``Pengetahuan ilmiah''}
menegaskan bahwa rekayasa berakar pada pemahaman fundamental tentang
cara kerja dunia. 3. \textbf{``Memanfaatkan kekuatan alam''} menyoroti
peran insinyur sebagai mediator antara alam dan kebutuhan manusia. 4.
\textbf{``Memecahkan masalah penting manusia''} menggarisbawahi tujuan
akhir dari semua upaya rekayasa, yaitu memberikan nilai bagi
kemanusiaan.

Namun, filosofi TISE melangkah lebih jauh dari definisi ini. Jika
rekayasa tradisional seringkali berfokus pada optimasi parameter teknis
(misalnya, efisiensi, kecepatan, biaya), TISE memperluas cakrawala
tujuan tersebut. \textbf{Filosofi TISE adalah bahwa artefak rekayasa
yang paling bernilai tidak hanya berfungsi secara teknis, tetapi juga
harus cerdas, adaptif, selaras dengan nilai-nilai kemanusiaan, dan
berkelanjutan secara ekologis}. Tujuannya bukan lagi hanya untuk
``memecahkan masalah,'' tetapi untuk menciptakan nilai holistik dan
meningkatkan kesejahteraan manusia secara berkelanjutan dalam
ekosistemnya. Ini adalah pergeseran dari rekayasa sebagai tindakan
pemecahan masalah murni menjadi \textbf{rekayasa sebagai tindakan
penciptaan dunia yang lebih baik}. Prinsip inti TISE adalah
\textbf{``Engineers empower humans''}. TISE memposisikan manusia sebagai
pusat, subjek, dan tujuan akhir rekayasa cerdas, bukan sebagai objek
yang dieliminasi demi efisiensi. Ini adalah pergeseran dari
\emph{problem-solving} menjadi \emph{capability-building}, di mana
solusi TISE membekali pemangku kepentingan dengan alat dan kapabilitas
untuk menyelesaikan masalah mereka sendiri.

TISE didasarkan pada tiga pilar fondasi: 1. \textbf{Teori Sistem
Sosio-Teknikal (STS)}, yang menekankan interaksi manusia-teknologi. STS
mengakui bahwa elemen sosial dan teknis tidak dapat dipisahkan dan harus
dioptimalkan secara bersama-sama. 2. \textbf{Model Kolaborasi Manusia-AI
(HAIC)}, untuk memetakan dinamika interaksi antara manusia dan AI. 3.
\textbf{Desain Berpusat pada Manusia (HCD) dan Desain Peka Nilai (VSD)},
sebagai metodologi implementasi.

\section{\texorpdfstring{\textbf{1.2 Enam Karakteristik Artefak Cerdas
TISE}}{1.2 Enam Karakteristik Artefak Cerdas TISE}}\label{enam-karakteristik-artefak-cerdas-tise}

Untuk mencapai tujuan filosofisnya, TISE menetapkan bahwa setiap artefak
rekayasa yang dihasilkan harus memiliki enam karakteristik yang saling
terkait. Karakteristik ini berfungsi sebagai kriteria desain dan tolok
ukur keberhasilan.

\begin{enumerate}
\def\labelenumi{\arabic{enumi}.}
\tightlist
\item
  \textbf{Strong (Kuat)}: Setiap artefak harus memiliki \textbf{Core
  Engine (Mesin Inti)} yang fundamental. Mesin ini bertanggung jawab
  atas konversi energi sumber (misalnya, listrik, kimia) menjadi energi
  kerja (misalnya, gerak, komputasi) yang menjadi dasar kekuatan dan
  kapabilitas artefak. Tanpa inti yang kuat, fungsi-fungsi cerdas
  lainnya tidak akan memiliki dasar untuk beroperasi. Metafora mesin
  termodinamika dengan siklus empat langkah dan \emph{flywheel}
  digunakan untuk menjelaskan Core Engine.
\item
  \textbf{Smart (Cerdas)}: Artefak harus memiliki \textbf{PUDAL Engine}
  sebagai inti kognitifnya. Mesin PUDAL (Perceive, Understand,
  Decision-making \& Planning, Act-Response, Learning-evaluating)
  memungkinkan artefak untuk merasakan lingkungannya, memahami konteks,
  membuat keputusan, bertindak, dan belajar dari hasilnya. Kehadiran
  fase `L' (Learning-evaluating) yang eksplisit membedakan PUDAL dari
  kerangka kerja keputusan lain seperti OODA, menjadikannya arsitektur
  kognitif sejati yang memungkinkan adaptasi dan evolusi kecerdasan.
\item
  \textbf{Extended Range (Jangkauan Luas)}: Nilai yang diciptakan oleh
  artefak tidak boleh bersifat mono-dimensi. Artefak harus memiliki
  \textbf{PSKVE Engine} yang mengelola dan mengoptimalkan penciptaan
  nilai di lima dimensi: Product (Produk), Service (Layanan), Knowledge
  (Pengetahuan), Value (Nilai ekonomi/sosial), dan Environmental
  (Lingkungan). Karakteristik ini memastikan bahwa artefak memberikan
  manfaat holistik. PSKVE Engine adalah kerangka kerja untuk optimasi
  multi-objektif dalam sistem sosio-teknis yang kompleks, mengatasi
  \emph{trade-off} antar tujuan.
\item
  \textbf{Realistic (Realistis)}: Klaim kinerja dan manfaat dari artefak
  tidak boleh hanya bersifat teoretis. Kinerjanya harus dapat
  diverifikasi dan divalidasi secara empiris dan ketat. TISE menggunakan
  kerangka kerja \textbf{PICOC (Population, Intervention, Control,
  Outcome, Context)} sebagai metodologi sistematis untuk melakukan
  validasi berbasis bukti ini. PICOC diterapkan di setiap lapisan
  Arsitektur ASTF.
\item
  \textbf{Doable (Dapat Dikerjakan)}: Kompleksitas dalam merancang dan
  membangun artefak cerdas harus dapat dikelola. TISE menggunakan
  dekomposisi hierarkis melalui arsitektur \textbf{ASTF (Application,
  System, Technology, Fundamental Research)} untuk memecah masalah besar
  menjadi lapisan-lapisan yang lebih kecil dan dapat dikelola, dari
  penelitian fundamental hingga aplikasi di dunia nyata.
\item
  \textbf{Methodic (Metodis)}: Proses pengembangan dari awal hingga
  akhir harus terstruktur, dapat dilacak, dan sistematis. TISE
  mengadopsi \textbf{W-Model} dari rekayasa sistem, yang secara logis
  menghubungkan setiap fase dekomposisi desain dengan fase sintesis,
  integrasi, dan validasi yang berkelanjutan. W-Model merupakan
  pengembangan dari V-Model tradisional, yang dirancang untuk menekankan
  verifikasi dan validasi berkelanjutan.
\end{enumerate}

\section{\texorpdfstring{\textbf{1.3 Orisinalitas dan Keunggulan
Paradigma TISE: Sebuah Analisis
Kritis}}{1.3 Orisinalitas dan Keunggulan Paradigma TISE: Sebuah Analisis Kritis}}\label{orisinalitas-dan-keunggulan-paradigma-tise-sebuah-analisis-kritis}

Sebuah paradigma baru tidak lahir dalam ruang hampa. Orisinalitas dan
keunggulan TISE terletak pada kemampuannya untuk menyintesiskan,
menstrukturkan, dan memperluas ide-ide kuat dari berbagai teori rekayasa
dan desain yang sudah mapan.

\section{\texorpdfstring{\textbf{TISE dan Socio-Technical Systems
(STS)}}{TISE dan Socio-Technical Systems (STS)}}\label{tise-dan-socio-technical-systems-sts}

Teori Socio-Technical Systems (STS), yang berasal dari studi di
Tavistock Institute pada tahun 1950-an, menyatakan bahwa sistem kerja
yang efektif hanya dapat dicapai dengan memahami interdependensi yang
tak terpisahkan antara subsistem sosial (manusia, budaya, struktur
organisasi, proses) dan subsistem teknis (teknologi, peralatan,
infrastruktur). Konsep inti dari STS adalah \emph{joint optimization}
(optimasi bersama), yang menegaskan bahwa mengoptimalkan satu subsistem
secara terpisah akan menurunkan kinerja sistem secara keseluruhan.

TISE secara eksplisit mengadopsi filosofi \emph{joint optimization} ini,
namun membawanya ke tingkat yang lebih operasional dan terstruktur. TISE
memberikan anatomi yang lebih rinci dengan memecah ``sosial'' menjadi
dimensi Manusia (Homocordium), Nilai (Value), dan Pengetahuan
(Knowledge), serta memecah ``teknis'' menjadi Produk (Product) dan
Layanan (Service) melalui kerangka Triune Intelligence dan mesin PSKVE.
Lebih jauh lagi, TISE menambahkan dimensi ketiga yang krusial, yaitu
Alam (Natural/Environmental Intelligence). Dengan demikian, TISE tidak
hanya mengadopsi filosofi STS, tetapi juga menyediakan serangkaian
kerangka kerja yang dapat ditindaklanjuti (PUDAL, PSKVE, ASTF, PICOC)
untuk secara sistematis mencapai \emph{joint optimization} yang holistik
ini. TISE dapat dipandang sebagai sebuah implementasi rekayasa yang
terstruktur dari filosofi STS.

\section{\texorpdfstring{\textbf{TISE dan Value-Sensitive Design
(VSD)}}{TISE dan Value-Sensitive Design (VSD)}}\label{tise-dan-value-sensitive-design-vsd}

Value-Sensitive Design (VSD) adalah pendekatan desain yang bertujuan
untuk secara proaktif dan sistematis memasukkan nilai-nilai
kemanusiaan---terutama nilai etika dan moral---ke dalam seluruh siklus
hidup desain teknologi. VSD menggunakan metodologi tripartit yang
melibatkan investigasi konseptual (mengartikulasikan nilai dan pemangku
kepentingan), empiris (mempelajari konteks pengguna), dan teknis
(menganalisis dan membangun teknologi). Fokus utamanya adalah pada
identifikasi pemangku kepentingan dan mengelola potensi ketegangan antar
nilai (\emph{value tensions}).

TISE dapat dilihat sebagai perluasan dan formalisasi dari VSD dalam
konteks rekayasa sistem cerdas. Komponen Homocordium (hati nurani,
nilai, etika) dalam Triune Intelligence dan dimensi Value serta Service
dalam mesin PSKVE adalah mekanisme konkret TISE untuk
mengimplementasikan prinsip-prinsip VSD. TISE menjawab salah satu
tantangan terbesar VSD: bagaimana menerapkannya pada sistem skala besar
dan kompleks seperti \emph{smart cities}. TISE menawarkan solusi
melalui: 1. \textbf{Dekomposisi ASTF}: Untuk mengelola kompleksitas dan
menganalisis implikasi nilai di setiap lapisan, dari fundamental hingga
aplikasi. 2. \textbf{Kerangka PSKVE}: Untuk secara eksplisit
menyeimbangkan keragaman nilai yang seringkali saling bertentangan
(misalnya, efisiensi vs.~privasi). 3. \textbf{Kerangka PICOC Berlapis}:
Untuk menyediakan metodologi empiris yang ketat guna memvalidasi apakah
nilai-nilai yang diinginkan benar-benar terwujud dan memberikan hasil
yang terukur pada artefak yang dibangun.

Dengan demikian, TISE tidak hanya mengadopsi semangat VSD, tetapi juga
melengkapinya dengan struktur rekayasa yang kuat untuk memastikan
nilai-nilai tersebut tidak hanya didiskusikan tetapi juga direalisasikan
dan divalidasi.

\section{\texorpdfstring{\textbf{TISE dan AI Value
Alignment}}{TISE dan AI Value Alignment}}\label{tise-dan-ai-value-alignment}

Masalah penyelarasan nilai (\emph{value alignment problem}) dalam
Kecerdasan Buatan (AI) adalah tantangan untuk memastikan bahwa tujuan
dan perilaku sistem AI selaras dengan nilai-nilai, etika, dan tujuan
kemanusiaan. Kegagalan dalam penyelarasan dapat menyebabkan AI
menghasilkan keputusan yang bias, tidak adil, atau bahkan berbahaya,
meskipun secara teknis berhasil mencapai tujuan yang diprogramkan.

TISE menawarkan solusi struktural untuk masalah penyelarasan nilai ini.
Alih-alih mencoba ``menyelaraskan'' AI sebagai langkah terakhir atau
terpisah, TISE menanamkan nilai-nilai kemanusiaan sejak awal proses
rekayasa melalui \emph{Value-Driven Problem Formulation}. Paradigma
Triune Intelligence secara fundamental menempatkan Kecerdasan Manusia
(Homocordium) sebagai pemandu dan kompas moral bagi Kecerdasan Buatan.
Mesin PUDAL yang ditingkatkan oleh TI dirancang secara eksplisit untuk
memiliki kapasitas ``bertindak etis dan tidak bias secara budaya''. Ini
berarti penyelarasan nilai bukanlah fitur tambahan, melainkan properti
fundamental yang melekat dalam arsitektur kognitif dan proses
pengembangan artefak itu sendiri. \textbf{TISE mengubah \emph{value
alignment} dari sebuah masalah yang harus dipecahkan menjadi sebuah
prinsip yang harus dirancang sejak awal}.

\begin{center}\rule{0.5\linewidth}{0.5pt}\end{center}

\bookmarksetup{startatroot}

\chapter{\texorpdfstring{\textbf{Bab 2: Fondasi Triune Intelligence
(TI)}}{Bab 2: Fondasi Triune Intelligence (TI)}}\label{bab-2-fondasi-triune-intelligence-ti}

Tentu, berikut adalah draf Bab 2 dari buku Anda, yang membahas fondasi
Triune Intelligence (TI), dengan merujuk pada sumber yang diberikan dan
riwayat percakapan kita:

\begin{center}\rule{0.5\linewidth}{0.5pt}\end{center}

Setelah memahami filosofi dan karakteristik umum TISE di Bab 1, bab ini
akan menyelami jantung paradigma tersebut: \textbf{Triune Intelligence
(TI)}. TI adalah konsep fundamental yang memungkinkan artefak cerdas
TISE untuk tidak hanya kuat dan cerdas secara teknis, tetapi juga
selaras dengan nilai-nilai manusia dan batasan alam. Bab ini akan
mendefinisikan TI, menguraikan peran setiap pilar kecerdasannya,
menjelaskan sinerginya, memetakannya ke model kolaborasi manusia-AI, dan
menyoroti metafora dasarnya.

\section{\texorpdfstring{\textbf{2.1 Definisi Triune Intelligence
(TI)}}{2.1 Definisi Triune Intelligence (TI)}}\label{definisi-triune-intelligence-ti}

\textbf{Triune Intelligence (TI)} didefinisikan sebagai interaksi antara
Kecerdasan Alami (Natural Intelligence - NI), Kecerdasan Budaya
(Cultural Intelligence - CI), dan Kecerdasan Buatan (Artificial
Intelligence - AI) yang secara bersama-sama menciptakan intelijen yang
\textbf{holistik dan \emph{emergent}}. Ini bukan sekadar penjumlahan
kapasitas NI, CI, dan AI; sebaliknya, TI adalah \textbf{sistem intelijen
hibrida dinamis} yang bersinergi dan menghasilkan kemampuan yang lebih
besar daripada jumlah bagian-bagiannya.

TI memungkinkan rekayasa artefak yang mampu beradaptasi dan berinovasi
dengan cara yang bertanggung jawab dan berpusat pada manusia. Kunci dari
TI adalah membuat ketiga bentuk kecerdasan ini \textbf{eksplisit melalui
bahasa, model, dan ontologi formal}. Dengan demikian, nilai-nilai etis
manusia atau prinsip-prinsip ekologi dapat dikodekan dan ditindaklanjuti
oleh sistem AI, memungkinkan integrasi sejati. Ontologi, misalnya,
menciptakan konseptualisasi bersama dan semantik umum antar sistem, yang
krusial untuk interoperabilitas. Hal ini membantu menangkap pengetahuan
manusia (NI) dan konteks budaya (CI) secara eksplisit, mengurangi bias,
dan memastikan keselarasan semantik.

\section{\texorpdfstring{\textbf{2.2 Peran Utama Setiap Kecerdasan dalam
TI}}{2.2 Peran Utama Setiap Kecerdasan dalam TI}}\label{peran-utama-setiap-kecerdasan-dalam-ti}

Dalam kerangka TI, setiap jenis kecerdasan memiliki peran distintif yang
berkontribusi pada fungsi keseluruhan sistem:

\begin{enumerate}
\def\labelenumi{\arabic{enumi}.}
\item
  \textbf{Cultural Intelligence (CI)}: Pilar ini memberikan
  \textbf{``WHY''}. CI berfokus pada persepsi, pemahaman, tujuan, nilai
  kolektif, dan kearifan masyarakat atau institusi pengetahuan. Ia
  menentukan kebijakan dan tata kelola yang memandu arah sistem cerdas.
  Dalam TISE, CI membantu mengartikulasikan dan mengukur dimensi nilai
  yang sulit diukur, seperti modal sosial, kepercayaan, atau kesehatan
  ekologis, dan memasukkannya ke dalam kerangka PSKVE. CI juga
  memastikan bahwa validasi mencakup penerimaan pengguna, kepatuhan
  etis, dan adaptabilitas budaya, dengan memasukkan studi pengguna dan
  umpan balik dari budaya target.
\item
  \textbf{Natural Intelligence (NI)}: Pilar ini memberikan
  \textbf{``WHAT''}. NI mencakup kemampuan pengambilan keputusan
  manusia, kontribusi orisinal, dan penalaran. Ia berfokus pada
  keterampilan sensorimotorik, afeksi, dan kesadaran situasional
  manusia. Dalam konteks TISE, Kecerdasan Manusia ini sering disebut
  \textbf{Homocordium} (dari \emph{Homo}, manusia, dan \emph{Cor},
  hati), yang merepresentasikan dimensi ``hati'' manusia: nilai-nilai,
  etika, moralitas, empati, kreativitas, intuisi, dan tujuan spiritual.
  Homocordium berfungsi sebagai kompas moral dan sumber utama definisi
  ``masalah penting manusia'' yang perlu dipecahkan. NI terlibat dalam
  menafsirkan hasil pengujian, melakukan pengujian eksploratif, dan
  memberikan penilaian akhir dalam validasi, terutama untuk kepuasan
  pemangku kepentingan.
\item
  \textbf{Artificial Intelligence (AI)}: Pilar ini memberikan
  \textbf{``HOW''}. AI berfokus pada bertindak, eksekusi, dan komputasi.
  Ia mencakup \emph{data fabric}, \emph{digital twin}, dan orkestrator
  keputusan dengan API \emph{explainability}. AI dalam TISE sering
  disebut \textbf{Homodeus} (manusia-dewa) atau \textbf{Homologos}
  (manusia-logika), merujuk pada potensi AI untuk mencapai kemampuan
  super dan perannya sebagai pemroses informasi dan logika yang kuat.
  Dalam TISE, AI bukanlah tuan, melainkan mitra kuat yang kemampuannya
  diarahkan oleh kebijaksanaan Homocordium. AI mendukung pembuatan
  pengujian otomatis (misalnya, menghasilkan skenario atau kasus
  ekstrem), deteksi anomali dalam hasil pengujian, dan simulasi berbasis
  AI untuk skenario yang kompleks atau berbahaya.
\end{enumerate}

\section{\texorpdfstring{\textbf{2.3 TI sebagai Sistem Intelijen Hibrida
Dinamis}}{2.3 TI sebagai Sistem Intelijen Hibrida Dinamis}}\label{ti-sebagai-sistem-intelijen-hibrida-dinamis}

TI adalah \textbf{sistem intelijen hibrida dinamis} yang bersinergi,
bukan sekadar penjumlahan. Ini berarti ketiga kecerdasan tersebut saling
memengaruhi dan beradaptasi secara berkelanjutan, menciptakan sebuah
\emph{feedback loop} yang memperkaya keseluruhan sistem. Misalnya,
simulasi dinamika P2P \emph{lending} dapat menggunakan ontologi
komprehensif (yang mencakup pemangku kepentingan, siklus pinjaman, dan
faktor makroekonomi) dalam Prolog, sementara Python mensimulasikan
perilaku agen berdasarkan \emph{multiplier} yang diinformasikan oleh
ontologi di berbagai skenario ekonomi. Sinergi Ontology-Prolog-Python
memungkinkan implementasi TI yang kuat: komputasi AI (Python), penalaran
logis (Prolog/NI), dan nilai-nilai (Ontology/CI).

\section{\texorpdfstring{\textbf{2.4 Pemetaan TI ke Model Kolaborasi
Manusia-AI
(HAIC)}}{2.4 Pemetaan TI ke Model Kolaborasi Manusia-AI (HAIC)}}\label{pemetaan-ti-ke-model-kolaborasi-manusia-ai-haic}

TISE menyediakan kerangka kerja yang fleksibel untuk berbagai mode
kolaborasi antara manusia dan AI: * \textbf{Human-Centric}: Manusia
memimpin, dan AI membantu. * \textbf{Symbiotic}: Kemitraan seimbang di
mana manusia dan AI bekerja secara sinergis. * \textbf{AI-Centric}: AI
memimpin, dengan manusia berperan sebagai pengawas atau pemantau.

Konsep TI ini melampaui kerangka kerja kolaborasi Manusia-AI yang umum,
seperti \emph{human-in-the-loop} atau \emph{centaur}, yang masih
memposisikan AI sebagai alat atau asisten. Model \emph{cyborg}, di mana
manusia dan AI terintegrasi sebagai mitra, lebih mendekati visi TI.
Namun, TISE melangkah lebih jauh dengan secara formal memasukkan
\textbf{Kecerdasan Alam/Lingkungan (Natural/Environmental Intelligence)}
sebagai mitra aktif ketiga. Pilar ketiga ini mengakui bahwa alam itu
sendiri memiliki ``kecerdasan'' melalui prinsip-prinsip inheren, hukum
fisika, dinamika ekosistem, dan batasan-batasan sumber daya. Rekayasa
yang bijaksana tidak melawan alam, tetapi bekerja selaras dengannya.
Kecerdasan Alam memberikan batasan-batasan realistis (misalnya, hukum
termodinamika) dan tujuan keberlanjutan (misalnya, keseimbangan
ekologis) bagi artefak yang dirancang.

Dengan demikian, TI dapat dipandang sebagai kerangka kerja untuk
\textbf{Kecerdasan Kolektif Sosial (Social Collective Intelligence)}
yang diperluas, di mana ``kolektif'' tersebut tidak hanya terdiri dari
manusia dan mesin, tetapi juga mencakup sistem alam sebagai pemangku
kepentingan dan sumber kecerdasan yang aktif.

\section{\texorpdfstring{\textbf{2.5 Metafora Dasar: Otak Triune MacLean
dan Id-Ego-Superego
Freud}}{2.5 Metafora Dasar: Otak Triune MacLean dan Id-Ego-Superego Freud}}\label{metafora-dasar-otak-triune-maclean-dan-id-ego-superego-freud}

Untuk memberikan pemahaman naratif yang mendalam tentang bagaimana
ketiga kecerdasan ini berinteraksi, TISE menggunakan metafora dari dua
model psikologis dan neuro-evolusioner yang berpengaruh:

\begin{enumerate}
\def\labelenumi{\arabic{enumi}.}
\tightlist
\item
  \textbf{Otak Triune MacLean}: Model neuro-evolusioner MacLean
  menggambarkan otak manusia dalam tiga lapisan yang berkembang secara
  progresif:

  \begin{itemize}
  \tightlist
  \item
    \textbf{Kompleks Reptilian (Reptilian Complex)}: Ini adalah bagian
    tertua otak, berfokus pada insting dasar bertahan hidup seperti
    \emph{fight-or-flight}, reproduksi, dan teritorialitas. Dalam
    analogi TISE, ini sering dihubungkan dengan \textbf{Natural
    Intelligence (NI)} dan \emph{id} Freud, yang beroperasi pada prinsip
    kesenangan dan dorongan impulsif yang belum terfilter. Bagian ini
    bersifat pre-verbal, impulsif, dan mencari pelepasan segera.
  \item
    \textbf{Sistem Paleomamalia (Limbic System)}: Lapisan tengah ini
    berkaitan dengan emosi, motivasi, keterikatan, dan pembelajaran
    berbasis penghargaan. Dalam TISE, ini memediasi antara dorongan
    dasar dan pertimbangan yang lebih tinggi.
  \item
    \textbf{Neokorteks (Neocortex)}: Ini adalah bagian otak yang paling
    baru berevolusi, bertanggung jawab atas pemikiran abstrak, penalaran
    logis, perencanaan jangka panjang, dan norma sosial. Dalam analogi
    TISE, ini sering dikaitkan dengan \textbf{Cultural Intelligence
    (CI)} dan \emph{superego} Freud, yang mewakili norma dan tujuan
    jangka panjang.
  \end{itemize}
\item
  \textbf{Id-Ego-Superego Freud}: Model struktural Freud juga menawarkan
  tiga elemen yang berinteraksi untuk menjaga perilaku dalam koridor
  yang dapat diterima:

  \begin{itemize}
  \tightlist
  \item
    \textbf{Id}: Reservoir dorongan bawaan yang beroperasi pada prinsip
    kesenangan. Seperti kompleks reptilian, ia bersifat pre-verbal,
    impulsif, dan mencari pelepasan segera.
  \item
    \textbf{Ego}: Prinsip realitas yang menengahi antara tuntutan
    \emph{id}, realitas eksternal, dan batasan \emph{superego}. Dalam
    analogi TISE, AI sering dihubungkan dengan \emph{ego} dan sistem
    limbik/neokorteks, yang berfungsi sebagai orkestrator keputusan dan
    pelaksana.
  \item
    \textbf{Superego}: Mewakili internalisasi norma sosial, moralitas,
    dan ideal. Ini sering dikaitkan dengan CI dan neokorteks, yang
    menyediakan kerangka etis dan nilai.
  \end{itemize}
\end{enumerate}

\textbf{Penting untuk dicatat bahwa analogi ini adalah jembatan naratif
dan pedagogis}, bukan pemetaan neurologis langsung. Otak dan psikis yang
sebenarnya adalah hierarki yang sangat terjalin, bukan modul yang
ditumpuk secara kaku. Namun, model-model ini secara efektif menangkap
\textbf{ketegangan antara impuls, emosi, dan penalaran} yang harus
dikelola dalam sistem cerdas.

Dalam TISE, terjemahan dari triad klasik ini ke dalam rancangan sistem
cerdas meliputi: * \textbf{Id ↔ Reptilian} diterjemahkan menjadi
\textbf{refleks \& afeksi NI}. Desainnya harus menginstrumentasikan
bio-sinyal dan tidak membiarkan otomasi melewati mereka. * \textbf{Ego ↔
Neokorteks} diterjemahkan menjadi \textbf{mesin keputusan AI}. Desainnya
harus memungkinkan AI menjelaskan pilihannya dalam bahasa CI. *
\textbf{Superego} diterjemahkan menjadi \textbf{norma \& visi CI}.
Aturan-aturan ini harus dapat diedit oleh komunitas.

Hasilnya adalah sistem yang menyeimbangkan diri, di mana penalaran mesin
yang cepat selalu dinegosiasikan dengan kebutuhan tubuh yang hidup dan
narasi moral bersama.

\section{\texorpdfstring{\textbf{2.6 TI dalam Meningkatkan Kapasitas
PUDAL dan
PSKVE}}{2.6 TI dalam Meningkatkan Kapasitas PUDAL dan PSKVE}}\label{ti-dalam-meningkatkan-kapasitas-pudal-dan-pskve}

Integrasi Triune Intelligence secara fundamental meningkatkan kapasitas
dan kualitas mesin \textbf{PUDAL (Perceive, Understand, Decision-making
\& Planning, Act-Response, Learning-evaluating)} dan \textbf{PSKVE
(Product, Service, Knowledge, Value, Environmental)}.

\begin{itemize}
\tightlist
\item
  \textbf{Peningkatan Kapasitas PUDAL}: Mesin PUDAL yang hanya didukung
  oleh AI mungkin efisien tetapi berisiko rapuh, bias, atau
  ``berhalusinasi''. TI memperkuat setiap fase PUDAL:

  \begin{itemize}
  \tightlist
  \item
    \textbf{Perceive}: Integrasi kecerdasan manusia dan alam memberikan
    konteks yang lebih kaya pada data yang dirasakan oleh AI. Sistem
    tidak hanya ``melihat'' data, tetapi ``memahami'' signifikansi
    sosial dan ekologisnya.
  \item
    \textbf{Understand}: Dengan kerangka etis (Homocordium) dan hukum
    alam (Natural Intelligence) yang eksplisit, pemahaman sistem menjadi
    lebih mendalam dan tidak bias, membantu AI menghindari kesimpulan
    yang salah dari data historis yang bias.
  \item
    \textbf{Decision-making \& Planning}: Keputusan dioptimalkan tidak
    hanya untuk metrik teknis, tetapi juga untuk keselarasan dengan
    nilai-nilai manusia dan keberlanjutan lingkungan.
  \item
    \textbf{Learning-evaluating}: Sistem belajar tidak hanya untuk
    menjadi lebih akurat, tetapi juga untuk menjadi lebih
    ``baik''---lebih adil, etis, dan berkelanjutan.
  \end{itemize}
\item
  \textbf{Peningkatan Kapasitas PSKVE}: Mengelola lima dimensi PSKVE
  adalah tantangan optimasi yang kompleks. TI membuatnya lebih dapat
  dikelola dengan:

  \begin{itemize}
  \tightlist
  \item
    \textbf{Mendefinisikan Nilai Implisit}: TI membantu
    mengartikulasikan dan mengukur dimensi nilai yang sebelumnya sulit
    diukur, seperti modal sosial atau kesehatan ekologis, dan
    memasukkannya ke dalam kerangka PSKVE.
  \item
    \textbf{Optimasi Holistik}: Dengan panduan dari Homocordium, AI
    dapat mengoptimalkan konversi antar dimensi PSKVE dengan cara yang
    lebih bijaksana, menyeimbangkan keuntungan ekonomi dengan dampak
    sosial dan lingkungan jangka panjang.
  \end{itemize}
\end{itemize}

\begin{center}\rule{0.5\linewidth}{0.5pt}\end{center}

\bookmarksetup{startatroot}

\chapter{\texorpdfstring{\textbf{Bab 3: Karakteristik Artefak Cerdas
dalam Paradigma
TISE}}{Bab 3: Karakteristik Artefak Cerdas dalam Paradigma TISE}}\label{bab-3-karakteristik-artefak-cerdas-dalam-paradigma-tise}

Tentu, berikut adalah draf Bab 3 dari buku Anda, yang menguraikan enam
karakteristik artefak cerdas dalam paradigma TISE:

\begin{center}\rule{0.5\linewidth}{0.5pt}\end{center}

Dalam Bab 1, kita telah memahami filosofi dan definisi rekayasa TISE,
serta kebutuhan akan paradigma ini di era AI yang kompleks. Bab 2
kemudian menyelami fondasi Triune Intelligence (TI), menjelaskan peran
Natural Intelligence (NI), Cultural Intelligence (CI), dan Artificial
Intelligence (AI) sebagai komponen inti dari kecerdasan artefak cerdas
TISE. Bab ini akan menguraikan \textbf{enam karakteristik distintif}
yang harus dimiliki oleh setiap artefak yang dirancang dalam paradigma
TISE. Karakteristik ini berfungsi sebagai kriteria desain yang memandu
pengembangan dan sebagai tolok ukur keberhasilan untuk memvalidasi bahwa
artefak tidak hanya cerdas secara teknis tetapi juga holistik dan
selaras nilai.

Setiap artefak rekayasa TISE harus memiliki enam karakteristik yang
saling terkait sebagai kriteria desain dan tolok ukur keberhasilan.

\section{\texorpdfstring{\textbf{3.1 Strong (Kuat): Fondasi Melalui Core
Engine}}{3.1 Strong (Kuat): Fondasi Melalui Core Engine}}\label{strong-kuat-fondasi-melalui-core-engine}

Artefak cerdas TISE harus \textbf{Kuat (Strong)}, yang berarti ia
memiliki kekuatan inheren melalui \textbf{Core Engine (Mesin Inti)} yang
fundamental. Core Engine adalah jantung keberadaan fisik artefak,
bertanggung jawab untuk mengubah beberapa sumber energi menjadi kerja
yang berguna yang dihasilkan oleh sistem. Ini bisa berupa mesin
pembakaran internal yang mengubah bahan bakar menjadi gerakan mekanis,
atau unit pemrosesan komputer yang mengubah daya listrik menjadi operasi
komputasi. Konsep Core Engine ini selaras dengan rekayasa klasik,
berfokus pada efisiensi, keandalan, dan \emph{output} mentah.

Model Core Engine sering kali dianalogikan sebagai \textbf{proses
siklus} yang mirip dengan roda gila (\emph{flywheel}): energi
\textbf{dikumpulkan} dari sumber masukan, \textbf{dikodekan/dikompresi}
menjadi bentuk yang dapat disimpan (misalnya, baterai terisi atau roda
gila berputar), kemudian \textbf{dilepaskan/didekodekan} menjadi kerja
produktif (roda gila menggerakkan mesin), dan akhirnya sebagian energi
dicadangkan untuk \textbf{mempertahankan/mengatur ulang} mesin untuk
siklus berikutnya. Model siklus ini memastikan mesin dapat beroperasi
terus-menerus. \emph{Output} dari Core Engine ini secara fundamental
adalah \textbf{Energi Produk (Product Energy)}, dimensi pertama dari
PSKVE, yang merepresentasikan kapasitas kerja langsung artefak.

Dalam konteks Triune Intelligence, AI mengoptimalkan dan mengontrol Core
Engine, NI berkontribusi pada desain dan keamanannya, dan CI
menginformasikan standar keselamatan serta relevansinya. Misalnya, mesin
penggerak kendaraan otonom mungkin menggunakan AI untuk kontrol
\emph{real-time}, tetapi aturan jalan yang diikutinya berasal dari
kecerdasan budaya (norma masyarakat yang dikodifikasi menjadi hukum),
dan desainnya berasal dari kecerdasan alami (pengetahuan insinyur).

\section{\texorpdfstring{\textbf{3.2 Smart (Cerdas): Adaptabilitas
Melalui PUDAL Engine
Triune}}{3.2 Smart (Cerdas): Adaptabilitas Melalui PUDAL Engine Triune}}\label{smart-cerdas-adaptabilitas-melalui-pudal-engine-triune}

Artefak cerdas TISE harus \textbf{Cerdas (Smart)}, yang berarti ia
memiliki adaptabilitas dan intelijen melalui \textbf{Triune-PUDAL
Cognitive Engine}. PUDAL (Perceive, Understand, Decision-making \&
Planning, Act-Response, Learning-evaluating) adalah siklus operasional
lima fase yang memungkinkan perilaku cerdas dan adaptif. Dalam paradigma
TISE, PUDAL di-re-arsitektur menjadi proses kolaboratif yang dinamis di
mana NI, CI, dan AI berinteraksi pada setiap tahap.

\begin{itemize}
\tightlist
\item
  \textbf{Perceive (P):} Fase akuisisi data ini utamanya digerakkan oleh
  \textbf{Artificial Intelligence} (misalnya, visi komputer, pemrosesan
  bahasa alami). Namun, NI dan CI dapat memberikan arahan tingkat
  tinggi, memandu AI tentang fenomena apa yang penting untuk dirasakan,
  menyelaraskan pengumpulan data dengan tujuan keseluruhan sistem.
\item
  \textbf{Understand (U):} Fase ini menjadi titik kolaborasi kritis
  antara \textbf{AI dan Cultural Intelligence}. AI memproses data mentah
  untuk mengidentifikasi ``apa yang terjadi'', sementara CI menyediakan
  konteks, nilai, dan kearifan kolektif untuk menafsirkan pola-pola ini
  dan menentukan ``mengapa itu penting''. Metodologi seperti
  \emph{Value-Sensitive Design} dan analisis sosio-teknis diterapkan di
  sini.
\item
  \textbf{Decision-making \& Planning (D):} Fase ini didominasi oleh
  \textbf{Natural Intelligence}, memanfaatkan kapasitas unik manusia
  untuk penilaian, kreativitas, pertimbangan etis, dan akuntabilitas. AI
  dan CI menginformasikan NI dengan analisis data dan interpretasi
  nilai. CI membentuk fase ``Keputusan'' dengan menyematkan konteks,
  prinsip etika, atau aturan sosial.
\item
  \textbf{Act-Response (A):} Setelah keputusan manusia, fase ini kembali
  digerakkan oleh \textbf{Artificial Intelligence}. AI melaksanakan
  tindakan atau rencana yang dipilih dengan presisi, kecepatan, dan
  keandalan.
\item
  \textbf{Learning-evaluating (L):} Fase terakhir ini merupakan
  \textbf{lingkaran penuh Triune Intelligence}. AI mengukur hasil
  teknis, CI mengevaluasi terhadap tujuan dan nilai yang ditetapkan, dan
  NI merefleksikan keberhasilan keputusannya. Hasilnya adalah artefak
  yang secara teknis benar, bermakna, dan dapat diterima dalam konteks.
  Sebagai contoh, sistem diagnostik medis berbasis AI yang cerdas akan
  menggabungkan keahlian dokter (NI), menghormati nilai-nilai budaya
  pasien dalam pilihan pengobatan (CI), dan menggunakan AI untuk
  menganalisis data dan memprediksi hasil.
\end{itemize}

\section{\texorpdfstring{\textbf{3.3 Extended Range (Jangkauan Luas):
Penciptaan Nilai Holistik Melalui PSKVE
Engine}}{3.3 Extended Range (Jangkauan Luas): Penciptaan Nilai Holistik Melalui PSKVE Engine}}\label{extended-range-jangkauan-luas-penciptaan-nilai-holistik-melalui-pskve-engine}

Artefak cerdas TISE harus memiliki \textbf{Jangkauan Luas (Extended
Range)}, yang berarti ia menciptakan nilai holistik melalui
\textbf{PSKVE Value Engine}. PSKVE Engine mengelola dan mengoptimalkan
penciptaan nilai di lima dimensi: Product (Produk), Service (Layanan),
Knowledge (Pengetahuan), Value (Nilai ekonomi/sosial), dan Environmental
(Lingkungan).

\begin{itemize}
\tightlist
\item
  \textbf{Product Energy (Energi Produk):} Kapasitas artefak untuk
  memberikan fungsi fisik atau komputasi utamanya. NI berkontribusi
  dalam rekayasa desain yang efisien, AI dalam optimasi dan kontrol,
  sementara CI menentukan batasan atau persyaratan kinerja produk
  (misalnya, margin keamanan, preferensi budaya).
\item
  \textbf{Service Energy (Energi Layanan):} Upaya yang dapat dicurahkan
  artefak untuk melayani kebutuhan pengguna dan kualitas pengalaman
  pengguna. AI memungkinkan personalisasi dan interaksi alami, NI
  meningkatkan layanan melalui desain berpusat pada manusia dan empati,
  sedangkan CI penting karena harapan layanan berbeda secara budaya.
\item
  \textbf{Knowledge Energy (Energi Pengetahuan):} Informasi dan keahlian
  yang tertanam dalam artefak. AI bertanggung jawab untuk memperoleh dan
  memanfaatkan pengetahuan, NI adalah sumber asli banyak pengetahuan
  (teori, model, interpretasi data dari para ahli), dan CI memengaruhi
  apa yang dianggap valid atau relevan (misalnya, basis pengetahuan
  lokal, bias budaya dalam data).
\item
  \textbf{Value Energy (Energi Nilai):} Nilai sosio-ekonomi dan budaya
  yang dapat diciptakan atau difasilitasi oleh artefak. NI berkontribusi
  pada pemikiran kewirausahaan dan etika (mengidentifikasi peluang
  nilai, memastikan etika), sementara CI membingkai nilai dalam istilah
  kolektif (misalnya, keadilan, kepercayaan pengguna). Penyelarasan
  tujuan AI dengan nilai-nilai manusia (NI dan CI) adalah kunci.
\item
  \textbf{Environmental Energy (Energi Lingkungan):} Dampak ekologis dan
  spasial sistem terhadap lingkungan. AI membantu manajemen sumber daya
  yang efisien dan simulasi dampak lingkungan, NI penting untuk
  kreativitas solusi ramah lingkungan dan pengelolaan, sementara CI
  tercermin dalam prioritas yang diberikan pada masalah lingkungan
  (misalnya, budaya keberlanjutan, norma spasial).
\end{itemize}

Secara keseluruhan, CI mendefinisikan portofolio nilai multi-dimensi, AI
mengoptimalkan konversi dan \emph{trade-off} antar dimensi, dan NI
menyuntikkan kreativitas serta tujuan untuk penciptaan nilai.

\section{\texorpdfstring{\textbf{3.4 Realistic (Realistis): Performa
yang Dapat Diverifikasi Melalui PICOC
Systematic}}{3.4 Realistic (Realistis): Performa yang Dapat Diverifikasi Melalui PICOC Systematic}}\label{realistic-realistis-performa-yang-dapat-diverifikasi-melalui-picoc-systematic}

Artefak cerdas TISE harus \textbf{Realistis (Realistic)}, yang berarti
performanya dapat diverifikasi melalui kerangka kerja \textbf{PICOC
Systematic} (Population, Intervention, Control, Outcome, Context). PICOC
memastikan bahwa solusi rekayasa terbukti efektif dan relevan di dunia
nyata, tidak hanya di atas kertas. Tiga kecerdasan terlibat untuk
validasi berbasis bukti.

\begin{itemize}
\tightlist
\item
  \textbf{Population (P):} Sasaran upaya rekayasa (pengguna akhir,
  skenario penggunaan, kelas sistem). Perspektif Triune memastikan
  populasi mencerminkan keragaman dunia nyata (variasi alami dan
  budaya).
\item
  \textbf{Intervention (I):} Inovasi atau ``perlakuan'' rekayasa yang
  diperkenalkan. Perspektif Triune menggambarkan intervensi tidak hanya
  secara teknis tetapi juga bagaimana hal itu mengubah peran atau alur
  kerja manusia.
\item
  \textbf{Comparison (C):} Pembanding yang digunakan (keadaan seni
  sebelumnya, sistem dasar, skenario kontrol). Perspektif Triune memilih
  perbandingan yang bermakna bagi pemangku kepentingan, termasuk kinerja
  manusia sebagai patokan.
\item
  \textbf{Outcome (O):} Metrik atau kriteria yang menentukan
  keberhasilan (teknis, terkait pengguna, terkait nilai). Perspektif
  Triune memastikan \emph{outcome} mencakup beberapa dimensi PSKVE dan
  hasil etis yang relevan.
\item
  \textbf{Context (C):} Kondisi di mana evaluasi dilakukan (pengaturan
  lingkungan, asumsi, batasan). Perspektif Triune secara eksplisit
  mencakup konteks budaya dan pengguna, bukan hanya fisik.
\end{itemize}

NI memainkan peran kunci dalam mendefinisikan parameter PICOC yang
bermakna (seringkali keputusan ini bergantung pada penilaian manusia dan
pengetahuan domain), AI dapat membantu dengan menganalisis data untuk
memilih dasar yang realistis atau mensimulasikan hasil untuk
perbandingan, dan CI memastikan bahwa konteks dan populasi dipahami
dalam istilah manusia dan budaya.

\section{\texorpdfstring{\textbf{3.5 Doable (Dapat Dilaksanakan):
Pengembangan Terstruktur Melalui Arsitektur
ASTF}}{3.5 Doable (Dapat Dilaksanakan): Pengembangan Terstruktur Melalui Arsitektur ASTF}}\label{doable-dapat-dilaksanakan-pengembangan-terstruktur-melalui-arsitektur-astf}

Artefak cerdas TISE harus \textbf{Dapat Dilaksanakan (Doable)}, yang
berarti pengembangannya terstruktur dan layak melalui
\textbf{Triune-ASTF Four-Layered Architecture} (Application, System,
Technology, Fundamental Research). ASTF adalah metode dekomposisi
hierarkis yang digunakan TISE untuk memecah masalah rekayasa yang besar
dan rumit menjadi empat lapisan yang lebih dapat dikelola dan saling
berhubungan. Ini membantu mendekomposisi R\&D untuk NI, CI, dan AI ke
dalam lapisan-lapisan yang dapat dikelola.

\begin{itemize}
\tightlist
\item
  \textbf{A - Application Layer (Lapisan Aplikasi):} Lapisan teratas
  yang paling dekat dengan pengguna dan dunia nyata. Fokusnya adalah
  memahami masalah pemangku kepentingan dan mendefinisikan solusi yang
  diinginkan. Lapisan ini didorong oleh CI, berfokus pada ``WHY''
  (tujuan, nilai, dan masalah pemangku kepentingan di dunia nyata).
\item
  \textbf{S - System Layer (Lapisan Sistem):} Lapisan ini berfokus pada
  desain arsitektur artefak atau sistem secara keseluruhan, termasuk
  antarmuka manusia-mesin dan protokol kolaborasi. Ini mengintegrasikan
  NI-CI-AI.
\item
  \textbf{T - Technology Layer (Lapisan Teknologi):} Lapisan ini
  berfokus pada pengembangan atau pemilihan mesin atau teknologi kunci
  yang menjadi komponen pembangun sistem, seperti algoritma AI, sensor,
  atau aktuator.
\item
  \textbf{F - Fundamental Research Layer (Lapisan Riset Fundamental):}
  Lapisan terdalam, berfokus pada penemuan atau validasi prinsip-prinsip
  ilmiah atau pengetahuan dasar yang mendasari teknologi.
\end{itemize}

Sebagai contoh, dalam sistem komuter cerdas Jakarta-Bandung, Lapisan
Aplikasi berurusan dengan masalah komuter; Lapisan Sistem merancang
sistem kapsul tidur; Lapisan Teknologi mengembangkan mesin listrik dan
dapur otomatis; dan Lapisan Fundamental meneliti optimasi jadwal dan
alokasi sumber daya.

\section{\texorpdfstring{\textbf{3.6 Methodic (Metodis): Proses
Pengembangan Sistematis Melalui
W-Model}}{3.6 Methodic (Metodis): Proses Pengembangan Sistematis Melalui W-Model}}\label{methodic-metodis-proses-pengembangan-sistematis-melalui-w-model}

Artefak cerdas TISE harus \textbf{Metodis (Methodic)}, yang berarti
proses pengembangannya dari awal hingga akhir terstruktur, dapat
dilacak, dan sistematis melalui \textbf{V-Method}. TISE mengadopsi
\textbf{W-Model}, sebuah pengembangan dari V-Model tradisional, yang
dirancang untuk menekankan verifikasi dan validasi berkelanjutan di
seluruh proses pengembangan. W-Model terdiri dari empat ``kaki'' yang
saling berhubungan:

\begin{enumerate}
\def\labelenumi{\arabic{enumi}.}
\tightlist
\item
  \textbf{Kaki Kiri Luar (Dekomposisi dan Definisi - Alur ASTF):}
  Bergerak turun dari kebutuhan tingkat tinggi ke spesifikasi detail,
  mengoperasionalkan dekomposisi hierarkis ASTF (Aplikasi ke Riset
  Fundamental). Fase ini sangat dipengaruhi oleh CI untuk menangkap
  ``MENGAPA'' dari sistem, termasuk nilai-nilai manusia dan hasil PSKVE
  yang diinginkan.
\item
  \textbf{Kaki Kanan Dalam (Proses Desain dan Sintesis Internal):}
  Berjalan paralel dengan Kaki Kiri Luar, menggambarkan proses internal
  iteratif di mana desain setiap lapisan dibuat. Proses ini bergantung
  pada Triune Intelligence: AI untuk pemodelan dan simulasi, NI untuk
  kreativitas dan pengambilan keputusan, serta CI untuk memastikan
  desain selaras dengan nilai-nilai budaya dan etika.
\item
  \textbf{Kaki Kanan Luar (Realisasi, Integrasi, dan Validasi - Alur
  FTSA):} Bergerak naik dari implementasi tingkat fundamental hingga
  penerapan aplikasi, merepresentasikan proses pembangunan dan pengujian
  (Riset Fundamental ke Aplikasi).
\item
  \textbf{Kaki Kiri Dalam (Validasi Berbasis PICOC):} Berjalan paralel
  dengan Kaki Kanan Luar, merepresentasikan validasi yang ketat dan
  berbasis bukti dari \emph{output} yang direalisasikan. PICOC
  diterapkan di setiap tahap yang selaras dengan ASTF. AI mendukung
  pengujian otomatis, NI terlibat dalam menafsirkan hasil pengujian, dan
  CI memastikan validasi mencakup penerimaan pengguna dan kepatuhan
  etis.
\end{enumerate}

Dengan mengintegrasikan proses desain dan validasi internal secara
eksplisit, serta menenun peran Triune Intelligence secara mendalam,
W-Model memastikan bahwa solusi rekayasa tidak hanya sehat secara teknis
tetapi juga relevan, efektif, dan diterima dalam konteks manusia dan
masyarakat di dunia nyata.

\bookmarksetup{startatroot}

\chapter{\texorpdfstring{\textbf{Bab 4: Kerangka Kerja ASTF: Dekomposisi
Masalah dari Aplikasi ke
Fundamental}}{Bab 4: Kerangka Kerja ASTF: Dekomposisi Masalah dari Aplikasi ke Fundamental}}\label{bab-4-kerangka-kerja-astf-dekomposisi-masalah-dari-aplikasi-ke-fundamental}

Tentu, berikut adalah draf Bab 4 dari buku Anda, yang membahas Kerangka
Kerja ASTF dalam paradigma TISE:

\begin{center}\rule{0.5\linewidth}{0.5pt}\end{center}

Setelah memahami filosofi dan karakteristik umum TISE (Bab 1) serta
fondasi Triune Intelligence (TI) yang menggerakkan artefak cerdas TISE
(Bab 2), dan karakteristik artefak cerdas TISE (Bab 3), bab ini akan
beralih ke salah satu pilar metodologis utama TISE: \textbf{Arsitektur
Empat Lapisan TISE (Application, System, Technology, Fundamental
Research - ASTF)}. Salah satu tantangan terbesar dalam rekayasa sistem
yang kompleks adalah mengelola kompleksitas itu sendiri. Kerangka kerja
ASTF adalah metode dekomposisi hierarkis yang digunakan TISE untuk
memecah masalah rekayasa yang besar dan rumit menjadi empat lapisan yang
lebih dapat dikelola dan saling berhubungan. Ini juga berfungsi sebagai
panduan inovasi.

\section{\texorpdfstring{\textbf{4.1 Membedah Kompleksitas dengan Empat
Lapisan}}{4.1 Membedah Kompleksitas dengan Empat Lapisan}}\label{membedah-kompleksitas-dengan-empat-lapisan}

Setiap lapisan dalam kerangka ASTF memiliki fokus, pertanyaan, dan
\emph{output} yang berbeda, memungkinkan peneliti untuk menavigasi
kompleksitas secara terstruktur:

\begin{enumerate}
\def\labelenumi{\arabic{enumi}.}
\tightlist
\item
  \textbf{A - Application Layer (Lapisan Aplikasi)}

  \begin{itemize}
  \tightlist
  \item
    \textbf{Fokus dan Peran}: Ini adalah lapisan teratas, yang paling
    dekat dengan pengguna dan dunia nyata. Fokusnya adalah memahami
    masalah pemangku kepentingan dan mendefinisikan solusi yang
    diinginkan. Lapisan ini didorong oleh \textbf{Kecerdasan Kultural
    (Cultural Intelligence - CI)}, yang berpusat pada \textbf{``WHY''}
    (tujuan, nilai, dan masalah pemangku kepentingan di dunia nyata).
    \textbf{Kecerdasan Alami (Natural Intelligence - NI)} juga terlibat
    aktif dalam berinteraksi dengan pengguna dan pembuat keputusan untuk
    menangkap kebutuhan sebenarnya.
  \item
    \textbf{Pertanyaan Kunci}: ``Masalah apa yang ingin kita selesaikan
    untuk siapa?'' dan ``Seperti apa solusi yang berhasil itu?''.
  \item
    \textbf{Output}: Serangkaian kebutuhan fungsional dan non-fungsional
    yang jelas.
  \end{itemize}
\item
  \textbf{S - System Layer (Lapisan Sistem)}

  \begin{itemize}
  \tightlist
  \item
    \textbf{Fokus dan Peran}: Lapisan ini berfokus pada desain
    arsitektur Rancang-Bangun (RB) atau sistem secara keseluruhan. Di
    sini, peneliti menerjemahkan kebutuhan dari Lapisan Aplikasi menjadi
    sebuah arsitektur sistem yang koheren. Lapisan ini melibatkan
    \textbf{integrasi NI-CI-AI} (desain arsitektur sistem keseluruhan,
    antarmuka manusia-mesin, protokol kolaborasi).
  \item
    \textbf{Pertanyaan Kunci}: ``Bagaimana kita dapat mengintegrasikan
    berbagai teknologi dan komponen untuk memberikan solusi yang
    dibutuhkan?''.
  \item
    \textbf{Output}: Model sistem, diagram arsitektur, dan spesifikasi
    antarmuka.
  \end{itemize}
\item
  \textbf{T - Technology Layer (Lapisan Teknologi)}

  \begin{itemize}
  \tightlist
  \item
    \textbf{Fokus dan Peran}: Lapisan ini berfokus pada pengembangan
    atau pemilihan mesin atau teknologi kunci yang menjadi komponen
    pembangun sistem. Ini adalah ``mesin'' spesifik yang melakukan
    tugas-tugas penting, seperti algoritma AI, sensor baru, atau
    aktuator. Ini adalah tempat \textbf{PUDAL Engine} dan \textbf{Core
    Engine} diciptakan.
  \item
    \textbf{Pertanyaan Kunci}: ``Teknologi apa yang kita butuhkan untuk
    mengimplementasikan fungsi-fungsi dalam arsitektur sistem?''.
  \item
    \textbf{Output}: Teknologi yang terbukti andal dan berkinerja
    tinggi.
  \end{itemize}
\item
  \textbf{F - Fundamental Research Layer (Lapisan Riset Fundamental)}

  \begin{itemize}
  \tightlist
  \item
    \textbf{Fokus dan Peran}: Ini adalah lapisan terdalam, yang berfokus
    pada penemuan atau validasi prinsip-prinsip ilmiah atau pengetahuan
    dasar yang mendasari teknologi.
  \item
    \textbf{Pertanyaan Kunci}: ``Apa hukum alam, teori matematika, atau
    prinsip ilmiah yang memungkinkan teknologi kita bekerja?''.
  \item
    \textbf{Output}: Pengetahuan baru, teori yang divalidasi, atau model
    fundamental.
  \end{itemize}
\end{enumerate}

\section{\texorpdfstring{\textbf{4.2 Contoh Penerapan ASTF: Sistem
Komuter Cerdas
Jakarta-Bandung}}{4.2 Contoh Penerapan ASTF: Sistem Komuter Cerdas Jakarta-Bandung}}\label{contoh-penerapan-astf-sistem-komuter-cerdas-jakarta-bandung}

Untuk memberikan ilustrasi konkret tentang bagaimana kerangka ASTF
bekerja dalam praktiknya, mari kita terapkan pada masalah sistem komuter
cerdas Jakarta-Bandung:

\begin{itemize}
\tightlist
\item
  \textbf{A (Aplikasi)}: Masalahnya adalah jutaan komuter akan melakukan
  perjalanan harian antara Jakarta dan Bandung yang melelahkan dan
  memakan waktu, sehingga menciptakan kebutuhan mendesak akan solusi
  untuk istirahat, makan, dan perjalanan yang efisien. Solusi yang
  diusulkan adalah model bisnis \textbf{``berbagi
  kamar-makanan-perjalanan'' (room-food-travel sharing)} yang
  terintegrasi untuk menyediakan akomodasi sementara, makanan yang
  nyaman, dan perjalanan yang efisien dalam satu platform.
\item
  \textbf{S (Sistem)}: Sistem yang diusulkan adalah \textbf{``Sistem
  Kapsul Tidur Komuter dan Makanan Siap Saji''
  (Commuter-Sleep-in-Capsule-Food-to-Go System)}. Arsitektur sistemnya
  mengintegrasikan kapsul tidur yang kompak dan nyaman di hub
  transportasi, layanan makanan siap saji yang dapat dipesan sebelumnya,
  dan integrasi yang mulus dengan jaringan transportasi (kereta cepat,
  bus cerdas). Sistem ini berfungsi mengubah input (permintaan komuter,
  energi, bahan makanan) menjadi output (istirahat, nutrisi, perjalanan
  efisien).
\item
  \textbf{T (Teknologi)}: Teknologi kuncinya meliputi \textbf{Mesin
  Listrik} untuk menggerakkan kendaraan transportasi secara efisien dan
  berkelanjutan, \textbf{Dapur Otomatis} untuk produksi makanan massal
  yang cepat, konsisten, dan higienis, \textbf{Uang Digital} untuk
  transaksi yang lancar dan tanpa gesekan, dan \textbf{Platform
  Pembiayaan Digital} untuk mengelola arus kas, investasi, dan
  keberlanjutan ekonomi jangka panjang dari seluruh sistem.
\item
  \textbf{F (Fundamental)}: Prinsip dasar yang diteliti meliputi
  \textbf{Tingkat Jam Kerja Manusia Minimum} (prinsip ekonomi industri
  untuk meminimalkan input tenaga kerja tanpa mengorbankan kualitas),
  \textbf{Optimasi Jadwal Waktu} (prinsip dari riset operasi untuk
  memaksimalkan \emph{throughput} dan meminimalkan waktu tunggu),
  \textbf{Alokasi Sumber Daya} (prinsip dari teori sistem untuk
  mendistribusikan sumber daya terbatas secara optimal), dan
  \textbf{Konversi Nilai PSKVE} (teori fundamental dari TISE tentang
  bagaimana berbagai bentuk ``energi''---Produk, Layanan, Pengetahuan,
  Nilai, Lingkungan---dapat ditransaksikan untuk menciptakan nilai
  holistik).
\end{itemize}

\section{\texorpdfstring{\textbf{4.3 ASTF sebagai Peta Jalan
Riset}}{4.3 ASTF sebagai Peta Jalan Riset}}\label{astf-sebagai-peta-jalan-riset}

Bagi seorang mahasiswa doktoral atau peneliti, kerangka ASTF berfungsi
sebagai peta jalan yang sangat berharga. Ia membantu dalam:

\begin{itemize}
\tightlist
\item
  \textbf{Mendefinisikan Ruang Lingkup}: Mahasiswa dapat dengan jelas
  memposisikan kontribusi utama penelitian mereka. Apakah mereka
  berinovasi di lapisan Teknologi dengan algoritma baru? Atau di lapisan
  Sistem dengan arsitektur baru? Atau mungkin di lapisan Fundamental
  dengan menemukan prinsip baru?.
\item
  \textbf{Mengidentifikasi Kontribusi Berlapis}: Penelitian yang kuat
  seringkali memberikan kontribusi di lebih dari satu lapisan. ASTF
  membantu mengartikulasikan bagaimana inovasi di satu lapisan
  (misalnya, penemuan di F) memungkinkan inovasi di lapisan lain
  (misalnya, teknologi baru di T).
\item
  \textbf{Membangun Narasi yang Koheren}: ASTF menyediakan struktur
  naratif yang logis untuk disertasi. Ceritanya dapat mengalir dari
  pemahaman masalah di lapisan A, turun ke prinsip-prinsip di F, dan
  kemudian kembali naik untuk menunjukkan bagaimana inovasi di setiap
  lapisan berkontribusi pada solusi akhir.
\end{itemize}

Tabel berikut menyediakan matriks yang dapat digunakan mahasiswa untuk
merencanakan penelitian mereka dalam kerangka ASTF:

\textbf{Tabel 4.1: Matriks Kerangka ASTF untuk Perencanaan Riset}

\begin{longtable}[]{@{}
  >{\raggedright\arraybackslash}p{(\linewidth - 6\tabcolsep) * \real{0.0657}}
  >{\raggedright\arraybackslash}p{(\linewidth - 6\tabcolsep) * \real{0.2771}}
  >{\raggedright\arraybackslash}p{(\linewidth - 6\tabcolsep) * \real{0.3914}}
  >{\raggedright\arraybackslash}p{(\linewidth - 6\tabcolsep) * \real{0.2657}}@{}}
\toprule\noalign{}
\begin{minipage}[b]{\linewidth}\raggedright
Lapisan
\end{minipage} & \begin{minipage}[b]{\linewidth}\raggedright
\textbf{Fokus/Tujuan Utama}
\end{minipage} & \begin{minipage}[b]{\linewidth}\raggedright
\textbf{Pertanyaan Kunci}
\end{minipage} & \begin{minipage}[b]{\linewidth}\raggedright
\textbf{Output Khas}
\end{minipage} \\
\midrule\noalign{}
\endhead
\bottomrule\noalign{}
\endlastfoot
\textbf{A (Aplikasi)} & Memahami masalah pemangku kepentingan dan
mendefinisikan solusi yang diinginkan, didorong oleh CI. & ``Masalah apa
yang ingin kita selesaikan untuk siapa?'' ``Seperti apa solusi yang
berhasil itu?'' & Kebutuhan fungsional \& non-fungsional, spesifikasi
kebutuhan pengguna. \\
\textbf{S (Sistem)} & Merancang arsitektur sistem keseluruhan untuk
memenuhi kebutuhan aplikasi, mengintegrasikan NI-CI-AI. & ``Bagaimana
kita dapat mengintegrasikan berbagai teknologi dan komponen untuk
memberikan solusi yang dibutuhkan?'' & Model sistem, diagram arsitektur,
spesifikasi antarmuka. \\
\textbf{T (Teknologi)} & Mengembangkan atau memilih teknologi kunci
(mesin) yang mengimplementasikan fungsi sistem. & ``Teknologi apa yang
kita butuhkan untuk mengimplementasikan fungsi-fungsi dalam arsitektur
sistem?'' & Modul teknologi, algoritma AI, prototipe teknologi. \\
\textbf{F (Riset Fundamental)} & Menemukan atau memvalidasi
prinsip-prinsip ilmiah atau pengetahuan dasar yang mendasari teknologi.
& ``Apa hukum alam, teori matematika, atau prinsip ilmiah yang
memungkinkan teknologi kita bekerja?'' ``Apa kesenjangan pengetahuan
mendasar?'' & Pengetahuan baru, teori yang divalidasi, model
fundamental, prinsip ilmiah baru. \\
\end{longtable}

\begin{center}\rule{0.5\linewidth}{0.5pt}\end{center}

\bookmarksetup{startatroot}

\chapter{\texorpdfstring{\textbf{Bab 5: Siklus Kognitif Triune-PUDAL
Engine}}{Bab 5: Siklus Kognitif Triune-PUDAL Engine}}\label{bab-5-siklus-kognitif-triune-pudal-engine}

Tentu, berikut adalah draf Bab 5 dari buku Anda, yang membahas Siklus
Kognitif Triune-PUDAL Engine, dengan mengacu pada semua sumber yang
diberikan dan riwayat percakapan kita:

\begin{center}\rule{0.5\linewidth}{0.5pt}\end{center}

Setelah menjelajahi struktur dekomposisi masalah melalui kerangka ASTF
di Bab 4, kini kita akan beralih ke inti kognitif dari setiap artefak
cerdas TISE: \textbf{Triune-PUDAL Cognitive Engine}. Jika Core Engine
adalah ``otot'' dari artefak, maka PUDAL Engine adalah ``otak''-nya. Bab
ini akan menguraikan siklus operasional lima fase PUDAL, menjelaskan
re-arsitekturnya sebagai kolaborasi dinamis dari Triune Intelligence
(TI), membedakannya dari kerangka kerja keputusan lain seperti OODA, dan
mengaitkannya dengan implementasi \emph{Human-in-the-Loop} (HITL).

\section{\texorpdfstring{\textbf{5.1 PUDAL Engine: Inti Kognitif untuk
Kecerdasan
Adaptif}}{5.1 PUDAL Engine: Inti Kognitif untuk Kecerdasan Adaptif}}\label{pudal-engine-inti-kognitif-untuk-kecerdasan-adaptif}

\textbf{PUDAL cycle} (siklus PUDAL) adalah inti kognitif yang
memungkinkan artefak cerdas untuk menunjukkan kecerdasan dan kemampuan
beradaptasi. Ini adalah \emph{loop} operasional fundamental yang terdiri
dari lima fase yang saling berhubungan: \textbf{Perceive
(Merasa/Mempersepsikan), Understand (Memahami), Decision-making \&
Planning (Membuat Keputusan \& Perencanaan), Act-Response
(Bertindak-Merespons), dan Learning-evaluating (Belajar-Mengevaluasi)}.

Dalam kerangka kerja Rekayasa Cerdas asli, siklus ini mungkin terlihat
terutama didukung oleh Kecerdasan Buatan (AI). Namun, paradigma Rekayasa
Triune (TISE) mengubah \textbf{``PUDAL Engine''} ini menjadi proses
kolaboratif yang dinamis di mana \textbf{Kecerdasan Alami (Natural
Intelligence - NI), Kecerdasan Budaya (Cultural Intelligence - CI), dan
Kecerdasan Buatan (Artificial Intelligence - AI)} berinteraksi pada
setiap tahap. Siklus PUDAL, dengan demikian, menjadi arena operasional
utama bagi Triune Intelligence, proses di mana logika
\textbf{``WHY-WHAT-HOW''} terus-menerus dieksekusi untuk memecahkan
masalah.

\section{\texorpdfstring{\textbf{5.2 Re-arsitektur PUDAL sebagai
Kolaborasi Triune
Intelligence}}{5.2 Re-arsitektur PUDAL sebagai Kolaborasi Triune Intelligence}}\label{re-arsitektur-pudal-sebagai-kolaborasi-triune-intelligence}

Integrasi Triune Intelligence secara fundamental meningkatkan kapasitas
dan kualitas mesin PUDAL. Mesin PUDAL yang hanya didukung oleh AI
mungkin efisien, tetapi berisiko rapuh, bias, atau bahkan
``berhalusinasi'' (menghasilkan \emph{output} yang tidak masuk akal). TI
memperkuat setiap fase PUDAL melalui kolaborasi NI, CI, dan AI:

\begin{enumerate}
\def\labelenumi{\arabic{enumi}.}
\item
  \textbf{Perceive (P) -- Merasakan/Mempersepsikan}: Fase akuisisi data
  ini utamanya digerakkan oleh \textbf{Artificial Intelligence} melalui
  teknologi seperti \emph{Computer Vision} dan \emph{Natural Language
  Processing} (NLP). Namun, \textbf{Natural Intelligence} (NI) dan
  \textbf{Cultural Intelligence} (CI) memberikan arahan tingkat tinggi,
  memandu AI tentang fenomena apa yang penting untuk dirasakan,
  menyelaraskan pengumpulan data dengan tujuan keseluruhan sistem. Ini
  berarti sistem tidak hanya ``melihat'' data, tetapi ``memahami''
  signifikansi sosial dan ekologisnya.
\item
  \textbf{Understand (U) -- Memahami}: Fase ini merupakan titik
  kolaborasi kritis antara \textbf{AI dan Cultural Intelligence}. AI
  memproses data mentah untuk mengidentifikasi ``apa yang terjadi'',
  sementara CI menyediakan konteks, nilai, dan kearifan kolektif untuk
  menafsirkan pola-pola ini dan menentukan ``mengapa itu penting''.
  Dengan memasukkan kerangka etis (Homocordium) dan hukum alam (Natural
  Intelligence) yang eksplisit, pemahaman sistem menjadi lebih mendalam
  dan tidak bias, membantu AI menghindari kesimpulan yang salah yang
  mungkin timbul dari data historis yang bias.
\item
  \textbf{Decision-making \& Planning (D) -- Membuat Keputusan \&
  Perencanaan}: Fase ini didominasi oleh \textbf{Natural Intelligence},
  memanfaatkan kapasitas unik manusia untuk penilaian, kreativitas,
  pertimbangan etis, dan akuntabilitas. AI berkontribusi melalui
  algoritma optimasi dan \emph{Reinforcement Learning} (RL), sementara
  CI memengaruhi keputusan desain agar selaras dengan nilai-nilai budaya
  dan batasan etis. Keputusan tidak lagi hanya dioptimalkan untuk metrik
  teknis, tetapi juga untuk keselarasan dengan nilai-nilai manusia dan
  keberlanjutan lingkungan.
\item
  \textbf{Act-Response (A) -- Bertindak-Merespons}: Sistem mengeksekusi
  tindakan atau rencana yang telah dipilih, berinteraksi dengan atau
  memodifikasi lingkungannya. Fase ini digerakkan oleh
  \textbf{Artificial Intelligence}, yang mengontrol eksekusi ini melalui
  sistem kontrol robotika atau aktuator dengan presisi, kecepatan, dan
  keandalan.
\item
  \textbf{Learning-evaluating (L) -- Belajar-Mengevaluasi}: Setelah
  bertindak, sistem menilai hasil dari tindakannya dan membandingkannya
  dengan hasil yang diharapkan. Fase ini merupakan \textbf{lingkaran
  penuh Triune Intelligence (NI + CI + AI)}. AI mengukur hasil teknis,
  CI mengevaluasi terhadap tujuan dan nilai yang ditetapkan, dan NI
  merefleksikan keberhasilan keputusannya. Berdasarkan evaluasi ini,
  sistem memperbarui pengetahuan, model, atau strateginya untuk
  meningkatkan kinerja di masa depan, yang didukung oleh berbagai teknik
  \emph{Machine Learning} (ML) dan \emph{Reinforcement Learning} (RL).
  Sistem belajar tidak hanya untuk menjadi lebih akurat, tetapi juga
  untuk menjadi lebih ``baik''---lebih adil, lebih etis, dan lebih
  berkelanjutan. Kemampuan belajar inilah yang menjadi jembatan krusial
  antara kognisi (PUDAL) dan penciptaan nilai (PSKVE).
\end{enumerate}

\section{\texorpdfstring{\textbf{5.3 PUDAL vs.~OODA: Sebuah Arsitektur
Kognitif
Sejati}}{5.3 PUDAL vs.~OODA: Sebuah Arsitektur Kognitif Sejati}}\label{pudal-vs.-ooda-sebuah-arsitektur-kognitif-sejati}

Siklus PUDAL bukanlah urutan linier, melainkan sebuah \emph{feedback
loop} yang berkelanjutan dan berulang. Kehadiran fase \textbf{`L'
(Learning-evaluating)} yang eksplisit adalah hal yang membedakan PUDAL
dari kerangka kerja keputusan lain seperti \textbf{OODA (Observe,
Orient, Decide, Act)} yang populer di kalangan militer. Meskipun OODA
memiliki \emph{feedback loop} implisit, fase `L' yang terstruktur dalam
PUDAL mengubahnya dari sekadar siklus keputusan taktis menjadi
\textbf{arsitektur kognitif sejati}. Ini berarti artefak dengan mesin
PUDAL tidak hanya dirancang untuk bereaksi secara adaptif terhadap
perubahan, tetapi juga untuk \textbf{belajar secara sistematis dan
mengembangkan pemahamannya dari waktu ke waktu}.

\section{\texorpdfstring{\textbf{5.4 PUDAL sebagai ``Meta-Prompt
Engine'' dan Arsitektur
Human-in-the-Loop}}{5.4 PUDAL sebagai ``Meta-Prompt Engine'' dan Arsitektur Human-in-the-Loop}}\label{pudal-sebagai-meta-prompt-engine-dan-arsitektur-human-in-the-loop}

Dalam operasionalnya, PUDAL Engine dapat berfungsi sebagai
\textbf{``Meta-Prompt Engine''} yang berinteraksi dengan Core Engine
melalui ``PROMPTS''. Ini mengindikasikan bahwa siklus kognitif ini
memberikan perintah atau panduan tingkat tinggi kepada mesin inti untuk
melakukan tugas-tugas fisiknya.

Implementasi PUDAL yang efektif dalam sistem cerdas seringkali
mengadopsi arsitektur \textbf{Human-in-the-Loop (HITL)}. Ini melibatkan
empat tahapan utama: * \textbf{Pre-processing}: Manusia terlibat dalam
mempersiapkan data atau aturan awal. * \textbf{In-the-loop (Blocking)}:
Manusia terlibat secara aktif dalam proses pengambilan keputusan
\emph{real-time}, seringkali untuk tugas-tugas kritis yang membutuhkan
penilaian manusia. * \textbf{Post-processing}: Manusia meninjau dan
memverifikasi \emph{output} dari sistem otomatis. * \textbf{Parallel
Feedback (Non-blocking)}: Manusia memberikan umpan balik berkelanjutan
yang digunakan untuk melatih dan meningkatkan sistem secara tidak
langsung.

Penting untuk dicatat bahwa \emph{Level of Autonomy} (LoA) dapat
memengaruhi kinerja manusia dan sistem. LoA yang lebih rendah dapat
menghasilkan kinerja sistem yang lebih baik, tetapi justru merugikan
kinerja pengguna. Hal ini menyoroti perlunya menyeimbangkan otomatisasi
dengan peran manusia untuk mencapai kinerja SoAS (System of Autonomous
Systems) secara keseluruhan yang optimal.

\section{\texorpdfstring{\textbf{5.5 Metrik Kapasitas
Triune-PUDAL}}{5.5 Metrik Kapasitas Triune-PUDAL}}\label{metrik-kapasitas-triune-pudal}

Untuk mengukur efektivitas dan kapasitas Triune-PUDAL Engine, TISE
menyarankan penggunaan metrik yang melampaui kinerja teknis AI semata,
mencakup dimensi NI dan CI. Beberapa klaster metrik yang relevan
meliputi:

\begin{itemize}
\tightlist
\item
  \textbf{Natural-Intelligence Fitness}: Mengukur kesehatan dan
  kewaspadaan ``lengkung refleks Id + limbik'' manusia.

  \begin{itemize}
  \tightlist
  \item
    \textbf{Bio-Responsiveness RT (Waktu Reaksi Bio)}: Waktu reaksi
    median dari stimulus fisik hingga aktuasi aman.
  \item
    \textbf{Cognitive-Load Index (Indeks Beban Kognitif)}: Usaha mental
    operator saat mengawasi AI.
  \item
    \textbf{Trust Calibration Error (Kesalahan Kalibrasi Kepercayaan)}:
    Kesenjangan antara kepercayaan manusia dan akurasi sebenarnya dari
    AI.
  \end{itemize}
\item
  \textbf{Collective Intelligence (Kecerdasan Kolektif)}: Mengukur
  kemampuan sistem untuk berkolaborasi dan belajar secara sinergis.

  \begin{itemize}
  \tightlist
  \item
    \textbf{Decision Alignment Index (Indeks Penyelarasan Keputusan)}:
    Porsi keputusan akhir yang mengutip masukan dari ketiga pilar
    kecerdasan (NI, CI, AI).
  \item
    \textbf{Resolution-Latency (Latensi Resolusi)}: Waktu rata-rata
    untuk menyelesaikan konflik NI-AI-CI yang ditandai oleh lapisan tata
    kelola.
  \end{itemize}
\item
  \textbf{Societal \& Ethical Impact (Dampak Sosial \& Etika)}: Mengukur
  nilai ``Rekayasa untuk Kemanusiaan'' tertinggi.

  \begin{itemize}
  \tightlist
  \item
    \textbf{Safety Incident Rate (Tingkat Insiden Keamanan)}: Kejadian
    kritis per 10 ribu jam operasi.
  \item
    \textbf{Inclusivity Spread (Penyebaran Inklusivitas)}: Ukuran
    distribusi nilai di berbagai kelompok pemangku kepentingan.
  \item
    \textbf{Sustainable-Benefit ROI (ROI Manfaat Berkelanjutan)}:
    Eksternalitas positif bersih jangka panjang (energi yang dihemat,
    kehidupan yang ditingkatkan) per unit biaya.
  \end{itemize}
\end{itemize}

Metrik-metrik ini, yang dikelompokkan dalam Indeks Kapasitas Triune
(TCI), memungkinkan kita untuk melampaui pertanyaan ``Apakah modelnya
berfungsi?'' menjadi ``Apakah seluruh sistem Triune aman, cerdas, adil,
dan terus belajar?''---bukti nyata bahwa TI memberikan janjinya untuk
kemanusiaan.

\begin{center}\rule{0.5\linewidth}{0.5pt}\end{center}

\bookmarksetup{startatroot}

\chapter{\texorpdfstring{\textbf{Bab 6: PSKVE Engine: Menciptakan Nilai
Holistik dan Berjangkauan
Luas}}{Bab 6: PSKVE Engine: Menciptakan Nilai Holistik dan Berjangkauan Luas}}\label{bab-6-pskve-engine-menciptakan-nilai-holistik-dan-berjangkauan-luas}

Tentu, berikut adalah draf Bab 6 dari buku Anda, yang membahas PSKVE
Engine dan peranannya dalam menciptakan nilai holistik dan berjangkauan
luas, dengan mengacu pada semua sumber yang diberikan dan riwayat
percakapan kita:

\begin{center}\rule{0.5\linewidth}{0.5pt}\end{center}

Di Bab 4, kita telah memahami bagaimana kerangka kerja ASTF digunakan
untuk dekomposisi masalah. Di Bab 5, kita mendalami Triune-PUDAL Engine
sebagai inti kognitif artefak cerdas. Kini, kita akan beralih ke
karakteristik ketiga dari artefak cerdas TISE: \textbf{Jangkauan Luas
(Extended Range)}, yang diwujudkan melalui \textbf{PSKVE Value Engine}.
Bab ini akan menjelaskan konsep energi multi-dimensi dalam PSKVE,
perannya dalam menciptakan nilai holistik, dan bagaimana Triune
Intelligence (TI) memengaruhi dan meningkatkan setiap dimensi nilai ini.

\section{\texorpdfstring{\textbf{6.1 Konsep PSKVE Energy: Abstraksi
Nilai
Multi-Dimensi}}{6.1 Konsep PSKVE Energy: Abstraksi Nilai Multi-Dimensi}}\label{konsep-pskve-energy-abstraksi-nilai-multi-dimensi}

Dalam kerangka \emph{Smart Engineering}, \textbf{PSKVE Energy} adalah
konseptualisasi multi-dimensi dari ``energi'' yang menangkap berbagai
bentuk nilai dan kapasitas yang dikelola dan dihasilkan oleh artefak
cerdas. Artefak cerdas adalah \emph{output} yang berwujud atau tidak
berwujud dari \emph{Smart Engineering} yang mewujudkan dan memanipulasi
energi-energi ini. PSKVE adalah akronim dari lima dimensi nilai:

\begin{itemize}
\tightlist
\item
  \textbf{Product (Produk)}
\item
  \textbf{Service (Layanan)}
\item
  \textbf{Knowledge (Pengetahuan)}
\item
  \textbf{Value (Nilai)}
\item
  \textbf{Environmental (Lingkungan)}
\end{itemize}

Konsep PSKVE adalah abstraksi dari solusi dan proses rekayasa. Ini
memungkinkan pandangan yang seragam terhadap berbagai masalah rekayasa
dan memungkinkan penciptaan model komputasi untuk rekayasa cerdas. PSKVE
yang komprehensif (Product-Service-Knowledge-Value-Environmental) adalah
output dari rekayasa modern.

\section{\texorpdfstring{\textbf{6.2 Peran PSKVE Engine: Memperluas
Jangkauan dan Memberikan Nilai
Holistik}}{6.2 Peran PSKVE Engine: Memperluas Jangkauan dan Memberikan Nilai Holistik}}\label{peran-pskve-engine-memperluas-jangkauan-dan-memberikan-nilai-holistik}

\textbf{PSKVE Engine} memperluas jangkauan artefak dan memberikan nilai
holistik. Ini adalah kerangka kerja untuk \textbf{optimasi
multi-objektif} dalam sistem sosio-teknis yang kompleks, mengatasi
\emph{trade-off} antar tujuan yang mungkin bertentangan. Sebuah
\emph{smart artefact} TISE tidak boleh hanya berfokus pada satu jenis
nilai (misalnya, nilai ekonomi), melainkan harus secara bersamaan
mempertimbangkan dampaknya pada semua dimensi PSKVE.

\section{\texorpdfstring{\textbf{6.3 Dampak Triune Intelligence pada
PSKVE}}{6.3 Dampak Triune Intelligence pada PSKVE}}\label{dampak-triune-intelligence-pada-pskve}

Triune Intelligence (NI, CI, AI) secara fundamental meningkatkan
kapasitas dan kualitas mesin PSKVE. Mengelola lima dimensi PSKVE adalah
tantangan optimasi yang sangat kompleks, dan TI membuatnya lebih dapat
dikelola dengan:

\begin{itemize}
\tightlist
\item
  \textbf{Mendefinisikan Nilai Implisit}: TI membantu mengartikulasikan
  dan mengukur dimensi nilai yang sebelumnya sulit diukur, seperti modal
  sosial, kepercayaan, atau kesehatan ekologis, dan memasukkannya ke
  dalam kerangka PSKVE.
\item
  \textbf{Optimasi Holistik}: Dengan panduan dari \emph{Homocordium}
  (NI), AI dapat mengoptimalkan konversi antar dimensi PSKVE dengan cara
  yang lebih bijaksana, menyeimbangkan keuntungan ekonomi dengan dampak
  sosial dan lingkungan jangka panjang.
\end{itemize}

Secara spesifik: * \textbf{Cultural Intelligence (CI)} mendefinisikan
portofolio nilai multi-dimensi. Ini memengaruhi apa yang dianggap
``bernilai'' dari perspektif sosial, budaya, dan etis, serta
menegosiasikan pertukaran nilai non-moneter. * \textbf{Artificial
Intelligence (AI)} mengoptimalkan konversi dan \emph{trade-off} antar
dimensi nilai. AI menganalisis pasar, mengoptimalkan harga, dan
mengelola transaksi keuangan serta aset digital secara efisien. Selain
itu, AI dapat memodelkan, memantau, dan mengoptimalkan penggunaan sumber
daya dan dampak lingkungan (misalnya, efisiensi energi, pengurangan
limbah). * \textbf{Natural Intelligence (NI)} menyuntikkan kreativitas
dan tujuan untuk penciptaan nilai. Ini melibatkan desain produk yang
ergonomis, estetis, dan memenuhi kebutuhan fungsional serta emosional
manusia. NI juga penting dalam memberikan empati, komunikasi, dan
sentuhan personal dalam layanan untuk membangun kepercayaan dan
kepuasan. Lebih lanjut, NI menciptakan pengetahuan baru melalui
kreativitas, pemikiran kritis, dan sintesis lintas disiplin. NI juga
memiliki kesadaran dan kepedulian etis terhadap dampak lingkungan,
mendorong praktik berkelanjutan.

\section{\texorpdfstring{\textbf{6.4 Dimensi PSKVE secara Rinci: Artefak
dan
Energi}}{6.4 Dimensi PSKVE secara Rinci: Artefak dan Energi}}\label{dimensi-pskve-secara-rinci-artefak-dan-energi}

Pemeriksaan lebih dalam setiap dimensi PSKVE mengungkapkan cakupan luas
dari konsep ini dan peran Triune Intelligence dalam masing-masing
dimensi:

\begin{enumerate}
\def\labelenumi{\arabic{enumi}.}
\tightlist
\item
  \textbf{Product (Produk)}: Ini adalah kapasitas artefak untuk
  memberikan fungsi fisik atau komputasi utamanya.

  \begin{itemize}
  \tightlist
  \item
    \textbf{Peran TI}: NI berkontribusi dalam mendesain produk yang
    ergonomis, estetis, dan memenuhi kebutuhan fungsional dan emosional
    manusia. AI mengoptimalkan desain produk untuk kinerja, efisiensi,
    dan keandalan manufaktur melalui simulasi dan analisis. Kecerdasan
    Alam/Lingkungan menyediakan bahan baku dan hukum fisika yang
    mengatur properti dan perilaku produk.
  \end{itemize}
\item
  \textbf{Service (Layanan)}: Ini adalah upaya yang dapat dicurahkan
  artefak untuk melayani kebutuhan pengguna dan kualitas pengalaman
  pengguna.

  \begin{itemize}
  \tightlist
  \item
    \textbf{Peran TI}: NI memberikan empati, komunikasi, dan sentuhan
    personal dalam layanan untuk membangun kepercayaan dan kepuasan. AI
    mengotomatiskan dan mempersonalisasi pengiriman layanan dalam skala
    besar, serta menganalisis umpan balik pelanggan. Kecerdasan
    Alam/Lingkungan menyediakan konteks lingkungan (misalnya, lokasi,
    cuaca) di mana layanan diberikan. Harapan layanan berbeda secara
    budaya, sehingga CI penting di sini.
  \end{itemize}
\item
  \textbf{Knowledge (Pengetahuan)}: Ini adalah energi intelektual yang
  terkandung dalam skema, algoritma, keahlian, dan data, yang merupakan
  kapasitas untuk memecahkan masalah yang sulit.

  \begin{itemize}
  \tightlist
  \item
    \textbf{Peran TI}: NI menciptakan pengetahuan baru melalui
    kreativitas, pemikiran kritis, dan sintesis lintas disiplin. AI
    mengelola, menganalisis, dan mengekstrak wawasan dari basis data
    pengetahuan yang sangat besar, serta menemukan pola tersembunyi.
    Kecerdasan Alam/Lingkungan menjadi sumber pengetahuan fundamental
    tentang cara kerja alam semesta. CI juga memengaruhi apa yang
    dianggap valid atau relevan (misalnya, basis pengetahuan lokal, bias
    budaya dalam data).
  \end{itemize}
\item
  \textbf{Value (Nilai)}: Ini adalah kapasitas untuk memberi,
  merepresentasikan, atau menukarkan nilai (finansial, ekonomi, sosial,
  budaya).

  \begin{itemize}
  \tightlist
  \item
    \textbf{Peran TI}: NI mendefinisikan apa yang dianggap ``bernilai''
    dari perspektif sosial, budaya, dan etis, serta menegosiasikan
    pertukaran nilai non-moneter. AI menganalisis pasar, mengoptimalkan
    harga, dan mengelola transaksi keuangan serta aset digital secara
    efisien. Kecerdasan Alam/Lingkungan menyediakan sumber daya alam
    yang memiliki nilai intrinsik dan menjadi dasar bagi semua nilai
    ekonomi. Penyelarasan tujuan AI dengan nilai-nilai manusia (NI dan
    CI) adalah kunci.
  \end{itemize}
\item
  \textbf{Environmental (Lingkungan)}: Ini adalah dampak ekologis dan
  spasial sistem terhadap lingkungan.

  \begin{itemize}
  \tightlist
  \item
    \textbf{Peran TI}: NI memiliki kesadaran dan kepedulian etis
    terhadap dampak lingkungan, mendorong praktik berkelanjutan. AI
    memodelkan, memantau, dan mengoptimalkan penggunaan sumber daya dan
    dampak lingkungan (misalnya, efisiensi energi, pengurangan limbah).
    Kecerdasan Alam/Lingkungan menjadi sistem yang terkena dampak dan
    memberikan umpan balik (misalnya, perubahan iklim, penipisan sumber
    daya) atas semua aktivitas. CI juga tercermin dalam prioritas yang
    diberikan pada masalah lingkungan (misalnya, budaya keberlanjutan,
    norma spasial).
  \end{itemize}
\end{enumerate}

\section{\texorpdfstring{\textbf{6.5 Prinsip Penciptaan Nilai dan Sistem
Kompetitif
PSKV-S}}{6.5 Prinsip Penciptaan Nilai dan Sistem Kompetitif PSKV-S}}\label{prinsip-penciptaan-nilai-dan-sistem-kompetitif-pskv-s}

Prinsip penciptaan nilai dalam PSKV-S dapat diilustrasikan sebagai
berikut: pada awalnya, sebuah PSKV mungkin memiliki biaya produksi yang
lebih tinggi daripada nilai gunanya. Namun, melalui proses rekayasa dan
inovasi, nilai guna dapat ditingkatkan dan biaya dapat ditekan sehingga
tercipta ``berlian PSKV'', di mana nilai guna melebihi nilai ongkos.
Nilai yang tercipta ini kemudian dibagi dua oleh harga; bagian atas
diambil oleh pembeli, dan bagian bawah diambil oleh penjual. Ini
mencerminkan pemampatan nilai ke dalam bentuk-bentuk berbiaya rendah.

Pengembangan sistem kompetitif menggunakan konsep PSKV-S adalah tujuan
penting dalam TISE. Sebuah model sistem kompetitif PSKV-S terdiri dari
enam elemen utama: 1. Nilai-Nilai Inti (Core Values) 2. Kapasitas Sumber
Daya (Resource Capacity) 3. Kompetensi Inti (Core Competence) 4. Proses
Inti (Core Processes) 5. Proposisi Nilai Pelanggan (Customer Value
Proposition - CVP) 6. Daya Serap dan Keberlanjutan Ekonomi, Sosial, dan
Lingkungan (Economical, Social, and Environmental Sustainability)

Setiap PSKV yang ditawarkan ke pasar harus menjadi \textbf{Proposisi
Nilai Pelanggan (CVP)}. Ini berarti: 1. Pelanggan memiliki masalah. 2.
Masalah tersebut berupa pekerjaan penting yang harus dilakukan oleh
pelanggan. 3. Penjual menawarkan PSKV yang mampu menggantikan pelanggan
melakukan pekerjaan itu dengan imbalan finansial dengan harga terjangkau
dan murah (nilai guna melebihi harga) di mata pelanggan. 4. Penjual
mampu menghasilkan PSKV tersebut dengan biaya ongkos yang lebih rendah
dari harga.

PSKVE Engine adalah kerangka kerja untuk optimasi multi-objektif dalam
sistem sosio-teknis yang kompleks, mengatasi \emph{trade-off} antar
tujuan. Contoh kasusnya adalah bagaimana sebuah sistem PSKV dapat
efektif mengembangkan kekayaan nilai dalam kondisi iklim positif
(bullish) dan mempertahankan kekayaan dalam kondisi iklim negatif
(bearish).

\section{\texorpdfstring{\textbf{6.6 Model \emph{Virtual Prototyping}
dan Evolusi
PSKV-S}}{6.6 Model Virtual Prototyping dan Evolusi PSKV-S}}\label{model-virtual-prototyping-dan-evolusi-pskv-s}

Untuk merancang dan memvalidasi PSKVE Engine, TISE memanfaatkan model
komputasi PSV-S untuk \emph{virtual prototyping}. Ini memungkinkan
pengujian dan penyempurnaan artefak cerdas dalam lingkungan virtual
sebelum implementasi fisik.

Proses pengembangan PSV-S melalui \emph{virtual prototyping} biasanya
melibatkan beberapa tahapan atau versi: * \textbf{Version 0 (Ideas)}:
Persyaratan perilaku PSV-S dikembangkan dan disimulasikan dalam
lingkungan \emph{virtual prototyping}. Pada tahap ini, F-PSV-S dan PSV-S
koordinator bersifat operasional. * \textbf{Version 1 (Alpha)}: PSV-S
skala sangat kecil bersifat operasional, dengan I-PSV-S selesai, dan
sampel pelanggan mengonfirmasi nilai
{[}\(S]. Pada tahap ini juga mengonfirmasi model I-PSV-S, terutama biaya riil produksi [\)IS{]}
dan model pemasok. * \textbf{Version 2 (Beta)}: PSV-S skala kecil
bersifat operasional, dengan S-PSV-S selesai. Mengonfirmasi model
S-PSV-S, terutama biaya riil dan pendapatan riil produksi {[}\$S{]}, dan
model pelanggan. * \textbf{Version 3 (Release Candidate)}: PSV-S skala
menengah bersifat operasional. Mengonfirmasi model A-PSV-S, terutama
efisiensinya dengan skala, dan model penyedia aset. * \textbf{Version 4
(Final Release)}: PSV-S skala penuh bersifat operasional. Mengonfirmasi
skalabilitas semua PSV-S internal, serta model eksternal pemodal.

Model komputasi PSV-S juga menggambarkan struktur internal secara
rekursif, biasanya terdiri dari empat PSV-S internal dan enam mata uang
internal. Model ini juga mencakup model dinamika penawaran-permintaan
dari berbagai sumber. Mata uang yang digunakan dalam pertukaran ini
dapat beragam, seperti hak akses, \emph{man-hour}, poin, klaim,
honorarium, uang, kredit, dan komitmen.

\section{\texorpdfstring{\textbf{6.7 Studi Kasus: Sistem Rekomendasi
Makanan Sehat
(MSRS)}}{6.7 Studi Kasus: Sistem Rekomendasi Makanan Sehat (MSRS)}}\label{studi-kasus-sistem-rekomendasi-makanan-sehat-msrs}

Studi kasus \textbf{Sistem Rekomendasi Makanan Sehat (MSRS)} adalah
contoh sempurna penerapan TISE pada masalah sosio-teknis yang kompleks,
menunjukkan bagaimana TI, PUDAL, dan PSKVE bekerja sama. MSRS berfungsi
sebagai ekosistem \emph{marketplace} makanan sehat, menghubungkan
berbagai pemangku kepentingan.

Masalah yang dihadapi dalam konteks PSKVE adalah multi-dimensi: *
\textbf{Produk \& Layanan}: Sulit menghasilkan makanan yang sesuai
dengan profil kesehatan spesifik (misalnya, diabetes) dan selera lokal.
* \textbf{Nilai}: Makanan sehat seringkali tidak terjangkau karena
rantai pasok yang tidak efisien. * \textbf{Lingkungan}: Praktik
pertanian dan peternakan seringkali tidak berkelanjutan. *
\textbf{Pengetahuan}: Kurangnya data tentang kebutuhan konsumen dan
inovasi produk lokal yang sehat.

Solusi MSRS, yang berbasis Triune Intelligence, secara aktif mengelola
konversi PSKVE. Inovasi (Knowledge) dari chef diubah menjadi Product
baru, yang memberikan Service kesehatan kepada konsumen, yang
menghasilkan Value ekonomi bagi UMKM dan operator, semuanya dengan
tujuan menggunakan bahan baku yang berkelanjutan (Environmental).

\begin{center}\rule{0.5\linewidth}{0.5pt}\end{center}

\bookmarksetup{startatroot}

\chapter{\texorpdfstring{\textbf{Bab 7: Metodologi Validasi PICOC
Sistematis}}{Bab 7: Metodologi Validasi PICOC Sistematis}}\label{bab-7-metodologi-validasi-picoc-sistematis}

Tentu, berikut adalah draf Bab 7 dari buku Anda, yang membahas
Metodologi Validasi PICOC Sistematis:

\begin{center}\rule{0.5\linewidth}{0.5pt}\end{center}

Di Bab 4, kita telah menjelajahi kerangka kerja ASTF untuk dekomposisi
masalah, dan di Bab 5, kita memahami Triune-PUDAL Engine sebagai inti
kognitif artefak cerdas. Bab 6 membahas PSKVE Engine dan penciptaan
nilai holistik. Kini, kita beralih ke karakteristik keempat dari artefak
cerdas TISE: \textbf{Realistis (Realistic)}, yang diwujudkan melalui
kerangka kerja \textbf{PICOC Systematic}. Bab ini akan menguraikan
\textbf{kerangka kerja PICOC berlapis} sebagai pilar metodologis TISE
yang memastikan bahwa setiap klaim kontribusi didukung oleh bukti yang
kuat, sistematis, dan dapat diandalkan. Proses validasi berbasis PICOC
ini membentuk \textbf{Kaki Kiri Dalam (Inner Left Leg) dari W-Model},
yang berjalan secara paralel dengan proses realisasi dan integrasi.

\section{\texorpdfstring{\textbf{7.1 PICOC sebagai Fondasi Riset
Berbasis
Bukti}}{7.1 PICOC sebagai Fondasi Riset Berbasis Bukti}}\label{picoc-sebagai-fondasi-riset-berbasis-bukti}

\textbf{PICOC (Population, Intervention, Control, Outcome, Context)}
adalah sebuah kerangka kerja mnemonik yang berasal dari praktik
kedokteran berbasis bukti (\emph{evidence-based medicine}). Tujuannya
adalah untuk membantu peneliti merumuskan pertanyaan penelitian yang
jelas dan terfokus, yang pada gilirannya memandu pencarian literatur dan
desain studi secara sistematis.

Dalam konteks rekayasa TISE, komponen PICOC dapat diadaptasi sebagai
berikut:

\begin{itemize}
\tightlist
\item
  \textbf{P - Population/Problem/Process (Populasi/Masalah/Proses)}:
  Kelompok, entitas, sistem, atau masalah spesifik yang menjadi subjek
  penelitian. Dalam rekayasa, ini bisa berupa jenis perangkat lunak,
  proses manufaktur, atau populasi pengguna tertentu (misalnya, pengguna
  akhir dari sistem rumah cerdas, populasi kota untuk solusi lalu lintas
  cerdas).
\item
  \textbf{I - Intervention/Improvement/Investigation
  (Intervensi/Peningkatan/Investigasi)}: Solusi, metode, teknologi, atau
  pendekatan baru yang diusulkan dan sedang dievaluasi. Ini adalah
  ``kontribusi'' utama peneliti (misalnya, aplikasi atau solusi cerdas
  baru yang didukung AI).
\item
  \textbf{C - Control/Comparison (Kontrol/Pembanding)}: Solusi, metode,
  atau kondisi yang ada saat ini (\emph{baseline}) yang digunakan
  sebagai pembanding untuk mengukur keunggulan intervensi. Ini bisa
  berupa teknologi lama, praktik standar, atau ketiadaan intervensi
  (misalnya, aplikasi atau metode tradisional yang sudah ada).
\item
  \textbf{O - Outcome (Hasil)}: Efek atau hasil terukur yang digunakan
  untuk mengevaluasi dan membandingkan Intervensi dan Kontrol. Hasil ini
  harus kuantitatif atau kualitatif yang dapat diukur secara objektif
  (misalnya, peningkatan efisiensi, pengurangan kesalahan, peningkatan
  kepuasan pengguna, peningkatan kepuasan penduduk, lebih sedikit
  keluhan publik, kecepatan pemrosesan data yang lebih baik, keandalan
  sistem yang lebih tinggi, akurasi yang lebih tinggi).
\item
  \textbf{Cx - Context (Konteks)}: Kondisi di mana evaluasi dilakukan.
  Ini mencakup pengaturan lingkungan, asumsi, batasan, masalah spesifik,
  atau celah pengetahuan yang ditangani (misalnya, masalah pengiriman
  layanan kesehatan yang tidak efisien untuk pasien kronis dan kebutuhan
  pemantauan berkelanjutan yang dipersonalisasi).
\end{itemize}

\section{\texorpdfstring{\textbf{7.2 Penerapan PICOC di Setiap Lapisan
ASTF (PICOC
Berlapis)}}{7.2 Penerapan PICOC di Setiap Lapisan ASTF (PICOC Berlapis)}}\label{penerapan-picoc-di-setiap-lapisan-astf-picoc-berlapis}

Salah satu kontribusi metodologis paling orisinal TISE adalah penerapan
PICOC secara sistematis di setiap lapisan ASTF. Ini mengubah PICOC dari
alat tunggal menjadi kerangka validasi multi-lapis yang komprehensif.
Dengan kata lain, setiap kali kita melakukan pengujian atau validasi
pada sebuah artefak atau komponen TISE, kita dapat mendefinisikannya
sebagai eksperimen PICOC di lapisan ASTF yang relevan. Ini memastikan
artefak cerdas TISE \emph{realistis} dan \emph{dapat dipercaya} dalam
praktiknya.

Mari kita gunakan contoh \textbf{Sistem Komuter Cerdas Jakarta-Bandung}
(yang diperkenalkan di Bab 4) untuk mengilustrasikan penerapan PICOC di
setiap lapisan ASTF:

\begin{enumerate}
\def\labelenumi{\arabic{enumi}.}
\tightlist
\item
  \textbf{PICOC di Lapisan Aplikasi (PICOC(A))}:

  \begin{itemize}
  \tightlist
  \item
    \textbf{P (Populasi)}: Untuk penduduk di sebuah distrik kota cerdas,
    atau komuter Jakarta-Bandung.
  \item
    \textbf{I (Intervensi)}: Sistem manajemen limbah Triune yang
    menggunakan umpan balik berbasis CI tentang nilai estetika komunitas
    untuk menjadwalkan pengambilan sampah. Atau solusi baru ``berbagi
    kamar-makanan-perjalanan''.
  \item
    \textbf{C (Kontrol)}: Dibandingkan dengan sistem berbasis AI murni
    yang hanya mengoptimalkan efisiensi bahan bakar. Atau solusi lama
    seperti bepergian dengan mobil pribadi tanpa fasilitas terintegrasi.
  \item
    \textbf{O (Hasil)}: Mengarah pada kepuasan penduduk yang dilaporkan
    lebih tinggi dan keluhan publik yang lebih sedikit. Atau peningkatan
    kinerja yang dirasakan pengguna (misalnya, pengurangan waktu stres,
    peningkatan produktivitas, kepuasan).
  \item
    \textbf{Cx (Konteks)}: Dalam konteks mempertahankan kendala anggaran
    kota. Atau konteks masalah komuter jarak jauh di kota metropolitan.
  \end{itemize}
\item
  \textbf{PICOC di Lapisan Sistem (PICOC(S))}:

  \begin{itemize}
  \tightlist
  \item
    \textbf{P (Populasi)}: Set data simulasi lalu lintas dan permintaan
    pengguna. Atau \emph{testbed} data untuk model AI, lingkungan urban
    simulasi untuk sistem kontrol lalu lintas, \emph{testbed
    hardware-in-the-loop}.
  \item
    \textbf{I (Intervensi)}: Sistem \emph{Commuter-Sleep-in-Capsule}
    yang diusulkan. Atau arsitektur sistem cerdas yang diusulkan
    (misalnya, jaringan sensor terdistribusi dengan pemrosesan AI di
    \emph{edge}).
  \item
    \textbf{C (Kontrol)}: Sistem transportasi yang ada (misalnya, sistem
    kereta api konvensional). Atau sistem pemrosesan data terpusat atau
    versi sebelumnya dari sistem.
  \item
    \textbf{O (Hasil)}: Peningkatan kinerja sistem (misalnya,
    \emph{throughput} penumpang per jam, utilisasi sumber daya, latensi
    layanan). Atau kecepatan pemrosesan data yang lebih baik, latensi
    yang lebih rendah, keandalan sistem yang lebih tinggi, akurasi model
    AI yang lebih baik.
  \item
    \textbf{Cx (Konteks)}: Kebutuhan akan sistem terintegrasi yang
    memenuhi persyaratan dari lapisan Aplikasi. Atau kebutuhan akan
    sistem pemrosesan data \emph{real-time} yang andal untuk aplikasi
    kesehatan, yang mampu menangani beragam \emph{input} sensor.
  \end{itemize}
\item
  \textbf{PICOC di Lapisan Teknologi (PICOC(T))}:

  \begin{itemize}
  \tightlist
  \item
    \textbf{P (Populasi)}: Bahan makanan mentah. Atau data sensor mentah
    (misalnya, sinyal ECG, tingkat aktivitas), sumber energi untuk modul
    daya.
  \item
    \textbf{I (Intervensi)}: Dapur otomatis berbasis AI. Atau algoritma
    AI baru untuk deteksi anomali pada sinyal ECG, teknologi komunikasi
    \emph{low-power} baru.
  \item
    \textbf{C (Kontrol)}: Dapur komersial konvensional. Atau algoritma
    deteksi anomali yang ada, komunikasi Bluetooth standar.
  \item
    \textbf{O (Hasil)}: Peningkatan kinerja teknologi (misalnya,
    kecepatan persiapan makanan per unit, konsistensi kualitas,
    efisiensi energi). Atau akurasi yang lebih tinggi dan tingkat
    \emph{false positive} yang lebih rendah untuk deteksi anomali,
    peningkatan masa pakai baterai karena konsumsi daya yang lebih
    rendah.
  \item
    \textbf{Cx (Konteks)}: Tantangan untuk menyediakan makanan
    berkualitas tinggi dalam volume besar secara cepat. Atau tantangan
    untuk mendeteksi anomali jantung yang halus dari data sensor yang
    bising dengan penggunaan daya minimal.
  \end{itemize}
\item
  \textbf{PICOC di Lapisan Riset Fundamental (PICOC(F))}:

  \begin{itemize}
  \tightlist
  \item
    \textbf{P (Populasi)}: Fenomena alokasi sumber daya dalam sistem
    yang kompleks. Atau fenomena propagasi sinyal dalam jaringan
    biologis, batas teoretis kompresi data.
  \item
    \textbf{I (Intervensi)}: Model optimasi baru berdasarkan teori
    konversi PSKVE. Atau model matematika baru yang menggambarkan
    atenuasi sinyal, pendekatan \emph{information-theoretic} baru untuk
    kompresi data.
  \item
    \textbf{C (Kontrol)}: Algoritma alokasi sumber daya standar
    (misalnya, optimasi linear). Atau model atenuasi sinyal yang ada,
    teori kompresi yang sudah mapan.
  \item
    \textbf{O (Hasil)}: Pengetahuan baru (misalnya, pemahaman yang lebih
    dalam tentang \emph{trade-off} antar dimensi nilai, validasi teori
    konversi). Atau model prediktif yang lebih akurat, teorema baru yang
    menetapkan batasan yang lebih ketat pada kompresi.
  \item
    \textbf{Cx (Konteks)}: Kesenjangan pengetahuan dalam teori optimasi
    untuk sistem sosio-teknis multi-dimensi. Atau kesenjangan
    pengetahuan dalam memahami bagaimana faktor fisiologis spesifik
    memengaruhi kualitas sinyal ECG, pencarian metode transmisi data
    yang lebih efisien.
  \end{itemize}
\end{enumerate}

\section{\texorpdfstring{\textbf{7.3 Membangun Argumen yang Kuat Melalui
PICOC
Berlapis}}{7.3 Membangun Argumen yang Kuat Melalui PICOC Berlapis}}\label{membangun-argumen-yang-kuat-melalui-picoc-berlapis}

Dengan menerapkan PICOC di setiap lapisan, seorang peneliti dapat
membangun sebuah argumen yang sangat kuat dan koheren. Hasil dari satu
lapisan menjadi bukti pendukung untuk lapisan di atasnya, menciptakan
\textbf{rantai bukti} yang logis dan tidak terbantahkan. Rantai bukti
ini, yang menghubungkan dari riset fundamental hingga dampak pada
pengguna, adalah inti dari kekuatan metodologis TISE.

Sebagai contoh: 1. Peneliti membuktikan di \textbf{lapisan F} bahwa
teori konversi PSKVE-nya menghasilkan alokasi sumber daya yang lebih
seimbang (O(F)). 2. Pengetahuan ini digunakan untuk merancang dapur
otomatis di \textbf{lapisan T} yang terbukti lebih efisien dalam
menggunakan energi dan bahan baku (O(T)). 3. Dapur otomatis ini kemudian
diintegrasikan ke dalam \textbf{lapisan S}, dan simulasi menunjukkan
bahwa sistem secara keseluruhan memiliki biaya operasional yang lebih
rendah dan limbah yang lebih sedikit (O(S)). 4. Akhirnya, di
\textbf{lapisan A}, solusi ini ditawarkan kepada pengguna, dan survei
menunjukkan bahwa harga yang lebih terjangkau dan aspek keberlanjutan
membuat solusi ini lebih disukai (O(A)).

\section{\texorpdfstring{\textbf{7.4 Peran Triune Intelligence dalam
Memperkuat Ketelitian
PICOC}}{7.4 Peran Triune Intelligence dalam Memperkuat Ketelitian PICOC}}\label{peran-triune-intelligence-dalam-memperkuat-ketelitian-picoc}

Integrasi Triune Intelligence (NI, CI, AI) secara fundamental
meningkatkan ketelitian dan komprehensivitas validasi PICOC. Setiap
pilar kecerdasan memiliki peran krusial dalam memastikan validasi yang
kuat dan bermakna:

\begin{itemize}
\tightlist
\item
  \textbf{Peran AI}: Kecerdasan Buatan mendukung pembuatan pengujian
  otomatis (misalnya, menghasilkan skenario atau kasus ekstrem), deteksi
  anomali dalam hasil pengujian, dan simulasi berbasis AI untuk skenario
  yang kompleks atau berbahaya. AI juga dapat menganalisis data untuk
  memilih \emph{baseline} yang realistis atau mensimulasikan hasil untuk
  perbandingan.
\item
  \textbf{Peran NI (Natural Intelligence)}: Kecerdasan manusia terlibat
  dalam menafsirkan hasil pengujian, melakukan pengujian eksploratif,
  dan memberikan penilaian akhir dalam validasi, terutama untuk kepuasan
  pemangku kepentingan. NI juga berperan dalam merumuskan pertanyaan
  PICOC yang bermakna, karena seringkali keputusan ini bergantung pada
  penilaian manusia dan pengetahuan domain.
\item
  \textbf{Peran CI (Cultural Intelligence)}: Kecerdasan Budaya
  memastikan bahwa validasi mencakup penerimaan pengguna, pemeriksaan
  kepatuhan etis, dan penilaian adaptabilitas budaya, dengan memasukkan
  studi pengguna dan umpan balik dari budaya target. CI memastikan bahwa
  konteks dan populasi dipahami dalam istilah manusia dan budaya.
\end{itemize}

Dengan demikian, TISE mengintegrasikan ketiga kecerdasan ini dalam
setiap tahapan validasi, memastikan bahwa artefak yang dibangun tidak
hanya berfungsi secara teknis, tetapi juga selaras dengan nilai-nilai
dan kebutuhan manusia serta batasan alam.

\section{\texorpdfstring{\textbf{7.5 Tantangan Verifikasi dan Validasi
dalam System of Autonomous Systems
(SoAS)}}{7.5 Tantangan Verifikasi dan Validasi dalam System of Autonomous Systems (SoAS)}}\label{tantangan-verifikasi-dan-validasi-dalam-system-of-autonomous-systems-soas}

Tantangan dalam memverifikasi dan memvalidasi Sistem dari Sistem (SoS)
karena perilaku evolusioner dan perubahan persyaratan juga berlaku untuk
\emph{System of Autonomous Systems} (SoAS) sebagai perpanjangan dari
SoS. Tantangan ini dapat memburuk ketika dipengaruhi oleh
masalah-masalah khusus yang ditimbulkan oleh penggunaan sistem berbasis
AI. SoAS dapat diberi peringkat lebih tinggi dalam hal Level Otonomi
(LoA) jika sistemnya mampu melakukan lebih banyak tugas misi secara
mandiri. Dalam hal ini, SoAS menggunakan algoritma kerja sama AI,
membuat sistem berkolaborasi dalam melakukan tugas-tugas misi. Namun,
mengintegrasikan otonomi ke dalam operasi SoS dapat memengaruhi kinerja
manusia. Penelitian menunjukkan bahwa LoA yang lebih rendah menghasilkan
kinerja sistem yang lebih baik, tetapi merugikan kinerja pengguna.
Dengan demikian, LoA sebagai faktor kompleksitas memiliki dampak
signifikan pada kinerja operator dan sistem, yang pada akhirnya
memengaruhi kinerja SoAS.

TISE, dengan kerangka PICOC berlapis dan penekanan pada Triune
Intelligence, menyediakan pendekatan yang lebih tangguh untuk
memverifikasi dan memvalidasi sistem-sistem kompleks ini, memastikan
bahwa aspek manusia dan budaya diperhitungkan di samping kinerja teknis.

\begin{center}\rule{0.5\linewidth}{0.5pt}\end{center}

\bookmarksetup{startatroot}

\chapter{\texorpdfstring{\textbf{Bab 8: W-Model untuk Rekayasa Cerdas:
Mengintegrasikan Desain, Sintesis, dan Validasi
Berkelanjutan}}{Bab 8: W-Model untuk Rekayasa Cerdas: Mengintegrasikan Desain, Sintesis, dan Validasi Berkelanjutan}}\label{bab-8-w-model-untuk-rekayasa-cerdas-mengintegrasikan-desain-sintesis-dan-validasi-berkelanjutan}

Tentu, berikut adalah draf Bab 8 dari buku Anda, yang membahas
Metodologi V-Method, khususnya W-Model, untuk Rekayasa Cerdas dalam
paradigma TISE, dengan mengacu pada semua sumber yang diberikan dan
riwayat percakapan kita:

\begin{center}\rule{0.5\linewidth}{0.5pt}\end{center}

Di Bab 4, kita telah menjelajahi kerangka kerja ASTF untuk dekomposisi
masalah, di Bab 5 kita mendalami Triune-PUDAL Engine sebagai inti
kognitif, Bab 6 membahas PSKVE Engine dan penciptaan nilai, serta Bab 7
menguraikan Metodologi Validasi PICOC Sistematis. Kini, kita akan
beralih ke karakteristik keenam dari artefak cerdas TISE:
\textbf{Metodis (Methodic)}, yang diwujudkan melalui \textbf{V-Method}.

Dalam TISE, jika ASTF adalah peta anatomi masalah, dan PICOC adalah
kerangka validasinya, maka \textbf{W-Model} adalah diagram alur proses
rekayasa itu sendiri. W-Model merupakan pengembangan dari V-Model
tradisional, yang dirancang untuk menekankan verifikasi dan validasi
berkelanjutan di seluruh proses pengembangan, serta mengintegrasikan
definisi masalah dan penciptaan solusi secara lebih eksplisit. Bentuk
`W' yang khas secara visual terdiri dari empat kaki yang saling
berhubungan, menyediakan kerangka kerja yang lebih komprehensif untuk
sistem kompleks yang melibatkan Kecerdasan Buatan (AI). W-Model
memastikan \textbf{keterlacakan} dalam pengembangan sistem.

\section{\texorpdfstring{\textbf{8.1 Prinsip V-Method: Keterlacakan dan
Kelayakan}}{8.1 Prinsip V-Method: Keterlacakan dan Kelayakan}}\label{prinsip-v-method-keterlacakan-dan-kelayakan}

Pada intinya, \textbf{V-Method} adalah model siklus hidup pengembangan
sistem berbentuk V yang secara sistematis menghubungkan setiap fase
pengembangan dengan fase pengujian, verifikasi, dan validasi yang
sesuai. Prinsip utamanya adalah memastikan \textbf{keterlacakan}
(\emph{traceability}) -- setiap persyaratan harus dapat dilacak ke
desain, implementasi, dan pengujiannya. Ini menjamin bahwa setiap
artefak rekayasa TISE adalah \textbf{Dapat Dilaksanakan (Doable)} dan
\textbf{Metodis (Methodic)}.

Namun, V-Model tradisional seringkali dianggap terlalu linier dan kurang
mampu menangani kompleksitas sistem modern, terutama yang melibatkan AI.
Oleh karena itu, TISE mengadopsi dan memperluasnya menjadi
\textbf{W-Model}.

\section{\texorpdfstring{\textbf{8.2 Empat Kaki W-Model: Desain,
Pengembangan, dan Validasi
Triune}}{8.2 Empat Kaki W-Model: Desain, Pengembangan, dan Validasi Triune}}\label{empat-kaki-w-model-desain-pengembangan-dan-validasi-triune}

W-Model memperluas V-Model dengan menambahkan dua ``kaki dalam'' yang
merepresentasikan proses internal desain dan validasi yang seringkali
implisit dalam model yang lebih sederhana. Proses internal ini
seringkali bersifat iteratif, berbeda dengan model air terjun yang kaku.
W-Model secara eksplisit mengakomodasi siklus iteratif (seperti
``mini-V'' atau ``sprint agile'') dalam fase-fase utamanya, yang krusial
untuk pengembangan komponen AI.

Berikut adalah empat kaki W-Model dan bagaimana mereka diintegrasikan
dengan kerangka TISE lainnya:

\begin{enumerate}
\def\labelenumi{\arabic{enumi}.}
\tightlist
\item
  \textbf{Kaki Kiri Luar: Dekomposisi dan Definisi (Alur ASTF)}

  \begin{itemize}
  \tightlist
  \item
    \textbf{Fokus}: Kaki ini identik dengan sisi kiri V-Model, bergerak
    turun dari kebutuhan tingkat tinggi ke spesifikasi detail. Ini
    mengoperasionalkan dekomposisi hierarkis kerangka \textbf{ASTF
    (Application, System, Technology, Fundamental Research)}:

    \begin{itemize}
    \tightlist
    \item
      \textbf{A (Aplikasi)}: Mendefinisikan kebutuhan pemangku
      kepentingan, tujuan bisnis, dan konteks masalah secara
      keseluruhan. Fase ini sangat dipengaruhi oleh \textbf{Kecerdasan
      Kultural (CI)} untuk menangkap \textbf{``MENGAPA''} dari sistem,
      termasuk nilai-nilai manusia dan hasil PSKVE yang diinginkan.
    \item
      \textbf{S (Sistem)}: Menerjemahkan kebutuhan aplikasi menjadi
      arsitektur sistem yang koheren.
    \item
      \textbf{T (Teknologi)}: Mengembangkan atau memilih teknologi kunci
      yang menjadi komponen pembangun sistem.
    \item
      \textbf{F (Riset Fundamental)}: Menemukan atau memvalidasi
      prinsip-prinsip ilmiah atau pengetahuan dasar yang mendasari
      teknologi.
    \end{itemize}
  \item
    \textbf{Peran TI}: Dalam fase ini, CI sangat dominan dalam
    menentukan persyaratan tingkat tinggi yang selaras dengan
    nilai-nilai dan budaya, sementara NI membantu dalam
    mengartikulasikan kebutuhan yang kompleks.
  \end{itemize}
\item
  \textbf{Kaki Kanan Dalam: Proses Desain dan Sintesis Internal}

  \begin{itemize}
  \tightlist
  \item
    \textbf{Fokus}: Kaki ini berjalan paralel dengan Kaki Kiri Luar. Ini
    merepresentasikan proses internal dan iteratif di mana desain untuk
    setiap lapisan (A, S, T, F) dibuat dan disempurnakan. Proses desain
    internal ini dapat dikonseptualisasikan sebagai kolaborasi
    \textbf{``Mesin Triune-PUDAL''}, yang merupakan arena operasional
    utama bagi Triune Intelligence dalam memecahkan masalah.
  \item
    \textbf{Peran Triune Intelligence}:

    \begin{itemize}
    \tightlist
    \item
      \textbf{AI}: Menyediakan kekuatan komputasi untuk pemodelan
      canggih, simulasi berbasis AI untuk memprediksi kinerja, dan
      bahkan membantu dalam pengkodean. Misalnya, AI dapat membantu
      dalam mengoptimalkan jadwal waktu atau alokasi sumber daya dalam
      simulasi.
    \item
      \textbf{NI (Kecerdasan Alami/Manusia)}: Insinyur, ilmuwan data,
      dan peneliti manusia memandu kreativitas, pemecahan masalah, dan
      pengambilan keputusan selama proses desain. NI menyuntikkan
      kontribusi orisinal dan penalaran.
    \item
      \textbf{CI (Kecerdasan Kultural)}: Memengaruhi keputusan desain
      agar selaras dengan nilai-nilai budaya, batasan etis, dan
      pertimbangan sosial (misalnya, masalah privasi saat menentukan
      sensor, atau kebutuhan akan AI yang dapat dijelaskan untuk
      membangun kepercayaan pengguna). CI menentukan ``mengapa itu
      penting''.
    \end{itemize}
  \end{itemize}
\item
  \textbf{Kaki Kanan Luar: Realisasi, Integrasi, dan Validasi (Alur
  FTSA)}

  \begin{itemize}
  \tightlist
  \item
    \textbf{Fokus}: Kaki ini identik dengan sisi kanan V-Model, bergerak
    naik dari implementasi di tingkat fundamental hingga penerapan di
    tingkat aplikasi. Kaki ini merepresentasikan proses pembangunan dan
    pengujian, mengikuti alur \textbf{FTSA (Fundamental, Teknologi,
    Sistem, Aplikasi)}:

    \begin{itemize}
    \tightlist
    \item
      \textbf{F (Validasi Riset Fundamental)}: Memvalidasi pengetahuan,
      prinsip, atau model baru yang menjadi dasar ilmiah untuk inovasi
      teknologi.
    \item
      \textbf{T (Pengembangan \& Verifikasi Teknologi)}: Mengembangkan
      komponen teknologi individu dan memverifikasinya terhadap
      spesifikasi desainnya.
    \item
      \textbf{S (Integrasi \& Verifikasi/Validasi Sistem)}:
      Mengintegrasikan teknologi yang telah diverifikasi ke dalam sistem
      Triune yang lengkap, memverifikasinya terhadap spesifikasi, dan
      memvalidasi bahwa kolaborasi NI-CI-AI berfungsi seperti yang
      diharapkan.
    \item
      \textbf{A (Penerapan \& Uji Penerimaan Aplikasi)}: Menerapkan
      sistem yang telah divalidasi untuk memberikan solusi aplikasi,
      yang menjalani uji penerimaan dengan pemangku kepentingan untuk
      memastikan sistem memenuhi kebutuhan yang ditentukan oleh CI dalam
      konteks dunia nyata.
    \end{itemize}
  \item
    \textbf{Peran TI}: AI adalah penggerak utama dalam implementasi
    teknis dan eksekusi, sementara NI dan CI memberikan panduan dan
    konteks untuk memastikan sistem dibangun sesuai tujuan dan nilai.
  \end{itemize}
\item
  \textbf{Kaki Kiri Dalam: Validasi Berbasis PICOC}

  \begin{itemize}
  \tightlist
  \item
    \textbf{Fokus}: Kaki ini berjalan paralel dengan Kaki Kanan Luar,
    bergerak naik saat komponen diintegrasikan dan diuji. Kaki ini
    merepresentasikan validasi yang ketat dan berbasis bukti dari
    \emph{output} yang direalisasikan di Kaki Kanan Luar. Proses
    validasi berbasis PICOC ini membentuk \textbf{Kaki Kiri Dalam (Inner
    Left Leg) dari W-Model}.
  \item
    \textbf{Penerapan PICOC}: Kerangka kerja \textbf{PICOC (Population,
    Intervention, Control, Outcome, Context)} diterapkan di setiap tahap
    yang selaras dengan ASTF untuk mendefinisikan tujuan, intervensi,
    perbandingan, hasil, dan konteks yang spesifik untuk tahap tersebut.
    Setiap pengujian dapat dijelaskan sebagai sebuah eksperimen PICOC.
  \item
    \textbf{Peran Triune Intelligence}:

    \begin{itemize}
    \tightlist
    \item
      \textbf{AI}: Mendukung pembuatan pengujian otomatis (misalnya,
      menghasilkan skenario atau kasus ekstrem), deteksi anomali dalam
      hasil pengujian, dan simulasi berbasis AI untuk skenario yang
      kompleks atau berbahaya.
    \item
      \textbf{NI}: Kecerdasan manusia terlibat dalam menafsirkan hasil
      pengujian, melakukan pengujian eksploratif, dan memberikan
      penilaian akhir dalam validasi, terutama untuk kepuasan pemangku
      kepentingan.
    \item
      \textbf{CI}: Memastikan bahwa validasi mencakup penerimaan
      pengguna, pemeriksaan kepatuhan etis, dan penilaian adaptabilitas
      budaya, dengan memasukkan studi pengguna dan umpan balik dari
      budaya target.
    \end{itemize}
  \end{itemize}
\end{enumerate}

\section{\texorpdfstring{\textbf{8.3 W-Model dan Pengelolaan Otonomi
dalam System of Autonomous Systems
(SoAS)}}{8.3 W-Model dan Pengelolaan Otonomi dalam System of Autonomous Systems (SoAS)}}\label{w-model-dan-pengelolaan-otonomi-dalam-system-of-autonomous-systems-soas}

Integrasi otonomi ke dalam operasi \emph{System of Systems} (SoS) telah
meningkatkan kapabilitas otonom, namun dapat berdampak pada kinerja
manusia. Penelitian menunjukkan bahwa \emph{Level of Autonomy} (LoA)
yang lebih rendah dapat menghasilkan kinerja sistem yang lebih baik,
tetapi justru merugikan kinerja pengguna. Hal ini mengindikasikan bahwa
LoA, sebagai faktor kompleksitas, memiliki dampak signifikan pada
kinerja operator dan sistem, yang pada akhirnya memengaruhi kinerja
\emph{System of Autonomous Systems} (SoAS).

W-Model, dengan penekanannya pada integrasi Triune Intelligence di
setiap kaki prosesnya, sangat relevan untuk mengatasi tantangan ini.
Dengan secara eksplisit mempertimbangkan peran NI dan CI dalam desain,
pengembangan, dan validasi, W-Model TISE membantu memastikan bahwa
sistem cerdas tidak hanya optimal secara teknis tetapi juga selaras
dengan kemampuan dan kebutuhan manusia. Ini mendukung pengembangan SoAS
yang menyeimbangkan otonomi dengan pemberdayaan manusia, menghindari
dampak negatif pada kinerja operator, dan mengoptimalkan kinerja SoAS
secara keseluruhan.

Dengan mengintegrasikan proses desain dan validasi internal secara
eksplisit melalui kaki dalamnya, dan dengan menenun peran Triune
Intelligence secara mendalam, W-Model memastikan bahwa solusi rekayasa
tidak hanya sehat secara teknis tetapi juga relevan, efektif, dan
diterima dalam konteks manusia dan masyarakat di dunia nyata. Ini adalah
metodologi yang kokoh untuk mengembangkan artefak cerdas yang kuat,
cerdas, berjangkauan luas, realistis, dapat dilaksanakan, dan metodis.

\begin{center}\rule{0.5\linewidth}{0.5pt}\end{center}

\bookmarksetup{startatroot}

\chapter{\texorpdfstring{\textbf{Bab 9: Struktur Disertasi dan Publikasi
Ilmiah Berbasis
TISE}}{Bab 9: Struktur Disertasi dan Publikasi Ilmiah Berbasis TISE}}\label{bab-9-struktur-disertasi-dan-publikasi-ilmiah-berbasis-tise}

Tentu, berikut adalah draf Bab 9 dari buku Anda, yang membahas Struktur
Disertasi dan Publikasi Ilmiah Berbasis TISE, dengan mengacu pada semua
sumber yang diberikan dan riwayat percakapan kita:

\begin{center}\rule{0.5\linewidth}{0.5pt}\end{center}

Setelah memahami metodologi inti TISE, termasuk dekomposisi masalah
melalui ASTF (Bab 4), siklus kognitif Triune-PUDAL (Bab 5), penciptaan
nilai holistik melalui PSKVE (Bab 6), metodologi validasi PICOC
Sistematis (Bab 7), serta kerangka kerja W-Model (Bab 8), kini kita
beralih ke aplikasi praktis dari semua konsep ini dalam ranah akademik:
\textbf{penulisan disertasi dan publikasi ilmiah}. Bab ini akan menjadi
panduan bagi mahasiswa dan peneliti untuk menyusun karya ilmiah yang
tidak hanya kuat secara metodologis tetapi juga koheren, berdampak, dan
mampu mengkomunikasikan kontribusi orisinal dalam kerangka TISE.

\subsection{\texorpdfstring{\textbf{9.1 Prinsip Penulisan
Disertasi/Paper Ilmiah Berbasis
TISE}}{9.1 Prinsip Penulisan Disertasi/Paper Ilmiah Berbasis TISE}}\label{prinsip-penulisan-disertasipaper-ilmiah-berbasis-tise}

Prinsip utama penulisan disertasi atau paper ilmiah dalam paradigma TISE
adalah \textbf{mengajukan (advancing) sebuah cara/titik pandang
(pengetahuan) yang baru, yang berdasarkan dari hasil riset yang valid}.
Pengetahuan baru ini harus berdasarkan pertumbuhan suatu rumpun keilmuan
dan ditulis dalam bahasa rumpun keilmuan tersebut. Proses riset itu
sendiri adalah sebuah metode untuk memperoleh pengetahuan yang sahih
(valid).

Dalam riset rekayasa, terutama yang berorientasi ilmu rekayasa
(eksperimental atau teoretis), tesis berupaya menghasilkan jawaban
terhadap pertanyaan mengenai realitas dunia dari perspektif keilmuan.
Ini bisa berupa pertanyaan eksperimental (apakah prediksi teori akurat?)
atau pertanyaan teoretis (teori apa yang memadai untuk menjelaskan
fenomena?). Tesis harus menjawab: apa rumusan masalah, mengapa jawaban
sebelumnya tidak memadai, apa jawaban peneliti, dan seberapa baik
jawaban peneliti dibandingkan dengan yang sudah ada.

TISE mengidentifikasi tiga jenis hipotesis utama dalam riset rekayasa
yang dapat menjadi fokus kontribusi: 1. \textbf{Hipotesis Solutif}:
Menghasilkan solusi terbaik untuk suatu kelas kasus penting. Ini
berurusan dengan masalah masyarakat umum (misalnya, masalah
pendidikan/pembelajaran, kinerja pembelajar) dan bagaimana menggunakan
solusi khusus (teknologi) dengan asumsi solusi khusus memiliki kemampuan
tertentu. 2. \textbf{Hipotesis Aplikatif}: Berfokus pada strategi dan
prosedur penerapan prinsip-prinsip untuk menghasilkan solusi khusus. 3.
\textbf{Hipotesis Teoretis}: Menyatakan adanya prinsip-prinsip
(misalnya, komputasi/elektrikal) baru yang menjadi dasar untuk mengatasi
masalah pembuatan ``solusi khusus''.

Disertasi adalah tulisan ilmiah yang memperlihatkan (i) penguasaan
subjek/bidang dan (ii) kontribusi pengetahuan hasil riset, di mana riset
membuktikan hipotesis/hipotesa yang efektif untuk menjawab
permasalahan/pertanyaan.

\subsection{\texorpdfstring{\textbf{9.2 Panduan Proses Riset Disertasi
dengan TISE: Delapan Tonggak
Pencapaian}}{9.2 Panduan Proses Riset Disertasi dengan TISE: Delapan Tonggak Pencapaian}}\label{panduan-proses-riset-disertasi-dengan-tise-delapan-tonggak-pencapaian}

Proses riset doktoral dalam kerangka TISE dapat dipecah menjadi delapan
tonggak pencapaian (\emph{milestones}) yang logis, mulai dari menemukan
topik hingga ujian akhir:

\begin{itemize}
\tightlist
\item
  \textbf{Milestone 1: Menemukan Topik}

  \begin{itemize}
  \tightlist
  \item
    \textbf{Hasil yang Diharapkan}: Dokumen Definisi Masalah dan Ruang
    Lingkup.
  \item
    \textbf{Fokus TISE}: Fase ini sepenuhnya berada di \textbf{Lapisan
    Aplikasi (A)}. Tugas pertama adalah mengidentifikasi masalah manusia
    atau masyarakat yang signifikan dan mendesak, menggunakan
    \textbf{Kecerdasan Manusia (Homocordium)} Anda. Tentukan apakah
    ruang lingkup masalah cukup fokus untuk dapat diselesaikan dalam
    jangka waktu disertasi.
  \end{itemize}
\item
  \textbf{Milestone 2: Studi Literatur}

  \begin{itemize}
  \tightlist
  \item
    \textbf{Hasil yang Diharapkan}: Dokumen Analisis Solusi yang Ada
    (Mesin Eksisting) dan Peta ASTF-nya.
  \item
    \textbf{Fokus TISE}: Fase ini adalah tentang membangun
    \textbf{Kontrol (C)} untuk kerangka PICOC Anda di semua lapisan.
    Identifikasi dan analisis solusi-solusi yang sudah ada, kemudian
    bedah dan petakan arsitektur ASTF-nya (aplikasi (A), sistem (S),
    teknologi (T), dan dasar ilmiahnya (F)). Evaluasi kapasitas PUDAL
    dan PSKVE mereka yang ada, serta identifikasi kelemahan atau
    kesenjangan penelitian (\emph{research gap}) di setiap lapisan ASTF
    dari solusi yang ada sebagai justifikasi untuk intervensi Anda.
  \end{itemize}
\item
  \textbf{Milestone 3: Mengajukan Proyek Riset}

  \begin{itemize}
  \tightlist
  \item
    \textbf{Hasil yang Diharapkan}: Proposal Riset yang berisi Ide Mesin
    Baru, Peta ASTF, dan Desain Kaki Kiri Luar W-Model.
  \item
    \textbf{Fokus TISE}: Ini adalah fase desain inti, di mana Anda
    mendefinisikan \textbf{Intervensi (I)} untuk kerangka PICOC Anda.
    Usulkan ``mesin cerdas baru'' Anda berdasarkan kesenjangan yang
    ditemukan, jelaskan bagaimana mesin ini akan memiliki enam
    karakteristik TISE, dan buat peta ASTF yang jelas untuk intervensi
    Anda. Rancang kaki kiri luar W-Model dengan mendekomposisikan
    kebutuhan dari lapisan A turun ke S, T, dan F. Rumuskan hipotesis
    atau pertanyaan penelitian untuk setiap lapisan menggunakan format
    PICOC.
  \end{itemize}
\item
  \textbf{Milestone 4 \& 5: Kemajuan 1 \& 2 (Pengembangan dan Validasi
  Awal)}

  \begin{itemize}
  \tightlist
  \item
    \textbf{Hasil yang Diharapkan}: \emph{Testbed} ASTF yang berfungsi;
    Kaki Kanan Luar W-Model setengah selesai dan tervalidasi.
  \item
    \textbf{Fokus TISE}: Fase ini adalah tentang implementasi dan
    pengujian, dimulai dari bawah ke atas
    (F-\textgreater T-\textgreater S). Bangun \emph{testbed} atau
    lingkungan simulasi untuk memvalidasi hipotesis Anda di Lapisan
    Fundamental dan Teknologi, melakukan eksperimen sesuai desain
    PICOC(F) dan PICOC(T) Anda. Ini adalah awal dari pendakian di kaki
    kanan luar W-Model, yang divalidasi secara paralel oleh kaki kiri
    dalam W-Model.
  \end{itemize}
\item
  \textbf{Milestone 6: Kemajuan 3 (Integrasi dan Validasi Komprehensif)}

  \begin{itemize}
  \tightlist
  \item
    \textbf{Hasil yang Diharapkan}: Kaki Kanan Luar W-Model selesai dan
    tervalidasi sepenuhnya.
  \item
    \textbf{Fokus TISE}: Integrasikan teknologi Anda (T) ke dalam sistem
    (S) dan uji kinerjanya menggunakan PICOC(S). Selanjutnya, uji sistem
    Anda dalam konteks aplikasi (A) untuk memvalidasi dampaknya pada
    pemangku kepentingan menggunakan PICOC(A). Ini menyelesaikan
    pendakian di kaki kanan luar W-Model, yang divalidasi secara ketat
    di setiap langkah oleh kaki kiri dalam.
  \end{itemize}
\item
  \textbf{Milestone 7 \& 8: Menulis dan Ujian Akhir}

  \begin{itemize}
  \tightlist
  \item
    \textbf{Hasil yang Diharapkan}: Manuskrip disertasi yang siap dan
    kelulusan.
  \item
    \textbf{Fokus TISE}: Dokumentasikan seluruh perjalanan metodis Anda.
    Tunjukkan penguasaan Anda atas paradigma TISE dengan menceritakan
    kisah penelitian Anda secara koheren. Saat ujian, Anda bukan hanya
    mempertahankan hasil, tetapi juga mempertahankan validitas proses
    rekayasa yang telah Anda jalani. Anda adalah seorang
    ``Vokator''---seseorang yang menyuarakan kebenaran yang ditemukan
    melalui riset yang bertanggung jawab.
  \end{itemize}
\end{itemize}

\subsection{\texorpdfstring{\textbf{9.3 Rekomendasi Struktur Penulisan
Disertasi Teknik Menggunakan Kerangka
TISE}}{9.3 Rekomendasi Struktur Penulisan Disertasi Teknik Menggunakan Kerangka TISE}}\label{rekomendasi-struktur-penulisan-disertasi-teknik-menggunakan-kerangka-tise}

Struktur disertasi yang mengikuti paradigma TISE akan secara alami
menjadi kuat, logis, dan mudah diikuti. Kerangka ASTF dan PICOC berlapis
menyediakan ``tulang punggung'' untuk narasi Anda. Berikut adalah
templat struktur bab per bab yang direkomendasikan:

\begin{itemize}
\tightlist
\item
  \textbf{Judul Disertasi}: Harus ringkas, informatif, dan mencerminkan
  kontribusi inti.
\item
  \textbf{Abstrak}: Gunakan struktur 6 poin yang efektif untuk menarik
  pembaca dan \emph{reviewer}: (1) Konteks/Motivasi (masalah penting di
  Lapisan Aplikasi Cx(A)), (2) Masalah/Kesenjangan (keterbatasan
  pendekatan yang ada C), (3) Intervensi/Kontribusi (solusi baru I), (4)
  Metodologi (bagaimana pendekatan dievaluasi menggunakan PICOC), (5)
  Hasil (temuan utama), dan (6) Kesimpulan/Dampak (implikasi lebih
  luas).
\item
  \textbf{Bab 1: Pendahuluan}

  \begin{itemize}
  \tightlist
  \item
    Mulai dengan ``kail'' yang kuat: jelaskan masalah dunia nyata yang
    penting dan mendesak (Konteks, Cx(A)).
  \item
    Deskripsikan siapa yang terpengaruh oleh masalah ini (Populasi,
    P(A)).
  \item
    Jelaskan solusi yang ada saat ini dan mengapa mereka tidak memadai
    (Kontrol, C(A) dan keterbatasan Hasilnya, O(C(A))).
  \item
    Nyatakan dengan jelas tujuan disertasi Anda: untuk mengusulkan dan
    memvalidasi sebuah solusi baru yang inovatif.
  \item
    Secara singkat, perkenalkan solusi Anda (Intervensi, I(A)) dan klaim
    utama tentang manfaatnya (Hasil yang diharapkan, O(A)).
  \item
    Sebutkan kontribusi penelitian Anda di setiap lapisan ASTF.
  \item
    Berikan peta jalan disertasi (outline bab-bab berikutnya).
  \end{itemize}
\item
  \textbf{Bab 2: Tinjauan Pustaka dan Landasan Teori}

  \begin{itemize}
  \tightlist
  \item
    Pendalaman Solusi Lama: Jelaskan bagaimana Solusi Lama (C(A))
    bekerja, termasuk sistem (C(S)), teknologi (C(T)), dan prinsip
    fundamental (C(F)) yang mendasarinya.
  \item
    Identifikasi Kesenjangan: Tinjau secara kritis literatur yang ada
    untuk menyoroti kesenjangan atau keterbatasan di setiap lapisan ASTF
    (A, S, T, F) yang relevan.
  \item
    Landasan Teori untuk Solusi Baru: Bangun fondasi teoretis untuk
    Intervensi (I) Anda. Jelaskan teori-teori (dari F), teknologi (dari
    T), dan pendekatan sistem (dari S) yang akan digunakan untuk
    membangun solusi Anda.
  \end{itemize}
\item
  \textbf{Bab 3: Metodologi Penelitian}

  \begin{itemize}
  \tightlist
  \item
    Ini adalah bab terpenting untuk menunjukkan ketegasan metodologis
    Anda.
  \item
    Jelaskan \textbf{W-Model TISE} sebagai kerangka kerja proses Anda
    secara keseluruhan, termasuk keempat kakinya dan interaksinya.
  \item
    Kemudian, dedikasikan sub-bab untuk setiap lapisan ASTF, menjelaskan
    secara rinci desain eksperimen menggunakan kerangka PICOC (P, I, C,
    O, Cx) untuk PICOC(A), PICOC(S), PICOC(T), dan PICOC(F). Pastikan
    detailnya cukup untuk reproduktifitas.
  \end{itemize}
\item
  \textbf{Bab 4: Hasil dan Pembahasan}

  \begin{itemize}
  \tightlist
  \item
    Strukturkan bab ini berdasarkan lapisan ASTF, dari bawah ke atas
    (F-\textgreater T-\textgreater S-\textgreater A) untuk membangun
    argumen Anda.
  \item
    Sajikan luaran (O) untuk setiap lapisan: O(F), O(T), O(S), O(A),
    menggunakan tabel, gambar, dan grafik secara efektif.
  \item
    Pembahasan harus menginterpretasikan hasil untuk setiap lapisan
    dalam konteksnya, menganalisis signifikansi peningkatan, dan secara
    eksplisit menghubungkan keberhasilan di lapisan bawah (misalnya, F
    atau T) yang memungkinkan atau berkontribusi pada luaran di lapisan
    atas (misalnya, S atau A). Ini menunjukkan koherensi penelitian
    Anda.
  \item
    Diskusikan implikasi (misalnya, O(A) bagi P(A)), keterbatasan studi,
    dan potensi ancaman terhadap validitas internal dan eksternal temuan
    Anda.
  \end{itemize}
\item
  \textbf{Bab 5: Kesimpulan dan Saran}

  \begin{itemize}
  \tightlist
  \item
    Ringkas kembali masalah, pendekatan, dan kontribusi utama Anda di
    setiap lapisan ASTF.
  \item
    Jawab kembali pertanyaan penelitian yang Anda ajukan di Bab 1.
  \item
    Diskusikan dampak yang lebih luas dari pekerjaan Anda.
  \item
    Berikan saran yang konkret dan dapat ditindaklanjuti untuk
    penelitian di masa depan, seperti mengatasi keterbatasan studi,
    memperluas solusi ke populasi atau konteks baru, atau menyelidiki
    perbaikan lebih lanjut di salah satu lapisan ASTF.
  \end{itemize}
\end{itemize}

\subsection{\texorpdfstring{\textbf{9.4 Menulis Publikasi Ilmiah (IEEE
Paper) Menggunakan Kerangka
TISE}}{9.4 Menulis Publikasi Ilmiah (IEEE Paper) Menggunakan Kerangka TISE}}\label{menulis-publikasi-ilmiah-ieee-paper-menggunakan-kerangka-tise}

Mempublikasikan hasil disertasi di jurnal atau konferensi bereputasi
seperti IEEE memerlukan strategi untuk memadatkan penelitian yang kaya
dan berlapis menjadi narasi yang ringkas, tajam, dan berdampak.

\begin{itemize}
\tightlist
\item
  \textbf{Strategi Penceritaan: Pilih Benang Merah Anda}

  \begin{itemize}
  \tightlist
  \item
    Sebuah paper tidak bisa menceritakan semua detail dari disertasi
    Anda; Anda harus memilih satu ``benang merah'' atau kontribusi utama
    sebagai fokus.
  \item
    \textbf{Contoh 1 (Fokus Teknologi)}: Jika kontribusi terkuat Anda
    adalah algoritma AI baru (Lapisan T), maka paper Anda akan berpusat
    pada itu. Pendahuluan akan memotivasi masalah dari Lapisan A. Bagian
    Metode akan merinci algoritma Anda (I(T)) dan bagaimana Anda
    membandingkannya dengan yang lain (C(T)). Bagian Hasil akan
    menunjukkan keunggulannya (O(T)). Kesimpulan akan membahas bagaimana
    teknologi superior ini dapat memungkinkan sistem yang lebih baik
    (implikasi untuk S) dan solusi yang lebih baik bagi pengguna
    (implikasi untuk A).
  \item
    \textbf{Contoh 2 (Fokus Sistem)}: Jika kontribusi Anda adalah
    arsitektur sistem baru (Lapisan S), paper Anda akan fokus pada
    desain dan evaluasi sistem tersebut.
  \end{itemize}
\item
  \textbf{Struktur Abstrak yang Efektif}

  \begin{itemize}
  \tightlist
  \item
    Abstrak adalah bagian terpenting untuk menarik pembaca dan
    \emph{reviewer}. Gunakan struktur 6 poin seperti yang dijelaskan
    sebelumnya.
  \end{itemize}
\item
  \textbf{Panduan Penulisan Makalah IEEE dengan PICOC Berlapis}

  \begin{itemize}
  \tightlist
  \item
    Struktur inti PICOC diterapkan pada setiap lapisan (P, I, C, O, Cx).
  \item
    Empat lapisan temuan (A, S, T, F) masing-masing dengan PICOC-nya
    sendiri harus dijelaskan. Misalnya, Lapisan Aplikasi (A) berfokus
    pada pemecahan masalah pemangku kepentingan, dengan P(A) sebagai
    pemangku kepentingan dan O(A) sebagai peningkatan kinerja. Lapisan
    Riset Fundamental (F) berfokus pada penambahan khazanah ilmu
    pengetahuan, dengan P(F) sebagai fenomena dan O(F) sebagai
    pengetahuan baru.
  \item
    Bagian pendahuluan harus memuat pernyataan masalah (Cx(A)), situasi
    saat ini (C(A)), pentingnya, solusi yang diusulkan (I(A)), tujuan
    dan kontribusi, serta garis besar makalah.
  \item
    Tinjauan pustaka harus mendalami solusi lama (C(A), C(S), C(T),
    C(F)), mengidentifikasi kesenjangan, dan membangun dasar untuk
    solusi baru (I(A), I(S), I(T), I(F)).
  \item
    Bagian metodologi harus merinci \emph{bagaimana} penelitian
    disiapkan dan dilakukan untuk mengevaluasi intervensi di setiap
    lapisan, memastikan detail yang cukup untuk reproduktifitas.
  \item
    Bagian hasil dan pembahasan harus menyajikan luaran (O) untuk setiap
    lapisan (A, S, T, F) secara jelas menggunakan gambar dan tabel,
    diikuti dengan pembahasan yang menginterpretasikan hasil,
    menganalisis signifikansi, implikasi, koneksi antar-lapisan,
    keterbatasan, dan ancaman terhadap validitas.
  \item
    Kesimpulan harus merangkum kontribusi, menjawab pertanyaan
    penelitian, membahas dampak lebih luas, dan menyarankan pengembangan
    selanjutnya.
  \end{itemize}
\end{itemize}

\subsection{\texorpdfstring{\textbf{9.5 Mengantisipasi Pertanyaan
\emph{Reviewer} dan Menghasilkan
Publikasi}}{9.5 Mengantisipasi Pertanyaan Reviewer dan Menghasilkan Publikasi}}\label{mengantisipasi-pertanyaan-reviewer-dan-menghasilkan-publikasi}

Dengan mengikuti kerangka TISE, Anda dapat secara efektif mengantisipasi
pertanyaan \emph{reviewer} yang umum dan menjelaskan kebaruan, validasi,
dan signifikansi praktis dari karya Anda. TISE membimbing Anda untuk
membangun argumen yang sangat kuat dan koheren, di mana hasil dari satu
lapisan menjadi bukti pendukung untuk lapisan di atasnya, menciptakan
``rantai bukti'' yang logis dari riset fundamental hingga dampak pada
pengguna.

Untuk membantu proses ini, buku ini juga menyediakan alat bantu praktis,
termasuk \textbf{Templat Kosong Kerangka PICOC Berlapis} dan
\textbf{Daftar Periksa Proposal Riset TISE}, serta \textbf{Daftar
Periksa Pra-pengiriman Paper IEEE} di Lampiran B. Daftar periksa ini
mencakup aspek struktur, konten, kualitas, dan koherensi, memastikan
bahwa paper Anda memenuhi standar publikasi ilmiah yang tinggi.

\begin{center}\rule{0.5\linewidth}{0.5pt}\end{center}

\bookmarksetup{startatroot}

\chapter{\texorpdfstring{\textbf{Bab 10: Peran Alat Bantu: Python,
Ontologi, Prolog, Quarto, dan Mermaid dalam Implementasi
TISE}}{Bab 10: Peran Alat Bantu: Python, Ontologi, Prolog, Quarto, dan Mermaid dalam Implementasi TISE}}\label{bab-10-peran-alat-bantu-python-ontologi-prolog-quarto-dan-mermaid-dalam-implementasi-tise}

Tentu, berikut adalah draf Bab 10 dari buku Anda, yang membahas peran
alat bantu seperti Python, Quarto, Mermaid, dan Triune Intelligence,
dengan mengacu pada semua sumber yang diberikan dan riwayat percakapan
kita:

\begin{center}\rule{0.5\linewidth}{0.5pt}\end{center}

Setelah memahami metodologi inti TISE, termasuk dekomposisi masalah
melalui ASTF (Bab 4), siklus kognitif Triune-PUDAL (Bab 5), penciptaan
nilai holistik melalui PSKVE (Bab 6), dan metodologi validasi PICOC
Sistematis (Bab 7), serta kerangka kerja W-Model (Bab 8), kini kita akan
fokus pada alat bantu praktis yang memungkinkan implementasi dan
dokumentasi paradigma TISE. Bab ini akan menguraikan bagaimana Python,
Ontologi, Prolog, Quarto, dan Mermaid berperan penting dalam mewujudkan
prinsip-prinsip Triune-Intelligence Smart-Engineering (TISE) dalam riset
dan pengembangan artefak cerdas.

\section{\texorpdfstring{\textbf{10.1 Python: Bahasa Pemrograman untuk
Implementasi dan Simulasi
AI}}{10.1 Python: Bahasa Pemrograman untuk Implementasi dan Simulasi AI}}\label{python-bahasa-pemrograman-untuk-implementasi-dan-simulasi-ai}

\textbf{Python} adalah bahasa pemrograman utama dan serbaguna yang
sangat diutamakan dalam kerangka TISE untuk \textbf{implementasi AI,
simulasi, dan integrasi}. Perannya sangat krusial di berbagai aspek
pengembangan artefak cerdas:

\begin{itemize}
\tightlist
\item
  \textbf{Implementasi AI dan Komputasi}: Python adalah pilihan ideal
  untuk mengembangkan komponen AI yang digerakkan oleh
  \textbf{Kecerdasan Buatan (AI)} dalam siklus PUDAL (Perceive,
  Understand, Decision-making \& Planning, Act-Response,
  Learning-evaluating). Ini mencakup implementasi algoritma untuk
  pengumpulan data sensorik (Perceive), analisis pola data (Understand),
  dan eksekusi tindakan (Act-Response). Kemampuan Python untuk komputasi
  dan pembelajaran AI mendukung fase-fase ini secara efisien.
\item
  \textbf{Optimasi Dimensi Nilai PSKVE}: Python digunakan untuk
  mengoptimalkan dimensi nilai dalam kerangka PSKVE (Product, Service,
  Knowledge, Value, Environmental). Ini dapat mencakup pengembangan
  model untuk menyeimbangkan \emph{trade-off} antar dimensi nilai atau
  untuk memprediksi dampak perubahan desain pada setiap dimensi.
\item
  \textbf{Dukungan Evaluasi PICOC}: Dalam metodologi validasi PICOC,
  Python sangat membantu dalam analisis statistik dan visualisasi data
  yang dihasilkan dari eksperimen dan simulasi. Ini memungkinkan
  peneliti untuk mengukur hasil (Outcome) secara objektif dan
  menginterpretasikan temuan secara efisien.
\item
  \textbf{Perekat Integrasi (\emph{Integration Glue})}: Python bertindak
  sebagai ``perekat integrasi'' yang menghubungkan berbagai komponen AI
  dan pengetahuan dalam sistem TISE yang kompleks.
\item
  \textbf{Tutorial Matematika dan Fisika}: Pustaka Python seperti
  \textbf{SymPy} dapat digunakan secara efektif untuk tutorial
  matematika dan fisika. SymPy menyediakan berbagai fungsi untuk
  kalkulus, aljabar, pemecahan persamaan, matriks, dan banyak lagi,
  memungkinkan pengguna untuk melakukan komputasi simbolik dan numerik.
  Ini mendukung pemahaman dan aplikasi prinsip-prinsip fundamental yang
  relevan di Lapisan Riset Fundamental (F) dari kerangka ASTF.
\end{itemize}

Sebagai contoh penerapan, dalam studi kasus simulasi alternatif
transportasi Jakarta-Bandung tahun 2030 atau dinamika P2P Lending,
Python digunakan untuk mengeksekusi logika simulasi, menghitung biaya,
emisi, waktu tempuh, atau mensimulasikan perilaku agen berdasarkan
\emph{multiplier} yang diinformasikan oleh ontologi di berbagai skenario
ekonomi.

\section{\texorpdfstring{\textbf{10.2 Ontologi dan Prolog:
Mengimplementasikan Triune
Intelligence}}{10.2 Ontologi dan Prolog: Mengimplementasikan Triune Intelligence}}\label{ontologi-dan-prolog-mengimplementasikan-triune-intelligence}

\textbf{Ontologi dan Prolog} memainkan peran sentral dalam mewujudkan
aspek \textbf{Triune Intelligence} dalam artefak cerdas TISE, terutama
dalam mengintegrasikan Kecerdasan Alami (NI) dan Kecerdasan Budaya (CI)
ke dalam sistem yang dapat diproses oleh AI.

\section{\texorpdfstring{\textbf{10.2.1 Ontologi: Konseptualisasi
Bersama dan Semantik
Umum}}{10.2.1 Ontologi: Konseptualisasi Bersama dan Semantik Umum}}\label{ontologi-konseptualisasi-bersama-dan-semantik-umum}

\textbf{Ontologi} adalah alat yang sangat penting untuk menciptakan
\textbf{konseptualisasi bersama} dan \textbf{semantik umum} antar
sistem, yang krusial untuk interoperabilitas dalam lingkungan rekayasa
yang kompleks. Perannya meliputi:

\begin{itemize}
\tightlist
\item
  \textbf{Menangkap Pengetahuan NI dan CI}: Ontologi memungkinkan
  penangkapan pengetahuan manusia (\textbf{Natural Intelligence - NI})
  dan konteks budaya (\textbf{Cultural Intelligence - CI}) secara
  eksplisit ke dalam bentuk yang dapat digunakan oleh algoritma AI. Ini
  membantu dalam memastikan \textbf{keselarasan semantik} antar disiplin
  ilmu dan mengurangi bias yang mungkin timbul dari interpretasi yang
  berbeda. Misalnya, ontologi dapat memformalkan definisi istilah,
  hubungan antar konsep, dan aturan domain yang mencerminkan pemahaman
  manusia dan nilai-nilai budaya.
\item
  \textbf{Formalisasi Konsep TISE}: Ontologi dapat digunakan untuk
  memformalkan konsep-konsep inti kerangka TISE itu sendiri, seperti
  struktur PUDAL, dimensi PSKVE, lapisan ASTF, dan elemen PICOC. Ini
  menciptakan bahasa umum yang konsisten untuk merancang dan
  menganalisis artefak cerdas.
\item
  \textbf{Model Metamodeling}: Metamodel OWL klasik, misalnya, mencakup
  tiga elemen inti: \emph{Individual, Property}, dan \emph{Class}.
  \emph{Property} dapat memiliki banyak tipe seperti
  \emph{OntologyProperty, DataProperty, AnnotationProperty,
  ObjectProperty}, dan \emph{DeprecatedProperty}.
  \emph{AnnotationProperty} dapat diketik oleh \emph{Label, versionInfo,
  comment, seeAlso}, dan \emph{isDefinedBy}. \emph{ObjectProperty}
  selanjutnya diwarisi oleh \emph{FunctionalProperty,
  InverseFunctionalProperty, TransitionProperty, SemetricProperty}, dan
  \emph{AsmmetricProperty}. Seseorang dapat mendefinisikan kelas khusus
  dengan menambahkan \emph{Restriction} ke \emph{ObjectProperty}, yang
  terdiri dari \emph{Cardinality Restriction, HasValue}, dan
  \emph{Quantifier}.
\item
  \textbf{Profil UML untuk Ontologi}: Profil UML dapat digunakan untuk
  mendefinisikan ontologi, seperti OWL Lite. Ada perangkat bantu seperti
  \emph{ontoUMLTool} yang dikembangkan berdasarkan profil UML untuk
  mendukung pembuatan ontologi dalam lingkungan seperti MagicDraw. Alat
  ini memungkinkan pengguna untuk membangun ontologi tingkat atas yang
  menggambarkan konsep, hubungan, dan tag khusus domain.
\end{itemize}

\section{\texorpdfstring{\textbf{10.2.2 Prolog: Representasi Pengetahuan
Deklaratif dan Penalaran
Logis}}{10.2.2 Prolog: Representasi Pengetahuan Deklaratif dan Penalaran Logis}}\label{prolog-representasi-pengetahuan-deklaratif-dan-penalaran-logis}

\textbf{Prolog} adalah bahasa pemrograman logika yang menyediakan
representasi pengetahuan deklaratif dan kemampuan \textbf{penalaran
logis} yang kuat.

\begin{itemize}
\tightlist
\item
  \textbf{Menggabungkan NI/CI ke dalam AI}: Prolog memungkinkan
  penggabungan Kecerdasan Alami (NI) dan Kecerdasan Budaya (CI) ke dalam
  sistem AI dengan mengkodekan aturan dan preseden logis. Ini penting
  untuk memodelkan aspek-aspek penalaran manusia dan aturan-aturan
  sosial yang kompleks yang tidak mudah diungkapkan melalui algoritma AI
  statistik murni.
\item
  \textbf{Simulasi Berbasis Pengetahuan}: Dalam simulasi, Prolog dapat
  digunakan untuk merepresentasikan pengetahuan domain dan aturan
  interaksi. Sebagai contoh, dalam simulasi transportasi, ontologi dan
  Prolog dapat mendefinisikan konsep kendaraan, rute, dan konsumsi
  energi, yang kemudian digunakan oleh Python untuk mengeksekusi logika
  simulasi. Demikian pula, untuk simulasi dinamika P2P Lending, ontologi
  komprehensif (pemangku kepentingan, siklus pinjaman, faktor
  makroekonomi) dapat dimodelkan dalam Prolog.
\end{itemize}

\section{\texorpdfstring{\textbf{10.2.3 Sinergi Ontologi-Prolog-Python
untuk Implementasi TI yang
Kuat}}{10.2.3 Sinergi Ontologi-Prolog-Python untuk Implementasi TI yang Kuat}}\label{sinergi-ontologi-prolog-python-untuk-implementasi-ti-yang-kuat}

Kombinasi \textbf{Ontologi, Prolog, dan Python} menciptakan sinergi yang
kuat untuk implementasi Triune Intelligence: * \textbf{Komputasi AI
(Python)}: Python menangani tugas-tugas komputasi berat dan implementasi
algoritma AI. * \textbf{Penalaran Logis (Prolog/NI)}: Prolog menyediakan
lapisan penalaran logis yang mengintegrasikan pengetahuan deklaratif
yang berasal dari NI. * \textbf{Nilai dan Konteks (Ontologi/CI)}:
Ontologi menyediakan representasi formal dari konsep, hubungan, dan
nilai-nilai yang membentuk Cultural Intelligence, memastikan sistem
beroperasi dalam konteks yang selaras secara budaya dan etis.

Sinergi ini memungkinkan artefak cerdas TISE untuk tidak hanya melakukan
tugas-tugas secara efisien (AI), tetapi juga untuk memahami konteks dan
tujuan yang lebih dalam (CI) dan membuat keputusan yang diinformasikan
oleh penalaran manusiawi (NI).

\section{\texorpdfstring{\textbf{10.3 Quarto dan Mermaid: Dokumentasi
dan Visualisasi
Komprehensif}}{10.3 Quarto dan Mermaid: Dokumentasi dan Visualisasi Komprehensif}}\label{quarto-dan-mermaid-dokumentasi-dan-visualisasi-komprehensif}

Dokumentasi yang jelas dan visualisasi yang efektif sangat penting dalam
rekayasa sistem yang kompleks dan penulisan disertasi. Dalam kerangka
TISE, \textbf{Quarto} dan \textbf{Mermaid} menawarkan solusi modern
untuk kebutuhan ini.

\section{\texorpdfstring{\textbf{10.3.1 Quarto: Sistem Publikasi Ilmiah
Berbasis
Markdown}}{10.3.1 Quarto: Sistem Publikasi Ilmiah Berbasis Markdown}}\label{quarto-sistem-publikasi-ilmiah-berbasis-markdown}

\textbf{Quarto} adalah sistem publikasi ilmiah yang memungkinkan
pembuatan buku, disertasi, artikel, presentasi, dan blog dengan sintaks
Markdown. Ini mendukung integrasi kode dan \emph{output} secara mulus,
memfasilitasi reproduktifitas dan kolaborasi. Fitur-fitur utama Quarto
meliputi: * \textbf{Penulisan Berbasis Markdown}: Memungkinkan penulisan
konten teknis dan ilmiah dengan sintaks yang sederhana namun kuat. *
\textbf{Integrasi Kode dan \emph{Output}}: Dapat mengeksekusi kode dari
berbagai bahasa (termasuk Python, R, Julia, Observable) dan menampilkan
\emph{output} (teks, tabel, grafik) langsung di dokumen. *
\textbf{Dukungan untuk Elemen Ilmiah}: Memfasilitasi pembuatan tabel,
gambar, kutipan, dan referensi silang, yang penting untuk publikasi
ilmiah. * \textbf{Manajemen Proyek dan Lingkungan Virtual}: Mendukung
pengaturan proyek, pengelolaan eksekusi, profil proyek, variabel
lingkungan, skrip proyek, dan lingkungan virtual, yang sangat berguna
untuk riset yang kompleks.

Dengan Quarto, seluruh proses penelitian, mulai dari analisis data
hingga penulisan disertasi, dapat didokumentasikan dalam satu alur kerja
yang kohesif, memastikan konsistensi dan kemudahan
\emph{reproducibility}.

\section{\texorpdfstring{\textbf{10.3.2 Mermaid: Diagram Berbasis Teks
untuk Visualisasi
Cepat}}{10.3.2 Mermaid: Diagram Berbasis Teks untuk Visualisasi Cepat}}\label{mermaid-diagram-berbasis-teks-untuk-visualisasi-cepat}

\textbf{Mer menggunakan sistem kontrol versi seperti Git, sama seperti
kode program atau dokumen teks lainnya. * }Peningkatan Kejelasan
Visual**: Mermaid dapat digunakan untuk memvisualisasikan arsitektur
TISE (misalnya, lapisan ASTF), siklus PUDAL, struktur sistem, alur
kerja, dan interaksi komponen. Ini sangat meningkatkan kejelasan visual
dari penjelasan konseptual dan teknis, yang merupakan kunci untuk
komunikasi yang efektif dalam disertasi dan publikasi ilmiah.

\emph{Perlu dicatat bahwa meskipun Quarto dan Mermaid adalah alat yang
sangat relevan dan sering digunakan dalam penulisan teknis modern,
sumber-sumber yang diberikan untuk buku ini secara langsung tidak
membahas Quarto atau Mermaid sebagai bagian dari kerangka TISE,
melainkan sebagai alat bantu umum. Oleh karena itu, penyebutan ini
didasarkan pada inferensi praktik terbaik dalam konteks penulisan buku
dan disertasi teknis.}

\section{\texorpdfstring{\textbf{10.4 Peran Alat Bantu dalam Mengelola
Kompleksitas SoAS dan
LoA}}{10.4 Peran Alat Bantu dalam Mengelola Kompleksitas SoAS dan LoA}}\label{peran-alat-bantu-dalam-mengelola-kompleksitas-soas-dan-loa}

Integrasi alat bantu ini sangat krusial dalam mengelola kompleksitas
\textbf{System of Autonomous Systems (SoAS)}, terutama saat
mempertimbangkan \textbf{Level of Autonomy (LoA)}. * \textbf{Taksonomi
Sistem Otonom}: Taksonomi berdasarkan LoA memberikan gambaran yang lebih
jelas tentang berbagai tingkat kemampuan otonom sistem, menghasilkan
bahasa umum untuk berbagai disiplin ilmu rekayasa. Ontologi dan bahasa
pemodelan formal dapat memfasilitasi pengembangan taksonomi ini. *
\textbf{Dampak pada Kinerja Manusia}: Integrasi otonomi yang lebih
tinggi ke dalam operasi SoS dapat memengaruhi kinerja manusia. Misalnya,
LoA yang lebih rendah dapat menghasilkan kinerja sistem yang lebih baik
tetapi merugikan kinerja pengguna. Alat seperti Python (untuk simulasi)
dan Ontologi/Prolog (untuk memodelkan perilaku dan batasan) dapat
digunakan untuk mengevaluasi \emph{trade-off} ini sebelum implementasi
dunia nyata. Dokumentasi yang jelas dari desain SoAS dan keputusan LoA,
difasilitasi oleh Quarto dan Mermaid, menjadi penting untuk
menginformasikan operator manusia dan memastikan pemahaman bersama.

Secara keseluruhan, alat-alat yang dibahas dalam bab ini secara kolektif
memberdayakan para insinyur dan peneliti TISE untuk merancang,
mengembangkan, memvalidasi, dan mendokumentasikan artefak cerdas dengan
cara yang sistematis, koheren, dan selaras dengan prinsip-prinsip Triune
Intelligence.

\begin{center}\rule{0.5\linewidth}{0.5pt}\end{center}

\bookmarksetup{startatroot}

\chapter{\texorpdfstring{\textbf{Bab 11: Implementasi Praktis dan Studi
Kasus
TISE}}{Bab 11: Implementasi Praktis dan Studi Kasus TISE}}\label{bab-11-implementasi-praktis-dan-studi-kasus-tise}

Tentu, berikut adalah draf Bab 11 dari buku Anda:

\begin{center}\rule{0.5\linewidth}{0.5pt}\end{center}

Setelah menguraikan fondasi filosofis TISE (Bagian 1) dan metodologi
rekayasanya yang komprehensif (Bagian 2), kini kita akan beralih ke
implementasi praktis. Bab ini akan menyajikan kerangka kerja sistematis
untuk menerapkan TISE, diikuti dengan dua studi kasus yang akan
mengilustrasikan bagaimana paradigma ini dapat digunakan untuk
merancang, mengembangkan, dan memvalidasi artefak cerdas. Studi kasus
ini juga akan menunjukkan pemanfaatan alat bantu modern yang dibahas di
Bab 10.

\section{\texorpdfstring{\textbf{11.1 Kerangka Implementasi TISE secara
Sistematis}}{11.1 Kerangka Implementasi TISE secara Sistematis}}\label{kerangka-implementasi-tise-secara-sistematis}

Menerapkan paradigma TISE membutuhkan pendekatan yang terstruktur. Tabel
berikut menguraikan kerangka kerja sistematis yang menghubungkan
lapisan-lapisan TISE dengan tujuan, prinsip/teori, aktivitas kunci,
artefak yang dihasilkan, dan metrik validasi. Ini berfungsi sebagai
panduan praktis untuk setiap proyek rekayasa cerdas.

\textbf{Tabel 11.1: Kerangka Implementasi TISE yang Sistematis}

\begin{longtable}[]{@{}
  >{\raggedright\arraybackslash}p{(\linewidth - 10\tabcolsep) * \real{0.0193}}
  >{\raggedright\arraybackslash}p{(\linewidth - 10\tabcolsep) * \real{0.1705}}
  >{\raggedright\arraybackslash}p{(\linewidth - 10\tabcolsep) * \real{0.3321}}
  >{\raggedright\arraybackslash}p{(\linewidth - 10\tabcolsep) * \real{0.1357}}
  >{\raggedright\arraybackslash}p{(\linewidth - 10\tabcolsep) * \real{0.1594}}
  >{\raggedright\arraybackslash}p{(\linewidth - 10\tabcolsep) * \real{0.1831}}@{}}
\toprule\noalign{}
\begin{minipage}[b]{\linewidth}\raggedright
Lapisan TISE
\end{minipage} & \begin{minipage}[b]{\linewidth}\raggedright
Tujuan Utama
\end{minipage} & \begin{minipage}[b]{\linewidth}\raggedright
Prinsip/Teori Pendukung
\end{minipage} & \begin{minipage}[b]{\linewidth}\raggedright
Aktivitas Kunci
\end{minipage} & \begin{minipage}[b]{\linewidth}\raggedright
Artefak yang Dihasilkan
\end{minipage} & \begin{minipage}[b]{\linewidth}\raggedright
Metrik Validasi (PICOC)
\end{minipage} \\
\midrule\noalign{}
\endhead
\bottomrule\noalign{}
\endlastfoot
\textbf{A (Aplikasi)} & Mengidentifikasi masalah nyata pemangku
kepentingan, mendefinisikan ``MENGAPA'' dari solusi, dan memahami dampak
di dunia nyata. & Desain Berpusat pada Manusia (HCD), Desain Peka Nilai
(VSD), Sistem Sosio-Teknikal (STS). Kecerdasan Kultural (CI) sebagai
pendorong utama. & Survei pengguna, FGD (Focus Group Discussion) dengan
pemangku kepentingan, analisis kebutuhan pasar. & Pernyataan masalah
terdefinisi jelas, persyaratan fungsional \& non-fungsional dari sudut
pandang pemangku kepentingan. & \textbf{P(A)}: Demografi pemangku
kepentingan. \textbf{I(A)}: Deskripsi solusi aplikasi. \textbf{C(A)}:
Kinerja solusi lama yang dirasakan. \textbf{O(A)}: Peningkatan kepuasan
pengguna, efisiensi yang dirasakan, pengurangan biaya. \textbf{Cx(A)}:
Lingkungan operasional aplikasi. \\
\textbf{S (Sistem)} & Merancang arsitektur sistem keseluruhan yang
mengintegrasikan berbagai komponen dan teknologi untuk memenuhi
kebutuhan aplikasi. & Triune-PUDAL Engine, Model Kolaborasi Manusia-AI
(HAIC), STS, W-Model. Integrasi NI-CI-AI. & Pemodelan arsitektur sistem
(misalnya, UML, SysML), simulasi sistem, desain antarmuka manusia-mesin.
& Diagram arsitektur sistem, spesifikasi antarmuka, \emph{testbed}
simulasi. & \textbf{P(S)}: Data set uji, \emph{testbed} simulasi,
lingkungan urban simulasi. \textbf{I(S)}: Arsitektur sistem yang
diusulkan. \textbf{C(S)}: Kinerja sistem \emph{baseline}. \textbf{O(S)}:
Akurasi, kecepatan, utilisasi sumber daya, keandalan sistem.
\textbf{Cx(S)}: Persyaratan sistem \emph{real-time}, kemampuan menangani
beragam \emph{input}. \\
\textbf{T (Teknologi)} & Mengembangkan atau memilih teknologi kunci
(mesin) yang efektif dalam melakukan tugas penting, berdasarkan
prinsip-prinsip fundamental. & Core Engine, Triune-PUDAL Engine. &
Pengembangan modul teknologi, pengujian \emph{benchmark}, implementasi
algoritma AI. & Algoritma AI, prototipe teknologi, modul perangkat
keras. & \textbf{P(T)}: Data sensor mentah, sumber energi.
\textbf{I(T)}: Algoritma AI baru, teknologi komunikasi \emph{low-power}.
\textbf{C(T)}: Algoritma/teknologi yang ada. \textbf{O(T)}: Akurasi
lebih tinggi, tingkat \emph{false positive} lebih rendah, masa pakai
baterai lebih lama. \textbf{Cx(T)}: Tantangan akurasi deteksi dari data
bising dengan daya minimal. \\
\textbf{F (Riset Fundamental)} & Menambah khazanah ilmu pengetahuan
tentang realitas, menemukan prinsip-prinsip baru, atau memvalidasi teori
yang menjadi dasar teknologi. & Teori Konversi PSKVE, Termodinamika,
Teori Informasi, Matematika. & Eksperimen laboratorium, penurunan
teoretis, pemodelan matematika, simulasi fundamental. & Pengetahuan
baru, teori yang divalidasi, model fundamental. & \textbf{P(F)}:
Fenomena propagasi sinyal, batas teoretis kompresi data. \textbf{I(F)}:
Model matematika baru, pendekatan \emph{information-theoretic} baru.
\textbf{C(F)}: Model/teori yang ada. \textbf{O(F)}: Model prediktif yang
lebih akurat, teorema baru. \textbf{Cx(F)}: Kesenjangan pengetahuan
dalam pemahaman faktor fisiologis yang memengaruhi kualitas sinyal. \\
\end{longtable}

\section{\texorpdfstring{\textbf{11.2 Studi Kasus 1: Sistem Komuter
Cerdas
Jakarta-Bandung}}{11.2 Studi Kasus 1: Sistem Komuter Cerdas Jakarta-Bandung}}\label{studi-kasus-1-sistem-komuter-cerdas-jakarta-bandung}

Studi kasus ini mendemonstrasikan bagaimana masalah rekayasa yang sangat
besar dan kompleks dapat dipecah menjadi bagian-bagian yang dapat
dikelola menggunakan kerangka ASTF, dan bagaimana Triune Intelligence
beroperasi dalam setiap komponen.

\begin{itemize}
\tightlist
\item
  \textbf{Lapisan Aplikasi (A): Solusi Berbagi Kamar-Makanan-Perjalanan}

  \begin{itemize}
  \tightlist
  \item
    \textbf{Masalah}: Pada tahun 2030, jutaan komuter akan melakukan
    perjalanan harian antara Jakarta dan Bandung yang melelahkan dan
    memakan waktu. Ini menciptakan kebutuhan mendesak akan solusi untuk
    istirahat, makan, dan perjalanan yang efisien.
  \item
    \textbf{Solusi yang Diusulkan}: Sebuah model bisnis
    \textbf{``berbagi kamar-makanan-perjalanan''}
    (\emph{room-food-travel sharing}) yang terintegrasi untuk
    menyediakan akomodasi sementara (kapsul tidur), makanan yang nyaman
    (siap saji), dan perjalanan yang efisien dalam satu platform.
  \item
    \textbf{Peran Triune Intelligence}:

    \begin{itemize}
    \tightlist
    \item
      \textbf{CI}: Mendefinisikan kebutuhan dan preferensi komuter
      terkait kenyamanan, privasi, dan kebersihan yang selaras dengan
      norma sosial dan budaya.
    \item
      \textbf{NI}: Terlibat dalam desain \emph{user experience} (UX)
      untuk memastikan antarmuka yang intuitif dan memenuhi harapan
      manusia. Manusia sebagai pengambil keputusan akhir dalam memilih
      opsi berbagi.
    \item
      \textbf{AI}: Mendukung personalisasi rekomendasi kamar, makanan,
      dan rute berdasarkan riwayat dan preferensi pengguna.
    \end{itemize}
  \end{itemize}
\item
  \textbf{Lapisan Sistem (S): Sistem Kapsul Tidur Komuter dan Makanan
  Siap Saji}

  \begin{itemize}
  \tightlist
  \item
    \textbf{Arsitektur Sistem}: Sebuah sistem terintegrasi yang terdiri
    dari: (1) Kapsul tidur (\emph{Sleep-in-Capsules}) yang kompak dan
    nyaman di \emph{hub} transportasi. (2) Layanan makanan siap saji
    (\emph{Food-to-Go}) yang dapat dipesan sebelumnya. (3) Integrasi
    yang mulus dengan jaringan transportasi (kereta cepat, bus cerdas).
  \item
    \textbf{Fungsi}: Mengubah \emph{input} (permintaan komuter, energi,
    bahan makanan) menjadi \emph{output} (istirahat, nutrisi, perjalanan
    efisien).
  \item
    \textbf{Peran Triune Intelligence}:

    \begin{itemize}
    \tightlist
    \item
      \textbf{CI}: Menentukan standar kebersihan, keamanan, dan
      \emph{branding} layanan, serta tata kelola data yang sensitif.
    \item
      \textbf{NI}: Mengawasi operasi sistem, menangani pengecualian atau
      situasi darurat yang tidak dapat ditangani AI. Pengemudi manusia
      dalam skenario otonomi bersama dengan kendaraan (\emph{shared
      autonomy}) yang optimal bisa jadi 40\% manusia dan 60\% kendaraan.
    \item
      \textbf{AI}: Mengelola alokasi kapsul secara dinamis,
      mengoptimalkan rute transportasi, dan memproses pesanan makanan
      \emph{real-time}.
    \end{itemize}
  \end{itemize}
\item
  \textbf{Lapisan Teknologi (T): Mesin Listrik, Dapur Otomatis, Uang
  Digital, Platform Pembiayaan Digital}

  \begin{itemize}
  \tightlist
  \item
    \textbf{Teknologi Kunci}:

    \begin{itemize}
    \tightlist
    \item
      \textbf{Mesin Listrik} untuk menggerakkan kendaraan transportasi
      secara efisien dan berkelanjutan.
    \item
      \textbf{Dapur Otomatis} untuk produksi makanan massal yang cepat,
      konsisten, dan higienis.
    \item
      \textbf{Uang Digital} untuk transaksi yang lancar dan tanpa
      gesekan.
    \item
      \textbf{Platform Pembiayaan Digital} untuk mengelola arus kas,
      investasi, dan keberlanjutan ekonomi jangka panjang dari seluruh
      sistem.
    \end{itemize}
  \item
    \textbf{Peran Triune Intelligence}:

    \begin{itemize}
    \tightlist
    \item
      \textbf{CI}: Menentukan persyaratan keamanan siber untuk uang
      digital, standar kebersihan dapur, dan keberlanjutan energi yang
      selaras dengan kebijakan publik.
    \item
      \textbf{NI}: Terlibat dalam desain dan \emph{tuning} mesin
      listrik, kalibrasi dapur otomatis, serta pengembangan algoritma
      keamanan untuk transaksi digital.
    \item
      \textbf{AI}: Mengoptimalkan kinerja mesin listrik (misalnya,
      efisiensi energi), mengontrol proses dapur otomatis, dan memproses
      transaksi uang digital secara aman dan efisien.
    \end{itemize}
  \end{itemize}
\item
  \textbf{Lapisan Riset Fundamental (F): Prinsip Optimasi dan Konversi
  Nilai}

  \begin{itemize}
  \tightlist
  \item
    \textbf{Prinsip yang Diteliti}:

    \begin{itemize}
    \tightlist
    \item
      \textbf{Tingkat Jam Kerja Manusia Minimum}: Prinsip ekonomi
      industri untuk meminimalkan \emph{input} tenaga kerja manusia
      tanpa mengorbankan kualitas.
    \item
      \textbf{Optimasi Jadwal Waktu}: Prinsip dari riset operasi untuk
      memaksimalkan \emph{throughput} dan meminimalkan waktu tunggu.
    \item
      \textbf{Alokasi Sumber Daya}: Prinsip dari teori sistem untuk
      mendistribusikan sumber daya terbatas (kapsul, bahan makanan,
      energi) secara optimal.
    \item
      \textbf{Teori Konversi Nilai PSKVE}: Teori fundamental dari TISE
      tentang bagaimana berbagai bentuk ``energi'' (Produk, Layanan,
      Pengetahuan, Nilai, Lingkungan) dapat ditransaksikan untuk
      menciptakan nilai holistik.
    \end{itemize}
  \item
    \textbf{Peran Triune Intelligence}:

    \begin{itemize}
    \tightlist
    \item
      \textbf{CI}: Memandu arah penelitian untuk menemukan prinsip yang
      mendukung keberlanjutan dan keadilan sosial dalam alokasi sumber
      daya.
    \item
      \textbf{NI}: Memformulasikan hipotesis, merancang eksperimen
      teoretis, dan menginterpretasikan hasil untuk mengembangkan
      pengetahuan baru.
    \item
      \textbf{AI}: Mendukung simulasi kompleks untuk menguji model
      optimasi, menganalisis data besar untuk menemukan pola baru dalam
      konversi nilai, atau memprediksi perilaku sistem yang kompleks
      (misalnya, lalu lintas).
    \end{itemize}
  \end{itemize}
\end{itemize}

\section{\texorpdfstring{\textbf{11.3 Studi Kasus 2: Analisis Sistem
Rekomendasi Makanan Sehat
(MSRS)}}{11.3 Studi Kasus 2: Analisis Sistem Rekomendasi Makanan Sehat (MSRS)}}\label{studi-kasus-2-analisis-sistem-rekomendasi-makanan-sehat-msrs}

Studi kasus ini adalah contoh sempurna penerapan TISE pada masalah
sosio-teknis yang sangat kompleks, menunjukkan bagaimana Triune
Intelligence, PUDAL, dan PSKVE bekerja sama untuk menciptakan solusi
yang berpusat pada manusia.

\begin{itemize}
\tightlist
\item
  \textbf{Masalah (Konteks PSKVE)}: Memenuhi kebutuhan pangan lokal yang
  sehat menghadapi tantangan multi-dimensi.

  \begin{itemize}
  \tightlist
  \item
    \textbf{Produk \& Layanan}: Sulit menghasilkan makanan yang sesuai
    dengan profil kesehatan spesifik (misalnya, diabetes, hipertensi)
    dan selera lokal.
  \item
    \textbf{Nilai}: Makanan sehat seringkali tidak terjangkau karena
    rantai pasok yang tidak efisien.
  \item
    \textbf{Lingkungan}: Praktik pertanian dan peternakan seringkali
    tidak berkelanjutan.
  \item
    \textbf{Pengetahuan}: Kurangnya data tentang kebutuhan konsumen dan
    inovasi produk lokal yang sehat.
  \end{itemize}
\item
  \textbf{Solusi (Berbasis Triune Intelligence)}: Sebuah
  \emph{marketplace} makanan sehat yang berfungsi sebagai ekosistem yang
  menghubungkan berbagai pemangku kepentingan, dengan \textbf{MSRS
  (Multi-Stakeholder Recommendation System)} sebagai komponen AI.

  \begin{itemize}
  \tightlist
  \item
    \textbf{Triune Intelligence diwujudkan melalui kolaborasi antara}:

    \begin{itemize}
    \tightlist
    \item
      \textbf{Kecerdasan Manusia (NI)}: Konsumen (dengan kebutuhan dan
      preferensinya), Chef/Ahli Kuliner (dengan kreativitas dan kearifan
      lokalnya), UMKM Pangan (dengan kapasitas produksinya).
    \item
      \textbf{Kecerdasan Budaya (CI)}: Otoritas Gizi \& Agama (dengan
      standar dan pedomannya), budaya makanan lokal, nilai-nilai
      keberlanjutan.
    \item
      \textbf{Kecerdasan Buatan (AI)}: Algoritma rekomendasi MSRS,
      sistem manajemen rantai pasok.
    \end{itemize}
  \end{itemize}
\item
  \textbf{PSKVE Conversion}: Sistem ini secara aktif mengelola konversi
  PSKVE:

  \begin{itemize}
  \tightlist
  \item
    \textbf{Inovasi (Knowledge)} dari \emph{chef} diubah menjadi
    \textbf{Product} baru (resep makanan sehat).
  \item
    \textbf{Product} ini kemudian memberikan \textbf{Service} kesehatan
    (makanan siap saji sesuai diet) kepada konsumen.
  \item
    Ini menghasilkan \textbf{Value} ekonomi bagi UMKM dan operator
    \emph{marketplace}.
  \item
    Semua ini dilakukan dengan tujuan menggunakan bahan baku yang
    berkelanjutan (\textbf{Environmental}).
  \end{itemize}
\item
  \textbf{Triune-PUDAL dalam MSRS}:

  \begin{itemize}
  \tightlist
  \item
    \textbf{Perceive}: AI mengumpulkan data tentang preferensi konsumen,
    ketersediaan bahan baku, resep, dan standar gizi. NI/CI membantu
    menentukan data apa yang relevan.
  \item
    \textbf{Understand}: AI memproses data untuk mengidentifikasi pola,
    sementara CI (melalui standar gizi dan budaya makanan) memberikan
    konteks untuk memahami kebutuhan diet dan preferensi rasa. NI
    menginterpretasikan tren kesehatan.
  \item
    \textbf{Decision-making \& Planning}: NI (chef, ahli gizi) membuat
    keputusan tentang resep baru atau rekomendasi yang dipersonalisasi.
    AI mengoptimalkan rekomendasi untuk kesesuaian diet dan
    ketersediaan, dengan CI memastikan kepatuhan etis dan budaya.
  \item
    \textbf{Act-Response}: AI memfasilitasi pemesanan, produksi, dan
    pengiriman makanan yang direkomendasikan.
  \item
    \textbf{Learning-evaluating}: Kinerja penjualan dan ulasan dari item
    baru ini menjadi masukan baru bagi MSRS, menciptakan siklus
    perbaikan berkelanjutan. AI mengukur metrik teknis, CI mengevaluasi
    terhadap tujuan nilai, dan NI merefleksikan keputusannya.
  \end{itemize}
\end{itemize}

\section{\texorpdfstring{\textbf{11.4 Pemanfaatan Alat Bantu dalam Studi
Kasus}}{11.4 Pemanfaatan Alat Bantu dalam Studi Kasus}}\label{pemanfaatan-alat-bantu-dalam-studi-kasus}

Alat bantu yang dibahas di Bab 10 berperan krusial dalam implementasi
praktis TISE, terutama dalam pemodelan, simulasi, penalaran logis, dan
dokumentasi sistem yang kompleks.

\begin{itemize}
\tightlist
\item
  \textbf{Python (Implementasi AI \& Simulasi)}:

  \begin{itemize}
  \tightlist
  \item
    \textbf{Simulasi Alternatif Transportasi (Bandung-Jakarta 2030)}:
    Python mengeksekusi logika simulasi, menghitung biaya, emisi, dan
    waktu tempuh untuk berbagai skenario transportasi. Ini memungkinkan
    pengujian hipotesis di Lapisan Sistem (S) dan Lapisan Fundamental
    (F). Pustaka seperti \textbf{SymPy} dapat digunakan untuk memodelkan
    dinamika fisika atau optimasi matematis yang mendasari sistem
    transportasi.
  \item
    \textbf{Simulasi Dinamika P2P Lending}: Python digunakan untuk
    mensimulasikan perilaku agen (\emph{agent-based modeling})
    berdasarkan \emph{multiplier} yang diinformasikan oleh ontologi di
    berbagai skenario ekonomi. Ini membantu memahami bagaimana keputusan
    ekonomi memengaruhi kesehatan finansial rumah tangga (Studi Kasus
    Rumah Cerdas).
  \end{itemize}
\item
  \textbf{Ontologi dan Prolog (Representasi Pengetahuan \& Penalaran
  Logis)}:

  \begin{itemize}
  \tightlist
  \item
    \textbf{Ontologi}: Untuk kedua studi kasus, ontologi dapat digunakan
    untuk menciptakan \textbf{konseptualisasi bersama} dari domain
    masalah.

    \begin{itemize}
    \tightlist
    \item
      Pada transportasi, ontologi dapat mendefinisikan konsep kendaraan,
      rute, jenis energi, dan aturan lalu lintas yang berasal dari
      \textbf{CI}.
    \item
      Pada MSRS, ontologi dapat memformalkan definisi nutrisi, alergen,
      preferensi diet, dan standar makanan halal/kosher yang berasal
      dari \textbf{CI}, serta sifat-sifat bahan baku yang berasal dari
      \textbf{NI}. Ini memastikan keselarasan semantik dan mengurangi
      bias dalam sistem AI.
    \end{itemize}
  \item
    \textbf{Prolog}: Menyediakan kemampuan \textbf{penalaran logis}
    berdasarkan ontologi.

    \begin{itemize}
    \tightlist
    \item
      Dalam simulasi transportasi, Prolog dapat digunakan untuk
      memodelkan aturan keputusan untuk kendaraan otonom atau sistem
      manajemen lalu lintas berdasarkan kondisi yang didefinisikan dalam
      ontologi.
    \item
      Dalam MSRS, Prolog dapat digunakan untuk menyimpulkan rekomendasi
      makanan yang memenuhi batasan diet kompleks dan preferensi
      individu.
    \end{itemize}
  \end{itemize}
\item
  \textbf{Quarto (Dokumentasi Komprehensif)}:

  \begin{itemize}
  \tightlist
  \item
    Meskipun tidak secara eksplisit disebut dalam sumber sebagai alat
    TISE, Quarto sangat relevan untuk dokumentasi seluruh proyek TISE.
    Semua fase proyek---dari definisi masalah di Lapisan Aplikasi hingga
    temuan riset fundamental---dapat didokumentasikan dalam satu format
    yang kohesif dan reproduktif. Ini mencakup narasi, kode Python untuk
    simulasi, dan hasil analisis data.
  \end{itemize}
\item
  \textbf{Mermaid (Visualisasi Arsitektur \& Alur Kerja)}:

  \begin{itemize}
  \tightlist
  \item
    Mermaid memungkinkan pembuatan diagram berbasis teks yang mudah
    diintegrasikan ke dalam dokumen Quarto. Ini membantu dalam
    memvisualisasikan arsitektur sistem yang kompleks dan alur kerja
    dalam kedua studi kasus:

    \begin{itemize}
    \tightlist
    \item
      Untuk transportasi, diagram alur perjalanan komuter, arsitektur
      sistem kapsul tidur dan dapur otomatis, atau diagram interaksi
      antara komponen TI.
    \item
      Untuk MSRS, diagram arsitektur sistem rekomendasi, alur proses
      PUDAL, atau diagram interaksi antar pemangku kepentingan di
      \emph{marketplace}.
    \end{itemize}
  \end{itemize}
\end{itemize}

\section{\texorpdfstring{\textbf{11.5 Tantangan Otonomi dan Kinerja
Manusia dalam Sistem
Cerdas}}{11.5 Tantangan Otonomi dan Kinerja Manusia dalam Sistem Cerdas}}\label{tantangan-otonomi-dan-kinerja-manusia-dalam-sistem-cerdas}

Implementasi praktis sistem cerdas, terutama \emph{System of Autonomous
Systems} (SoAS), menghadirkan tantangan signifikan terkait otonomi dan
kinerja manusia. TISE secara proaktif mengatasi hal ini:

\begin{itemize}
\tightlist
\item
  \textbf{Kinerja Operator}: Penelitian menunjukkan bahwa \emph{Level of
  Autonomy} (LoA) yang lebih rendah, di mana manusia terlibat lebih
  banyak, kadang-kadang dapat menghasilkan kinerja sistem yang lebih
  baik tetapi merugikan kinerja operator manusia. Ini berarti bahwa
  tingkat otomatisasi yang tinggi, meskipun meningkatkan kapabilitas
  otonom, dapat meningkatkan beban kognitif dan kelelahan mental
  manusia.
\item
  \textbf{Keamanan Sistem Otomotif}: Dalam transportasi darat, integrasi
  kendaraan otonom menimbulkan masalah keamanan dan keamanan data.
  Misalnya, waktu reaksi yang lebih rendah dari kendaraan konvensional
  dapat menjadi faktor utama tabrakan dari belakang. Delaurentis
  menemukan bahwa titik operasi optimal dalam \emph{automotive swarm}
  adalah ketika kontrol terbagi 40\% manusia dan 60\% kendaraan.
\item
  \textbf{Pendekatan TISE}: Filosofi TISE, ``Engineers empower humans'',
  secara langsung mengatasi masalah ini. TISE tidak berupaya
  menghilangkan manusia, melainkan mengoptimalkan kolaborasi antara NI,
  CI, dan AI.

  \begin{itemize}
  \tightlist
  \item
    Dalam konteks \emph{shared autonomy} (seperti dalam transportasi
    darat), TISE dapat merancang sistem yang mencapai titik optimal
    seperti 40\% manusia dan 60\% kendaraan dengan mengintegrasikan
    keputusan AI untuk efisiensi dan kecepatan, dengan pengawasan dan
    intervensi NI untuk keamanan dan pertimbangan etis yang lebih
    tinggi.
  \item
    Simulasi (\emph{multi-UAV planning}, \emph{agent-based design},
    \emph{swarm planning optimization}) dapat digunakan untuk
    menganalisis perilaku SoAS dan dampak LoA. Alat bantu seperti
    Python, Ontologi, dan Prolog menjadi sangat penting dalam memodelkan
    dan mensimulasikan skenario ini untuk mengidentifikasi LoA optimal
    dan meminimalkan dampak negatif pada kinerja manusia.
  \end{itemize}
\end{itemize}

Melalui penerapan kerangka sistematis, studi kasus nyata, dan
pemanfaatan alat bantu canggih, TISE menyediakan jalur yang jelas bagi
para insinyur dan peneliti untuk membangun artefak cerdas yang tidak
hanya canggih secara teknologi tetapi juga bertanggung jawab secara
sosial, berkelanjutan secara ekologis, dan berpusat pada pemberdayaan
manusia.

\begin{center}\rule{0.5\linewidth}{0.5pt}\end{center}

\bookmarksetup{startatroot}

\chapter{\texorpdfstring{\textbf{Bab 12: Tantangan dan Arah Penelitian
Masa Depan dalam
TISE}}{Bab 12: Tantangan dan Arah Penelitian Masa Depan dalam TISE}}\label{bab-12-tantangan-dan-arah-penelitian-masa-depan-dalam-tise}

Tentu, berikut adalah draf Bab 12 dari buku Anda, yang membahas
tantangan dan arah penelitian masa depan dalam paradigma
Triune-Intelligence Smart-Engineering (TISE), dengan mengacu pada semua
sumber yang diberikan dan riwayat percakapan kita:

\begin{center}\rule{0.5\linewidth}{0.5pt}\end{center}

Paradigma Triune-Intelligence Smart-Engineering (TISE) telah menyajikan
kerangka kerja yang komprehensif untuk merancang artefak cerdas yang
berpusat pada manusia di era Kecerdasan Buatan (AI). Namun, seperti
halnya setiap pendekatan inovatif, TISE juga menghadapi tantangan yang
signifikan dan membuka berbagai jalan untuk penelitian di masa depan.
Bab ini akan membahas tantangan-tantangan ini dan menguraikan arah-arah
penelitian yang prospektif yang akan memperkaya dan mengembangkan
kerangka TISE lebih lanjut.

\section{\texorpdfstring{\textbf{12.1 Tantangan dalam Mengukur dan
Menyeimbangkan Kecerdasan Triune (NI, CI,
AI)}}{12.1 Tantangan dalam Mengukur dan Menyeimbangkan Kecerdasan Triune (NI, CI, AI)}}\label{tantangan-dalam-mengukur-dan-menyeimbangkan-kecerdasan-triune-ni-ci-ai}

Salah satu tantangan inheren dalam TISE adalah sifat kompleks dan
seringkali \emph{lunak} (soft) dari metrik yang terkait dengan
\textbf{Kecerdasan Kultural (CI)} dan \textbf{Kecerdasan Alami (NI)}.
Meskipun Kecerdasan Buatan (AI) dapat diukur dengan metrik kinerja
teknis yang jelas (misalnya, akurasi, kecepatan), pengukuran dimensi
budaya, etika, dan pengalaman manusia jauh lebih sulit.

\begin{itemize}
\tightlist
\item
  \textbf{Pengukuran Metrik CI yang ``Lembut''}: Mengukur dampak dan
  kontribusi CI terhadap sebuah artefak cerdas TISE merupakan tantangan
  tersendiri. Bagaimana kita mengukur ``nilai kolektif,'' ``kearifan,''
  atau ``keselarasan etis'' secara kuantitatif? Metrik seperti
  \textbf{Pertumbuhan \emph{Knowledge-Graph}} (jumlah triple yang
  divalidasi per bulan dari CI) dan \textbf{Indeks Penyelarasan
  Keputusan} (persentase keputusan yang mengutip masukan dari NI, CI,
  dan AI) telah diusulkan. Namun, pengembangan metodologi yang lebih
  kuat dan standar untuk mengkuantifikasi aspek-aspek ini tetap menjadi
  area penelitian utama.
\item
  \textbf{Pembobotan yang Adil Antar-Kutub Kecerdasan}: Memastikan
  pembobotan yang adil dan dinamis antar-kutub kecerdasan (NI, CI, AI)
  dalam proses pengambilan keputusan dan penciptaan nilai adalah
  krusial. Bagaimana kita menentukan kontribusi relatif dari NI, CI, dan
  AI dalam sebuah keputusan sistem? Penelitian perlu mengeksplorasi
  model alokasi bobot adaptif yang dapat berubah berdasarkan konteks,
  Level of Autonomy (LoA) sistem, dan sensitivitas terhadap nilai-nilai
  manusia.
\end{itemize}

\section{\texorpdfstring{\textbf{12.2 Pengembangan Taksonomi
Komprehensif untuk \emph{System of Autonomous Systems}
(SoAS)}}{12.2 Pengembangan Taksonomi Komprehensif untuk System of Autonomous Systems (SoAS)}}\label{pengembangan-taksonomi-komprehensif-untuk-system-of-autonomous-systems-soas}

Integrasi otonomi ke dalam sistem yang lebih besar, khususnya
\emph{System of Autonomous Systems} (SoAS), telah meningkatkan
kapabilitas otonom, namun juga menciptakan kompleksitas baru.

\begin{itemize}
\tightlist
\item
  \textbf{Kebutuhan Taksonomi LoA}: Diperlukan pengembangan taksonomi
  yang komprehensif untuk SoAS yang mempertimbangkan otonomi manajerial
  dan operasional. Taksonomi berdasarkan \emph{Level of Autonomy} (LoA)
  memberikan gambaran yang lebih jelas tentang berbagai tingkat
  kemampuan otonom sistem, menghasilkan bahasa umum untuk berbagai
  disiplin ilmu rekayasa. Sebuah SoAS dapat diberi peringkat lebih
  tinggi dalam LoA jika sistemnya mampu melakukan lebih banyak tugas
  misi secara mandiri.
\item
  \textbf{Peran AI dalam Kooperasi SoAS}: SoAS menggunakan algoritma
  kerja sama AI, membuat sistem berkolaborasi dalam melakukan tugas
  misi. Penelitian di masa depan harus fokus pada pengembangan taksonomi
  yang dapat mengklasifikasikan dan mengevaluasi algoritma kerja sama AI
  ini, serta mengukur bagaimana tingkat otonomi dalam SoAS memengaruhi
  kinerja sistem secara keseluruhan. Sumber juga mengutip penelitian
  tentang taksonomi untuk sistem otonom.
\end{itemize}

\section{\texorpdfstring{\textbf{12.3 Formalisasi Kerangka Kerja untuk
Penutupan
Sistem}}{12.3 Formalisasi Kerangka Kerja untuk Penutupan Sistem}}\label{formalisasi-kerangka-kerja-untuk-penutupan-sistem}

Sistem dalam konteks rekayasa TISE seringkali berinteraksi dengan
lingkungannya, dan pemahaman yang jelas tentang ``penutupan'' sistem ini
sangat penting untuk stabilitas dan prediktabilitas.

\begin{itemize}
\tightlist
\item
  \textbf{Kerangka Kerja Formal}: Diperlukan kerangka kerja formal untuk
  hubungan antara berbagai jenis penutupan sistem (fungsional,
  informasional). Misalnya, penelitian telah dilakukan untuk formalisasi
  teoretis sistem tertutup, khususnya dalam hal penutupan informasional.
  Sebuah teorema menunjukkan hubungan untuk tingkat informasi bersama
  (\emph{mutual information}) yang disajikan di batas sistem tertutup
  secara informasional. Untuk menjaga penutupan pada keadaan \emph{n} +
  1, keluaran sistem tertutup ke lingkungan pada keadaan \emph{n}
  (jumlah informasi yang ditransmisikan dari sistem ke lingkungannya)
  harus mengikuti teorema ini.
\item
  \textbf{Jenis Penutupan}: Selain penutupan informasional, penelitian
  di masa depan perlu mendefinisikan dan memformalkan jenis penutupan
  lainnya, seperti penutupan fungsional atau struktural, dan bagaimana
  mereka berinteraksi dalam sistem sosio-teknis TISE. Ini akan membantu
  dalam merancang sistem yang lebih mandiri dan kuat terhadap gangguan
  eksternal.
\end{itemize}

\section{\texorpdfstring{\textbf{12.4 Peningkatan Interoperabilitas Alat
dan Standardisasi Bahasa Pemodelan untuk Penggunaan Ulang
Pola}}{12.4 Peningkatan Interoperabilitas Alat dan Standardisasi Bahasa Pemodelan untuk Penggunaan Ulang Pola}}\label{peningkatan-interoperabilitas-alat-dan-standardisasi-bahasa-pemodelan-untuk-penggunaan-ulang-pola}

Meskipun Bab 10 membahas alat bantu yang efektif, peningkatan
berkelanjutan dalam interoperabilitas dan standardisasi sangat penting
untuk efisiensi rekayasa TISE.

\begin{itemize}
\tightlist
\item
  \textbf{Interoperabilitas Alat}: Peningkatan interoperabilitas antar
  alat rekayasa dan simulasi akan mempercepat proses pengembangan.
  Tantangan dalam Model-Based Systems Engineering (MBSE) mencakup
  kurangnya dukungan untuk implementasi MBSE yang efisien. Integrasi
  yang mulus antara Python untuk simulasi, Ontologi dan Prolog untuk
  penalaran berbasis pengetahuan, dan alat dokumentasi seperti Quarto
  dan Mermaid, perlu terus dikembangkan.
\item
  \textbf{Standardisasi Bahasa Pemodelan}: Standardisasi bahasa
  pemodelan untuk penggunaan ulang pola (\emph{reuse patterns}) adalah
  area penelitian yang menjanjikan. Penggunaan pola rekayasa dalam
  kerangka MBSE dapat membantu dalam kapitalisasi dan penggunaan ulang
  pengetahuan. Ontologi, dengan kemampuannya menciptakan konseptualisasi
  bersama dan semantik umum {[}Bab 10 dalam riwayat percakapan{]}, dapat
  menjadi dasar untuk standardisasi ini, memungkinkan definisi pola
  rekayasa TISE (misalnya, pola PUDAL, ASTF) yang dapat digunakan
  kembali di berbagai proyek.
\end{itemize}

\section{\texorpdfstring{\textbf{12.5 Implikasi LoA terhadap Kinerja
Operator dan
Sistem}}{12.5 Implikasi LoA terhadap Kinerja Operator dan Sistem}}\label{implikasi-loa-terhadap-kinerja-operator-dan-sistem}

Integrasi otonomi ke dalam operasi \emph{System of Systems} (SoS) telah
meningkatkan kapabilitas otonom, namun dapat memengaruhi kinerja
manusia.

\begin{itemize}
\tightlist
\item
  \textbf{Dampak pada Kinerja Manusia}: Penelitian menunjukkan bahwa
  \emph{Level of Autonomy} (LoA) yang lebih rendah dapat menghasilkan
  kinerja sistem yang lebih baik tetapi merugikan kinerja pengguna. LoA
  sebagai faktor kompleksitas memiliki dampak signifikan pada kinerja
  operator dan sistem, dan karenanya memengaruhi kinerja SoAS.
\item
  \textbf{Optimasi Kolaborasi Manusia-AI}: Arah penelitian masa depan
  harus berfokus pada menemukan titik \emph{sweet spot} (keseimbangan
  optimal) antara otomatisasi dan intervensi manusia, terutama dalam
  konteks Triune Intelligence. TISE, dengan filosofi ``Engineers empower
  humans,'' bertujuan untuk mengoptimalkan kolaborasi NI-CI-AI, bukan
  menggantikan manusia. Ini akan melibatkan penelitian tentang bagaimana
  LoA yang berbeda memengaruhi beban kognitif manusia, kepercayaan, dan
  efektivitas secara keseluruhan dalam sistem berbasis TISE. Simulasi,
  seperti yang digunakan dalam perencanaan multi-UAV (\emph{multi-UAV
  planning}) dan analisis berbasis agen (\emph{agent-based analysis}),
  akan menjadi alat penting untuk menguji berbagai tingkat LoA dan
  dampaknya pada kinerja manusia-sistem.
\end{itemize}

\section{\texorpdfstring{\textbf{12.6 Arah Penelitian Lanjutan di
Berbagai Lapisan
ASTF}}{12.6 Arah Penelitian Lanjutan di Berbagai Lapisan ASTF}}\label{arah-penelitian-lanjutan-di-berbagai-lapisan-astf}

Kerangka ASTF TISE menyediakan peta jalan yang jelas untuk penelitian di
masa depan di setiap lapisannya.

\begin{itemize}
\tightlist
\item
  \textbf{Lapisan Riset Fundamental (F)}: Penelitian dapat terus
  menggali prinsip-prinsip dasar yang mendukung setiap kecerdasan. Ini
  termasuk Fraktal Processing, Multifraktal Processing, dan teori-teori
  baru yang mendukung Rekayasa Sistem Pemrosesan Sinyal (SPS) dan
  Rekayasa PSKV.
\item
  \textbf{Lapisan Teknologi (T)}: Pengembangan teknologi baru untuk
  mewujudkan artefak TISE akan terus menjadi fokus. Ini bisa meliputi
  robotika, teknologi untuk ruang hidup cerdas (\emph{smart living
  engineering}), dan teknologi untuk ruang spiritual cerdas (\emph{smart
  spiritual engineering}).
\item
  \textbf{Lapisan Sistem (S)}: Desain arsitektur sistem yang lebih
  adaptif, tangguh, dan terintegrasi, yang mampu mengelola interaksi
  kompleks antara NI, CI, dan AI dalam berbagai domain aplikasi
  (misalnya, sistem kesehatan, transportasi, energi).
\item
  \textbf{Lapisan Aplikasi (A)}: Aplikasi TISE yang inovatif dalam
  domain seperti masyarakat sadar Covid, pasar \emph{home care}, rumah
  sehat cerdas, pasar produktivitas \emph{home learning}, ekonomi rumah
  tangga berbasis virtual, dan pasar kreatif spiritual.
\end{itemize}

\section{\texorpdfstring{\textbf{12.7 Kesimpulan: Menuju Masa Depan
Rekayasa yang Lebih Cerdas dan Bertanggung
Jawab}}{12.7 Kesimpulan: Menuju Masa Depan Rekayasa yang Lebih Cerdas dan Bertanggung Jawab}}\label{kesimpulan-menuju-masa-depan-rekayasa-yang-lebih-cerdas-dan-bertanggung-jawab}

TISE telah menetapkan dasar yang kuat untuk rekayasa cerdas yang
berpusat pada manusia, sadar nilai, dan berkelanjutan. Namun, perjalanan
untuk sepenuhnya mewujudkan potensi paradigma ini masih panjang.
Tantangan dalam pengukuran, formalisasi, interoperabilitas, dan optimasi
kolaborasi manusia-AI menuntut upaya penelitian yang berkelanjutan dan
terkoordinasi. Dengan terus mengeksplorasi arah-arah ini, komunitas
rekayasa dapat memperkuat TISE, menciptakan artefak cerdas yang tidak
hanya canggih secara teknis tetapi juga bertanggung jawab secara etis
dan memberdayakan manusia untuk menjalani kehidupan yang lebih istimewa
di abad ke-21.

\begin{center}\rule{0.5\linewidth}{0.5pt}\end{center}

\bookmarksetup{startatroot}

\chapter{\texorpdfstring{\textbf{Bab 13: Etika dan Tanggung Jawab dalam
Rekayasa
TISE}}{Bab 13: Etika dan Tanggung Jawab dalam Rekayasa TISE}}\label{bab-13-etika-dan-tanggung-jawab-dalam-rekayasa-tise}

Tentu, berikut adalah draf Bab 13 dari buku Anda, yang membahas Etika
dan Tanggung Jawab dalam Rekayasa TISE:

\begin{center}\rule{0.5\linewidth}{0.5pt}\end{center}

Paradigma Triune-Intelligence Smart-Engineering (TISE) pada intinya
adalah sebuah filosofi rekayasa yang berpusat pada manusia, sadar nilai,
dan berkelanjutan. Setelah membahas metodologi inti TISE dan
implementasi praktisnya, bab ini akan mengeksplorasi dimensi etika dan
tanggung jawab yang melekat dalam kerangka TISE. Mengintegrasikan
Kecerdasan Buatan (AI) ke dalam sistem yang kompleks menimbulkan
berbagai pertanyaan etis yang harus dijawab secara proaktif. TISE
menyediakan landasan untuk memastikan bahwa teknologi tidak hanya kuat
dan cerdas, tetapi juga adil, transparan, akuntabel, dan memberdayakan
manusia.

\section{\texorpdfstring{\textbf{13.1 Filosofi TISE: Memberdayakan
Manusia dan Selaras dengan
Nilai}}{13.1 Filosofi TISE: Memberdayakan Manusia dan Selaras dengan Nilai}}\label{filosofi-tise-memberdayakan-manusia-dan-selaras-dengan-nilai}

Filosofi inti TISE adalah \textbf{``Engineers empower humans''}. Manusia
diposisikan sebagai pusat, subjek, dan tujuan akhir rekayasa cerdas,
bukan sebagai objek yang dieliminasi demi efisiensi. Tujuan utama TISE
melampaui sekadar memecahkan masalah teknis; ia bertujuan untuk
membangun ``teater kehidupan yang megah'' (\emph{splendid theaters of
life}) bagi umat manusia, yaitu lingkungan dan sistem yang memungkinkan
manusia menjalani eksistensi yang bermakna, kreatif, dan sejahtera
secara holistik.

Prinsip pemberdayaan ini memiliki implikasi etis yang mendalam: *
\textbf{Fokus pada Kapabilitas, Bukan Kekurangan}: Desainer harus
bekerja pada kapabilitas pelanggan, bukan pada kesenjangan atau
defisiensi mereka. Desain harus menjadi alat yang memfasilitasi cara
bagi orang untuk memenuhi kebutuhan mereka sendiri, memberikan solusi
seumur hidup. * \textbf{Teknologi sebagai Perluasan Diri}: Esensi
teknologi adalah memberdayakan kekuatan alam untuk membesarkan kapasitas
manusia, bukan untuk menggantikan mereka sepenuhnya. Ini berarti
teknologi yang dirancang dengan TISE harus selaras dengan tubuh manusia
(\emph{natural intelligence}) dan budaya (\emph{cultural intelligence}).

\section{\texorpdfstring{\textbf{13.2 Desain Peka Nilai (Value-Sensitive
Design) dalam
TISE}}{13.2 Desain Peka Nilai (Value-Sensitive Design) dalam TISE}}\label{desain-peka-nilai-value-sensitive-design-dalam-tise}

TISE secara eksplisit mengintegrasikan \textbf{Desain Peka Nilai
(Value-Sensitive Design - VSD)} sebagai metodologi implementasi. VSD
adalah pendekatan desain yang secara proaktif dan sistematis memasukkan
nilai-nilai kemanusiaan, terutama nilai etika dan moral, ke dalam
seluruh siklus hidup desain teknologi. TISE memperluas dan memformalkan
VSD dalam konteks rekayasa sistem cerdas dengan menyediakan mekanisme
konkret:

\begin{itemize}
\tightlist
\item
  \textbf{Homocordium sebagai Kompas Moral}: Komponen \textbf{Kecerdasan
  Manusia (Natural Intelligence - NI)}, yang disebut
  \textbf{Homocordium}, merepresentasikan dimensi ``hati'' manusia:
  nilai-nilai, etika, moralitas, empati, kreativitas, intuisi, dan
  tujuan spiritual. Homocordium berfungsi sebagai kompas moral dan
  sumber utama definisi ``masalah penting manusia'' yang perlu
  dipecahkan.
\item
  \textbf{Kecerdasan Kultural (CI) untuk Konteks Etis}:
  \textbf{Kecerdasan Kultural (CI)} menyediakan konteks, tujuan, nilai
  kolektif, dan kearifan masyarakat. CI membantu mengartikulasikan dan
  mengukur dimensi nilai yang sulit diukur, seperti modal sosial,
  kepercayaan, atau kesehatan ekologis, dan memasukkannya ke dalam
  kerangka PSKVE. Dalam proses validasi, CI memastikan bahwa hasil
  mencakup penerimaan pengguna, kepatuhan etis, dan adaptabilitas
  budaya.
\item
  \textbf{PUDAL Engine yang Sadar Nilai}: Mesin Triune-PUDAL yang
  ditingkatkan oleh NI, CI, dan AI dirancang secara eksplisit untuk
  memiliki kapasitas ``bertindak etis dan tidak bias secara budaya''.
  Sistem belajar tidak hanya untuk menjadi lebih akurat, tetapi juga
  untuk menjadi lebih ``baik''---lebih adil, lebih etis, dan lebih
  berkelanjutan.
\end{itemize}

TISE menjawab tantangan terbesar VSD, yaitu bagaimana menerapkannya pada
sistem skala besar dan kompleks seperti \emph{smart cities}. TISE
menawarkan solusi melalui dekomposisi \textbf{ASTF} untuk menganalisis
implikasi nilai di setiap lapisan, kerangka \textbf{PSKVE} untuk
menyeimbangkan keragaman nilai yang seringkali saling bertentangan
(misalnya, efisiensi vs.~privasi), dan kerangka \textbf{PICOC Berlapis}
untuk memvalidasi secara empiris apakah nilai-nilai yang diinginkan
benar-benar terwujud dan memberikan hasil yang terukur pada artefak yang
dibangun.

\section{\texorpdfstring{\textbf{13.3 Mengatasi Masalah Penyelarasan
Nilai (Value Alignment Problem)
AI}}{13.3 Mengatasi Masalah Penyelarasan Nilai (Value Alignment Problem) AI}}\label{mengatasi-masalah-penyelarasan-nilai-value-alignment-problem-ai}

Salah satu masalah etis paling krusial dalam pengembangan AI adalah
\textbf{masalah penyelarasan nilai} (\emph{value alignment problem}).
Ini adalah tantangan untuk memastikan bahwa tujuan dan perilaku sistem
AI selaras dengan nilai-nilai, etika, dan tujuan kemanusiaan. Kegagalan
dalam penyelarasan dapat menyebabkan AI menghasilkan keputusan yang
bias, tidak adil, atau bahkan berbahaya, meskipun secara teknis berhasil
mencapai tujuan yang diprogramkan.

TISE menawarkan solusi struktural untuk masalah ini. Alih-alih mencoba
``menyelaraskan'' AI sebagai langkah terakhir atau terpisah, TISE
\textbf{menanamkan nilai-nilai kemanusiaan sejak awal proses rekayasa
melalui \emph{Value-Driven Problem Formulation}}. Ini berarti
penyelarasan nilai bukanlah fitur tambahan, melainkan properti
fundamental yang melekat dalam arsitektur kognitif dan proses
pengembangan artefak itu sendiri. TISE mengubah \emph{value alignment}
dari sebuah masalah yang harus dipecahkan menjadi sebuah prinsip yang
harus dirancang sejak awal.

\section{\texorpdfstring{\textbf{13.4 Akuntabilitas, Transparansi, dan
Kolaborasi
Manusia-AI}}{13.4 Akuntabilitas, Transparansi, dan Kolaborasi Manusia-AI}}\label{akuntabilitas-transparansi-dan-kolaborasi-manusia-ai}

Dalam sistem yang semakin otonom, pertanyaan tentang siapa yang
bertanggung jawab ketika sistem membuat kesalahan menjadi sangat
penting. TISE menekankan akuntabilitas dan transparansi melalui:

\begin{itemize}
\tightlist
\item
  \textbf{Desain \emph{Human-in-the-Loop} (HITL)}: TISE mengadopsi
  arsitektur HITL dalam implementasi PUDAL, yang melibatkan manusia
  dalam berbagai tahapan, mulai dari \emph{pre-processing},
  \emph{in-the-loop (blocking)}, hingga \emph{post-processing} dan
  \emph{parallel feedback (non-blocking)}. Ini memastikan bahwa manusia
  memiliki pengawasan dan kemampuan intervensi pada titik-titik kritis.
\item
  \textbf{Kolaborasi Simbiotik}: TISE menyediakan kerangka kerja yang
  fleksibel untuk berbagai mode kolaborasi Manusia-AI (Human-Centric,
  Symbiotic, AI-Centric). Ini mengarah pada optimasi titik \emph{sweet
  spot} antara otomatisasi dan intervensi manusia, di mana keputusan AI
  untuk efisiensi dan kecepatan diimbangi dengan pengawasan dan
  intervensi NI untuk keamanan dan pertimbangan etis yang lebih tinggi
  {[}konversasi sebelumnya, tidak ada di new sources secara eksplisit,
  namun konsisten dengan filosofi{]}.
\item
  \textbf{Penjelasan (Explainability) AI}: AI dalam TISE berfokus pada
  \emph{digital twin} dan orkestrator keputusan dengan API
  \emph{explainability}. Kemampuan AI untuk menjelaskan
  pilihan-pilihannya dalam bahasa CI sangat penting untuk membangun
  kepercayaan pengguna dan memastikan transparansi.
\item
  \textbf{Komunikasi yang Kolokial}: Penting untuk tidak berasumsi bahwa
  pengguna akhir akrab dengan bahasa formal yang umum digunakan oleh
  kalangan teknis. Representasi visual dari kasus penggunaan harus
  menggunakan notasi yang lebih ``kolokial'' ketika diarahkan kepada
  pengguna akhir, yang dapat dihubungkan dengan informasi tentang
  perilaku sistem yang sesuai dengan setiap langkah tindakan pengguna.
  Ini adalah bagian dari tanggung jawab untuk membuat sistem mudah
  dipahami dan transparan bagi semua pemangku kepentingan.
\end{itemize}

\section{\texorpdfstring{\textbf{13.5 Mengelola Risiko dan Keamanan
dalam Sistem
Cerdas}}{13.5 Mengelola Risiko dan Keamanan dalam Sistem Cerdas}}\label{mengelola-risiko-dan-keamanan-dalam-sistem-cerdas}

Integrasi otonomi ke dalam \emph{System of Systems} (SoS) telah
menimbulkan tantangan dalam verifikasi dan validasi, terutama dengan
adanya masalah khusus dari sistem berbasis AI. Masalah yang belum
terpecahkan dalam keselamatan \emph{Machine Learning} (ML) terus menjadi
perhatian utama.

\begin{itemize}
\tightlist
\item
  \textbf{Validasi Berbasis Bukti}: Metodologi PICOC berlapis TISE
  menyediakan kerangka kerja yang ketat untuk validasi berbasis bukti,
  memastikan bahwa artefak cerdas TISE \emph{realistis} dan \emph{dapat
  dipercaya} dalam praktiknya. Setiap lapisan ASTF divalidasi dengan
  PICOC, menciptakan rantai bukti yang logis dari riset fundamental
  hingga dampak pada pengguna.
\item
  \textbf{Pengujian Komprehensif}: Peran AI dalam PICOC mencakup
  pembuatan pengujian otomatis (misalnya, menghasilkan skenario atau
  kasus ekstrem) dan deteksi anomali dalam hasil pengujian. Ini membantu
  mengidentifikasi potensi kegagalan dan kerentanan keamanan sebelum
  sistem diterapkan.
\item
  \textbf{Pertimbangan Konteks dan Sosial}: CI memastikan bahwa validasi
  mencakup penerimaan pengguna dan kepatuhan etis. Ini sangat penting
  dalam desain sistem AI-enabled, seperti sistem kedirgantaraan, yang
  memiliki tantangan rekayasa sistem yang signifikan.
\end{itemize}

\section{\texorpdfstring{\textbf{13.6 Peran Insinyur TISE sebagai
Vokator
Etis}}{13.6 Peran Insinyur TISE sebagai Vokator Etis}}\label{peran-insinyur-tise-sebagai-vokator-etis}

TISE mempersiapkan para insinyur dan peneliti untuk menjadi
\textbf{``Vokator''}---seseorang yang menyuarakan kebenaran yang
ditemukan melalui riset yang bertanggung jawab dan membangun kehidupan
istimewa di tengah disrupsi abad ke-21. Menjadi seorang Vokator dalam
konteks TISE berarti: * \textbf{Memahami Esensi Disrupsi}: Vokator
memahami esensi disrupsi dan mampu menuntun masyarakat menghadapinya. *
\textbf{Manusia Hati (Homocordium)}: Dalam menghadapi disrupsi teknologi
yang menghilangkan nilai dari kekuatan otot dan keterampilan berpikir,
Vokator adalah manusia hati, \emph{Homocordium}. Ini menekankan dimensi
etis dan moral sebagai inti dari identitas insinyur TISE.

Dengan demikian, etika dan tanggung jawab bukanlah tambahan opsional
dalam Rekayasa TISE, melainkan merupakan benang merah yang terjalin erat
dalam setiap aspek filosofi, metodologi, dan implementasinya. TISE
membimbing para insinyur untuk menciptakan sistem yang tidak hanya
secara teknis canggih tetapi juga secara fundamental etis, bertanggung
jawab, dan memberdayakan seluruh umat manusia.

\begin{center}\rule{0.5\linewidth}{0.5pt}\end{center}

\bookmarksetup{startatroot}

\chapter{\texorpdfstring{\textbf{Bab 14: Kesimpulan: Visi Rekayasa
Simbiotik dan Pemberdayaan
Manusia}}{Bab 14: Kesimpulan: Visi Rekayasa Simbiotik dan Pemberdayaan Manusia}}\label{bab-14-kesimpulan-visi-rekayasa-simbiotik-dan-pemberdayaan-manusia}

Tentu, berikut adalah draf Bab 14 dari buku Anda, yang berfungsi sebagai
kesimpulan komprehensif:

\begin{center}\rule{0.5\linewidth}{0.5pt}\end{center}

Kita telah menempuh perjalanan yang panjang dan mendalam melalui
paradigma \textbf{Triune-Intelligence Smart-Engineering (TISE)}. Dimulai
dari identifikasi kebutuhan akan pendekatan rekayasa yang lebih holistik
di tengah kompleksitas Abad ke-21, hingga menjelajahi fondasi filosofis,
metodologi yang ketat, implementasi praktis, serta tantangan dan
implikasi etisnya. Bab terakhir ini akan merangkum visi inti TISE,
mengulang kembali pilar-pilar utamanya, dan menegaskan kembali perannya
dalam membentuk masa depan rekayasa yang lebih cerdas, bertanggung
jawab, dan berpusat pada manusia.

\section{\texorpdfstring{\textbf{14.1 Mengulang Kembali Visi Inti TISE:
Rekayasa untuk Kehidupan
Istimewa}}{14.1 Mengulang Kembali Visi Inti TISE: Rekayasa untuk Kehidupan Istimewa}}\label{mengulang-kembali-visi-inti-tise-rekayasa-untuk-kehidupan-istimewa}

Pada intinya, TISE adalah sebuah manifestasi dari keyakinan bahwa
\textbf{``Engineers empower humans''}. Ini bukan sekadar moto; ini
adalah komitmen filosofis yang mendasari setiap aspek paradigma ini.
TISE menggeser fokus rekayasa dari sekadar \emph{problem-solving}
menjadi \emph{capability-building}, dengan tujuan utama menciptakan
``teater kehidupan yang megah'' (\emph{splendid theaters of life}) bagi
umat manusia {[}Prakata buku{]}. Ini berarti merancang lingkungan dan
sistem yang tidak hanya mengatasi masalah, tetapi juga memperluas
potensi manusia untuk menjalani kehidupan yang istimewa, kreatif, aman,
sehat, dan berkelanjutan.

\section{\texorpdfstring{\textbf{14.2 Pilar-Pilar Utama TISE: Sebuah
Sintesis}}{14.2 Pilar-Pilar Utama TISE: Sebuah Sintesis}}\label{pilar-pilar-utama-tise-sebuah-sintesis}

TISE menyajikan pendekatan yang \textbf{seimbang, partisipatif, dan
selaras nilai} untuk rekayasa di era AI, mengintegrasikan tiga pilar
kecerdasan secara eksplisit:

\begin{itemize}
\tightlist
\item
  \textbf{Kecerdasan Alami (Natural Intelligence - NI)}: Representasi
  hati nurani, nilai, etika, dan kreativitas manusia (Homocordium). NI
  memberikan \textbf{``WHAT''}---kemampuan pengambilan keputusan dan
  kontribusi orisinal.
\item
  \textbf{Kecerdasan Budaya (Cultural Intelligence - CI)}: Representasi
  kearifan kolektif, tujuan, dan nilai-nilai sosial. CI memberikan
  \textbf{``WHY''}---persepsi, pemahaman, dan tata kelola.
\item
  \textbf{Kecerdasan Buatan (Artificial Intelligence - AI)}:
  Representasi kekuatan komputasi, penalaran logis, dan eksekusi. AI
  memberikan \textbf{``HOW''}---bertindak, eksekusi, dan komputasi.
\end{itemize}

Interplay dinamis dari ketiga kecerdasan ini membentuk \textbf{Triune
Intelligence (TI)}, sebuah sistem hibrida yang menghasilkan intelijen
holistik dan \emph{emergent}, jauh melampaui penjumlahan
bagian-bagiannya.

Paradigma ini termanifestasi dalam \textbf{Enam Karakteristik Artefak
Cerdas TISE}, yang memastikan setiap rancang bangun: 1. \textbf{Strong
(Kuat)}, melalui Core Engine. 2. \textbf{Smart (Cerdas)}, melalui
Triune-PUDAL Cognitive Engine. 3. \textbf{Extended Range (Jangkauan
Luas)}, melalui PSKVE Value Engine. 4. \textbf{Realistic (Realistis)},
melalui PICOC Systematic. 5. \textbf{Doable (Dapat Dilaksanakan)},
melalui Triune-ASTF Four-Layered Architecture. 6. \textbf{Methodic
(Metodis)}, melalui W-Model (pengembangan dari V-Method).

Metodologi TISE, yang diuraikan dalam buku ini, memberikan sebuah peta
jalan yang terstruktur untuk penelitian dan pengembangan: *
\textbf{Arsitektur ASTF (Application, System, Technology, Fundamental
Research)} berfungsi sebagai kerangka dekomposisi kompleksitas,
memungkinkan inovasi di setiap lapisan, dari kebutuhan pemangku
kepentingan hingga prinsip-prinsip ilmiah fundamental. *
\textbf{Triune-PUDAL Cognitive Engine} (Perceive, Understand,
Decision-making \& Planning, Act-Response, Learning-evaluating) adalah
inti kognitif yang memungkinkan artefak belajar dan beradaptasi, dengan
setiap fasenya ditingkatkan oleh kolaborasi NI, CI, dan AI. *
\textbf{PSKVE Value Engine} (Product, Service, Knowledge, Value,
Environmental) memastikan penciptaan nilai holistik, menyeimbangkan
berbagai dimensi untuk keberlanjutan dan dampak yang lebih luas. *
\textbf{PICOC Systematic (Population, Intervention, Control, Outcome,
Context)} diterapkan secara berlapis di setiap lapisan ASTF, menyediakan
metodologi validasi berbasis bukti yang ketat dan memastikan
keterandalan serta validitas klaim penelitian. * \textbf{W-Model}
mengintegrasikan proses desain, sintesis, dan validasi secara
berkelanjutan, memastikan keterlacakan dan kelayakan pengembangan
artefak cerdas yang kompleks.

\section{\texorpdfstring{\textbf{14.3 Peran Alat Bantu dan Implementasi
Praktis}}{14.3 Peran Alat Bantu dan Implementasi Praktis}}\label{peran-alat-bantu-dan-implementasi-praktis}

Pemanfaatan alat bantu modern sangat esensial dalam mewujudkan TISE.
\textbf{Python} berfungsi sebagai bahasa pemrograman utama untuk
implementasi AI, simulasi, dan integrasi, mendukung berbagai fase PUDAL
dan evaluasi PICOC. \textbf{Ontologi} dan \textbf{Prolog} memungkinkan
formalisasi pengetahuan manusia (NI) dan konteks budaya (CI) ke dalam
bentuk yang dapat diproses AI, memfasilitasi penalaran logis dan
semantik umum antar sistem. \textbf{Quarto} dan \textbf{Mermaid}
(meskipun tidak secara eksplisit disebutkan dalam semua sumber, tetapi
relevan untuk konteks ini) memungkinkan dokumentasi yang komprehensif
dan visualisasi yang jelas dari arsitektur sistem dan alur kerja, yang
krusial untuk komunikasi ilmiah dan reproduktifitas penelitian.

Studi kasus, seperti Sistem Komuter Cerdas Jakarta-Bandung dan Sistem
Rekomendasi Makanan Sehat (MSRS), telah mendemonstrasikan bagaimana
kerangka TISE dan alat bantu ini dapat diterapkan untuk mengatasi
masalah sosio-teknis yang kompleks, mulai dari dekomposisi masalah
hingga optimasi konversi nilai PSKVE.

\section{\texorpdfstring{\textbf{14.4 Implikasi Etika dan Tanggung
Jawab: Menciptakan
Vokator}}{14.4 Implikasi Etika dan Tanggung Jawab: Menciptakan Vokator}}\label{implikasi-etika-dan-tanggung-jawab-menciptakan-vokator}

Dimensi etika dan tanggung jawab adalah inti dari TISE, bukan sekadar
pelengkap. Paradigma ini secara proaktif mengatasi masalah penyelarasan
nilai AI (\emph{AI value alignment}) dengan menanamkan nilai-nilai
kemanusiaan dan budaya sejak awal proses rekayasa. Dengan
\textbf{Homocordium} sebagai kompas moral dan \textbf{Kecerdasan
Kultural (CI)} yang membentuk konteks etis, TISE memastikan sistem tidak
hanya cerdas tetapi juga adil, transparan, dan akuntabel.

Melalui filosofi ``Engineers empower humans'', TISE membentuk para
insinyur dan peneliti menjadi \textbf{``Vokator''}---individu yang
memahami esensi disrupsi, menyuarakan kebenaran melalui riset yang
bertanggung jawab, dan membangun kehidupan istimewa di tengah Abad ke-21
yang penuh gejolak. Vokator adalah ``manusia hati''
(\emph{homocordium}), yang tidak hanya menguasai teknologi tetapi juga
memegang teguh nilai-nilai kemanusiaan di era di mana kekuatan otot dan
keterampilan berpikir mungkin digantikan oleh teknologi.

\section{\texorpdfstring{\textbf{14.5 Menuju Masa Depan Rekayasa yang
Lebih Cerdas dan Bertanggung
Jawab}}{14.5 Menuju Masa Depan Rekayasa yang Lebih Cerdas dan Bertanggung Jawab}}\label{menuju-masa-depan-rekayasa-yang-lebih-cerdas-dan-bertanggung-jawab}

Meskipun TISE telah menetapkan landasan yang kuat, perjalanan untuk
sepenuhnya mewujudkan potensinya masih terus berlanjut. Bab 12
menguraikan tantangan seperti pengukuran metrik CI yang ``lunak'',
pengembangan taksonomi komprehensif untuk \emph{System of Autonomous
Systems} (SoAS) yang mempertimbangkan otonomi manajerial dan
operasional, perlunya kerangka kerja formal untuk penutupan sistem
informasional, serta peningkatan interoperabilitas alat dan
standardisasi bahasa pemodelan. Tantangan-tantangan ini, bukan sebagai
penghalang, melainkan sebagai undangan untuk penelitian dan inovasi
berkelanjutan.

CISE menawarkan visi rekayasa simbiotik---sebuah masa depan di mana
Kecerdasan Alami, Budaya, dan Buatan berinteraksi secara harmonis, tidak
untuk mendominasi, tetapi untuk memberdayakan dan memperkaya pengalaman
manusia. Ini adalah ajakan untuk menciptakan sistem yang tidak hanya
canggih secara teknis tetapi juga berharga secara holistik, berlandaskan
etika, dan diterapkan secara bertanggung jawab, guna membentuk
``panggung'' kehidupan yang luar biasa bagi seluruh umat manusia.

\begin{center}\rule{0.5\linewidth}{0.5pt}\end{center}

\bookmarksetup{startatroot}

\chapter{\texorpdfstring{\textbf{Lampiran A: Studi Kasus
Rinci}}{Lampiran A: Studi Kasus Rinci}}\label{lampiran-a-studi-kasus-rinci}

Tentu, berikut adalah Lampiran A untuk buku Anda, yang berisi studi
kasus rinci tentang penerapan paradigma TISE:

\begin{center}\rule{0.5\linewidth}{0.5pt}\end{center}

Lampiran ini menyediakan dua studi kasus yang dianalisis secara mendalam
menggunakan kerangka TISE untuk memberikan contoh nyata penerapan
paradigma ini. Studi kasus ini mengilustrasikan bagaimana Triune
Intelligence, PUDAL, PSKVE, dan ASTF bekerja sama untuk mengatasi
masalah sosio-teknis yang kompleks.

\section{\texorpdfstring{\textbf{A.1: Dekomposisi ASTF untuk Sistem
Komuter Cerdas
Jakarta-Bandung}}{A.1: Dekomposisi ASTF untuk Sistem Komuter Cerdas Jakarta-Bandung}}\label{a.1-dekomposisi-astf-untuk-sistem-komuter-cerdas-jakarta-bandung}

Studi kasus ini mendemonstrasikan bagaimana masalah rekayasa yang sangat
besar dan kompleks dapat dipecah menjadi bagian-bagian yang dapat
dikelola menggunakan kerangka Arsitektur Empat Lapisan TISE
(Application, System, Technology, Fundamental Research - ASTF). Contoh
ini juga telah digunakan dalam Bab 4 untuk mengilustrasikan kerangka
ASTF.

\begin{itemize}
\tightlist
\item
  \textbf{Lapisan Aplikasi (A): Solusi Berbagi Kamar-Makanan-Perjalanan}

  \begin{itemize}
  \tightlist
  \item
    \textbf{Masalah}: Pada tahun 2030, jutaan komuter akan melakukan
    perjalanan harian antara Jakarta dan Bandung. Perjalanan ini
    melelahkan dan memakan waktu, menciptakan kebutuhan mendesak akan
    solusi untuk istirahat, makan, dan perjalanan yang efisien.
  \item
    \textbf{Solusi yang Diusulkan}: Sebuah model bisnis
    \textbf{``berbagi kamar-makanan-perjalanan''}
    (\emph{room-food-travel sharing}) yang terintegrasi untuk
    menyediakan akomodasi sementara (kapsul tidur), makanan yang nyaman
    (siap saji), dan perjalanan yang efisien dalam satu platform.
  \item
    \textbf{Peran Triune Intelligence}:

    \begin{itemize}
    \tightlist
    \item
      \textbf{Kecerdasan Kultural (CI)}: Memahami kebutuhan dan
      preferensi komuter terkait kenyamanan, privasi, dan kebersihan
      yang selaras dengan norma sosial dan budaya.
    \item
      \textbf{Kecerdasan Alami (NI)}: Terlibat dalam desain \emph{user
      experience} (UX) untuk memastikan antarmuka yang intuitif dan
      memenuhi harapan manusia.
    \item
      \textbf{Kecerdasan Buatan (AI)}: Mendukung personalisasi
      rekomendasi kamar, makanan, dan rute berdasarkan riwayat dan
      preferensi pengguna.
    \end{itemize}
  \end{itemize}
\item
  \textbf{Lapisan Sistem (S): Sistem Kapsul Tidur Komuter dan Makanan
  Siap Saji}

  \begin{itemize}
  \tightlist
  \item
    \textbf{Arsitektur Sistem}: Sebuah sistem terintegrasi yang terdiri
    dari: (1) Kapsul tidur (\emph{Sleep-in-Capsules}) yang kompak dan
    nyaman di \emph{hub} transportasi. (2) Layanan makanan siap saji
    (\emph{Food-to-Go}) yang dapat dipesan sebelumnya. (3) Integrasi
    yang mulus dengan jaringan transportasi (kereta cepat, bus cerdas).
  \item
    \textbf{Fungsi}: Mengubah \emph{input} (permintaan komuter, energi,
    bahan makanan) menjadi \emph{output} (istirahat, nutrisi, perjalanan
    efisien).
  \item
    \textbf{Peran Triune Intelligence}:

    \begin{itemize}
    \tightlist
    \item
      \textbf{CI}: Menentukan standar kebersihan, keamanan, dan
      \emph{branding} layanan, serta tata kelola data yang sensitif.
    \item
      \textbf{NI}: Mengawasi operasi sistem, menangani pengecualian atau
      situasi darurat yang tidak dapat ditangani AI.
    \item
      \textbf{AI}: Mengelola alokasi kapsul secara dinamis,
      mengoptimalkan rute transportasi, dan memproses pesanan makanan
      \emph{real-time}.
    \end{itemize}
  \end{itemize}
\item
  \textbf{Lapisan Teknologi (T): Mesin Listrik, Dapur Otomatis, dan
  Keuangan Digital}

  \begin{itemize}
  \tightlist
  \item
    \textbf{Mesin Kunci}: (1) \textbf{Mesin Listrik}: Untuk menggerakkan
    kendaraan transportasi secara efisien dan berkelanjutan. (2)
    \textbf{Dapur Otomatis}: Untuk produksi makanan massal yang cepat,
    konsisten, dan higienis. (3) \textbf{Uang Digital}: Untuk transaksi
    yang lancar dan tanpa gesekan di seluruh ekosistem. (4)
    \textbf{Platform Pembiayaan Digital}: Untuk mengelola arus kas,
    investasi, dan memastikan keberlanjutan ekonomi jangka panjang dari
    seluruh sistem.
  \item
    \textbf{Peran Triune Intelligence}:

    \begin{itemize}
    \tightlist
    \item
      \textbf{CI}: Menentukan persyaratan keamanan siber untuk uang
      digital, standar kebersihan dapur, dan keberlanjutan energi yang
      selaras dengan kebijakan publik.
    \item
      \textbf{NI}: Terlibat dalam desain dan \emph{tuning} mesin
      listrik, kalibrasi dapur otomatis, serta pengembangan algoritma
      keamanan untuk transaksi digital.
    \item
      \textbf{AI}: Mengoptimalkan kinerja mesin listrik (misalnya,
      efisiensi energi), mengontrol proses dapur otomatis, dan memproses
      transaksi uang digital secara aman dan efisien.
    \end{itemize}
  \end{itemize}
\item
  \textbf{Lapisan Riset Fundamental (F): Prinsip Optimasi dan Konversi
  Nilai}

  \begin{itemize}
  \tightlist
  \item
    \textbf{Prinsip yang Diteliti}: (1) \textbf{Tingkat Jam Kerja
    Manusia Minimum}: Prinsip ekonomi industri untuk meminimalkan
    \emph{input} tenaga kerja manusia tanpa mengorbankan kualitas. (2)
    \textbf{Optimasi Jadwal Waktu}: Prinsip dari riset operasi untuk
    memaksimalkan \emph{throughput} dan meminimalkan waktu tunggu. (3)
    \textbf{Alokasi Sumber Daya}: Prinsip dari teori sistem untuk
    mendistribusikan sumber daya terbatas (kapsul, bahan makanan,
    energi) secara optimal. (4) \textbf{Konversi Nilai PSKVE}: Teori
    fundamental dari TISE tentang bagaimana berbagai bentuk ``energi''
    (Produk, Layanan, Pengetahuan, Nilai, Lingkungan) dapat
    ditransaksikan untuk menciptakan nilai holistik.
  \item
    \textbf{Peran Triune Intelligence}:

    \begin{itemize}
    \tightlist
    \item
      \textbf{CI}: Memandu arah penelitian untuk menemukan prinsip yang
      mendukung keberlanjutan dan keadilan sosial dalam alokasi sumber
      daya.
    \item
      \textbf{NI}: Memformulasikan hipotesis, merancang eksperimen
      teoretis, dan menginterpretasikan hasil untuk mengembangkan
      pengetahuan baru.
    \item
      \textbf{AI}: Mendukung simulasi kompleks untuk menguji model
      optimasi, menganalisis data besar untuk menemukan pola baru dalam
      konversi nilai, atau memprediksi perilaku sistem yang kompleks
      (misalnya, lalu lintas).
    \end{itemize}
  \end{itemize}
\end{itemize}

\section{\texorpdfstring{\textbf{A.2: Analisis Sistem Rekomendasi
Makanan Sehat dengan Kerangka
TISE}}{A.2: Analisis Sistem Rekomendasi Makanan Sehat dengan Kerangka TISE}}\label{a.2-analisis-sistem-rekomendasi-makanan-sehat-dengan-kerangka-tise}

Studi kasus ini adalah contoh sempurna dari penerapan TISE pada masalah
sosio-teknis yang sangat kompleks, menunjukkan bagaimana \textbf{Triune
Intelligence, PUDAL, dan PSKVE} bekerja sama.

\begin{itemize}
\tightlist
\item
  \textbf{Masalah (Konteks PSKVE)}: Memenuhi kebutuhan pangan lokal yang
  sehat menghadapi tantangan multi-dimensi:

  \begin{itemize}
  \tightlist
  \item
    \textbf{Produk \& Layanan}: Sulit menghasilkan makanan yang sesuai
    dengan profil kesehatan spesifik (misalnya, diabetes, hipertensi)
    dan selera lokal.
  \item
    \textbf{Nilai}: Makanan sehat seringkali tidak terjangkau karena
    rantai pasok yang tidak efisien.
  \item
    \textbf{Lingkungan}: Praktik pertanian dan peternakan seringkali
    tidak berkelanjutan.
  \item
    \textbf{Pengetahuan}: Kurangnya data tentang kebutuhan konsumen dan
    inovasi produk lokal yang sehat.
  \end{itemize}
\item
  \textbf{Solusi (Berbasis Triune Intelligence)}: Sebuah
  \emph{marketplace} makanan sehat yang berfungsi sebagai ekosistem yang
  menghubungkan berbagai pemangku kepentingan. Triune Intelligence
  diwujudkan melalui kolaborasi antara:

  \begin{itemize}
  \tightlist
  \item
    \textbf{Kecerdasan Manusia (NI)}: Konsumen (dengan kebutuhan dan
    preferensinya), Chef/Ahli Kuliner (dengan kreativitas dan kearifan
    lokalnya), UMKM Pangan (dengan kapasitas produksinya), dan Otoritas
    Gizi \& Agama (dengan standar dan pedomannya).
  \item
    \textbf{Kecerdasan Buatan (AI)}: Sistem Rekomendasi
    Multi-Stakeholder (MSRS) yang akan kita bedah.
  \item
    \textbf{Kecerdasan Alam (Natural Intelligence)}: Fokus pada
    penggunaan pasokan bahan baku lokal yang berkelanjutan (\emph{local
    supplies}).
  \end{itemize}
\item
  \textbf{Mekanisme (Mesin PUDAL \& PSKVE dalam Aksi)}: Operator
  \emph{marketplace} menggunakan MSRS (sebagai komponen AI) untuk
  membuat keputusan strategis dalam siklus PUDAL:

  \begin{itemize}
  \tightlist
  \item
    \textbf{Perceive}: MSRS mengumpulkan data dari semua pemangku
    kepentingan: pencarian tak terjawab dari konsumen, resep inovatif
    dari \emph{chef}, data kapasitas dari UMKM, dan panduan dari
    otoritas gizi/agama.
  \item
    \textbf{Understand}: AI dalam MSRS menganalisis data ini untuk
    mengidentifikasi ``celah pasar''---misalnya, permintaan tinggi untuk
    ``camilan sore rendah gula'' yang belum terpenuhi. Kecerdasan
    Kultural (CI) membantu dalam memahami signifikansi dari celah pasar
    ini dalam konteks norma dan preferensi diet lokal.
  \item
    \textbf{Decision-making}: MSRS kemudian mencocokkan celah ini dengan
    resep yang ada (misalnya, resep ``Bola Ubi Ungu Gluten-Free'' dari
    seorang \emph{chef}) dan kapasitas produksi (misalnya, ``UMKM Sehat
    Selalu'' memiliki pengalaman dengan produk bebas gluten). Sistem
    kemudian memberikan rekomendasi kepada operator. Kecerdasan Manusia
    (NI) operator membuat keputusan akhir, dengan informasi dari AI dan
    panduan dari CI.
  \item
    \textbf{Act}: Operator, berdasarkan rekomendasi, memfasilitasi
    kolaborasi antara \emph{chef} dan UMKM untuk memproduksi item baru
    tersebut dan memasukkannya ke dalam menu.
  \item
    \textbf{Learning}: Kinerja penjualan dan ulasan dari item baru ini
    menjadi masukan baru bagi MSRS, menciptakan siklus perbaikan
    berkelanjutan. AI mengukur metrik teknis, CI mengevaluasi terhadap
    tujuan nilai dan norma budaya, dan NI merefleksikan keberhasilan
    keputusannya.
  \end{itemize}
\end{itemize}

Sistem ini secara aktif mengelola konversi PSKVE: \textbf{Inovasi
(Knowledge)} dari \emph{chef} diubah menjadi \textbf{Product} baru, yang
memberikan \textbf{Service} kesehatan kepada konsumen, yang menghasilkan
\textbf{Value} ekonomi bagi UMKM dan operator, semuanya dengan tujuan
menggunakan bahan baku yang berkelanjutan (\textbf{Environmental}).

\begin{center}\rule{0.5\linewidth}{0.5pt}\end{center}

\bookmarksetup{startatroot}

\chapter{\texorpdfstring{\textbf{Lampiran B: Templat dan Daftar
Periksa}}{Lampiran B: Templat dan Daftar Periksa}}\label{lampiran-b-templat-dan-daftar-periksa}

Tentu, berikut adalah Lampiran B untuk buku Anda, yang berisi alat bantu
praktis untuk mahasiswa:

\begin{center}\rule{0.5\linewidth}{0.5pt}\end{center}

Lampiran ini berisi alat bantu praktis yang dapat langsung digunakan
oleh mahasiswa untuk membantu merancang penelitian, menyusun proposal,
dan menyiapkan publikasi ilmiah dalam kerangka paradigma
Triune-Intelligence Smart-Engineering (TISE).

\section{\texorpdfstring{\textbf{B.1: Templat Kosong Kerangka PICOC
Berlapis}}{B.1: Templat Kosong Kerangka PICOC Berlapis}}\label{b.1-templat-kosong-kerangka-picoc-berlapis}

Templat ini dapat digunakan untuk merancang kerangka validasi penelitian
Anda di setiap lapisan ASTF, sebagaimana dijelaskan dalam Bab 7
(Metodologi Validasi PICOC Sistematis) dan Tabel 6.1 dalam buku. Ini
akan membantu Anda merumuskan pertanyaan penelitian dan merencanakan
studi empiris secara sistematis.

\textbf{Tabel B.1: Templat PICOC Berlapis untuk Riset Disertasi}

\begin{longtable}[]{@{}
  >{\raggedright\arraybackslash}p{(\linewidth - 8\tabcolsep) * \real{0.0831}}
  >{\raggedright\arraybackslash}p{(\linewidth - 8\tabcolsep) * \real{0.2418}}
  >{\raggedright\arraybackslash}p{(\linewidth - 8\tabcolsep) * \real{0.2217}}
  >{\raggedright\arraybackslash}p{(\linewidth - 8\tabcolsep) * \real{0.2217}}
  >{\raggedright\arraybackslash}p{(\linewidth - 8\tabcolsep) * \real{0.2317}}@{}}
\toprule\noalign{}
\begin{minipage}[b]{\linewidth}\raggedright
Komponen PICOC
\end{minipage} & \begin{minipage}[b]{\linewidth}\raggedright
\textbf{Lapisan A: Aplikasi (Application Layer)}
\end{minipage} & \begin{minipage}[b]{\linewidth}\raggedright
\textbf{Lapisan S: Sistem (System Layer)}
\end{minipage} & \begin{minipage}[b]{\linewidth}\raggedright
\textbf{Lapisan T: Teknologi (Technology Layer)}
\end{minipage} & \begin{minipage}[b]{\linewidth}\raggedright
\textbf{Lapisan F: Riset Fundamental (Fundamental Research Layer)}
\end{minipage} \\
\midrule\noalign{}
\endhead
\bottomrule\noalign{}
\endlastfoot
\textbf{P (Populasi/Masalah/Proses)} & Pemangku kepentingan atau masalah
spesifik yang ditargetkan. & Kumpulan data uji, \emph{testbed},
lingkungan simulasi, atau dataset. & Sumber energi, nilai masukan, data
mentah, atau material. & Fenomena, entitas fundamental, atau nilai
sumber yang diselidiki. \\
\textbf{I (Intervensi/Peningkatan/Investigasi)} & Solusi baru yang
diusulkan. & Sistem yang diusulkan yang menggabungkan teknologi kunci. &
Mesin/modul teknologi dengan instrumen atau metode baru. & Proses,
teori, model, atau pendekatan eksperimental baru. \\
\textbf{C (Kontrol/Perbandingan)} & Solusi lama atau praktik saat ini
yang digunakan oleh pemangku kepentingan. & Sistem yang ada atau sistem
dasar (\emph{baseline}). & Mesin/modul teknologi lama atau
instrumen/metode yang ada. & Proses, teori, model, atau pendekatan
eksperimental lama/yang sudah ada. \\
\textbf{O (Outcome/Luaran)} & Peningkatan kinerja yang dirasakan
(misalnya, efisiensi, pengurangan biaya, kepuasan). & Peningkatan
kinerja tingkat sistem (misalnya, akurasi, kecepatan, keandalan). &
Peningkatan kinerja dalam melakukan konversi atau tugas spesifik
(misalnya, efisiensi, presisi). & Pengetahuan baru, pemahaman yang lebih
dalam, prinsip yang tervalidasi, atau efek baru. \\
\textbf{Cx (Konteks)} & Masalah spesifik yang harus dipecahkan dan
persyaratan solusi yang berhasil. & Kebutuhan akan sistem yang lebih
baik yang memenuhi persyaratan solusi. & Tantangan untuk menemukan atau
meningkatkan instrumen/metode. & Pencarian pengetahuan baru, pemahaman
mekanisme fundamental, atau eksplorasi wilayah ilmiah. \\
\end{longtable}

\section{\texorpdfstring{\textbf{B.2: Daftar Periksa Proposal Riset
TISE}}{B.2: Daftar Periksa Proposal Riset TISE}}\label{b.2-daftar-periksa-proposal-riset-tise}

Daftar periksa ini dirancang untuk membantu Anda menyusun proposal riset
yang komprehensif dan selaras dengan paradigma TISE, mencakup
aspek-aspek kunci yang dibahas dalam Bab 9.

\textbf{Bagian 1: Definisi Masalah (Lapisan A)} * {[} {]} Masalah telah
didefinisikan dengan jelas dari perspektif pemangku kepentingan. * {[}
{]} Pemangku kepentingan utama (langsung dan tidak langsung) telah
diidentifikasi. * {[} {]} Solusi yang ada (Kontrol) telah dianalisis dan
kelemahannya telah diidentifikasi. * {[} {]} Metrik keberhasilan dari
sudut pandang pengguna (Outcome) telah ditetapkan.

\textbf{Bagian 2: Usulan Intervensi (Semua Lapisan)} * {[} {]}
Kontribusi orisinal di setiap lapisan ASTF yang relevan telah dinyatakan
dengan jelas. * {[} {]} Peta ASTF untuk intervensi yang diusulkan telah
dibuat. * {[} {]} Hipotesis atau pertanyaan penelitian untuk setiap
lapisan telah dirumuskan menggunakan format PICOC.

\textbf{Bagian 3: Metodologi} * {[} {]} Kerangka W-Model untuk proses
penelitian telah diuraikan. * {[} {]} Desain eksperimen/studi untuk
setiap validasi PICOC telah dirinci (termasuk populasi, prosedur, dan
metrik). * {[} {]} Alat, perangkat lunak, dan sumber daya yang
dibutuhkan telah diidentifikasi.

\section{\texorpdfstring{\textbf{B.3: Daftar Periksa Pra-pengiriman
Paper
IEEE}}{B.3: Daftar Periksa Pra-pengiriman Paper IEEE}}\label{b.3-daftar-periksa-pra-pengiriman-paper-ieee}

Daftar periksa ini akan membantu Anda memastikan paper Anda memenuhi
standar publikasi ilmiah, terutama untuk jurnal atau konferensi IEEE,
dengan fokus pada struktur, konten, kualitas, dan koherensi, sesuai
dengan pedoman yang diuraikan dalam Bab 9 dan sumber eksternal.

\textbf{Struktur dan Konten} * {[} {]} Judul ringkas, informatif, dan
mencerminkan kontribusi utama. * {[} {]} Abstrak mengikuti struktur 6
poin (Konteks, Masalah, Intervensi, Metodologi, Hasil, Dampak). * {[}
{]} Pendahuluan dengan jelas menyatakan masalah, solusi lama, solusi
baru, dan kontribusi. * {[} {]} Tinjauan pustaka secara kritis
mengidentifikasi kesenjangan penelitian. * {[} {]} Bagian metode cukup
rinci untuk direplikasi. * {[} {]} Hasil disajikan dengan jelas
menggunakan gambar dan tabel. * {[} {]} Pembahasan menghubungkan hasil
teknis dengan implikasi praktis. * {[} {]} Kesimpulan merangkum temuan
dan menyarankan langkah selanjutnya.

\textbf{Kualitas dan Koherensi} * {[} {]} ``Benang merah'' paper jelas
dan konsisten. * {[} {]} Argumen didukung oleh data dari validasi PICOC.
* {[} {]} Hubungan antar lapisan ASTF (jika relevan) dijelaskan. * {[}
{]} Semua referensi diformat sesuai standar IEEE dan dikutip dengan
benar. * {[} {]} Paper telah diperiksa tata bahasa, ejaan, dan
kejelasannya.

\begin{center}\rule{0.5\linewidth}{0.5pt}\end{center}

\bookmarksetup{startatroot}

\chapter{\texorpdfstring{\textbf{Lampiran C: Detail Metrik dan Model
Kinerja
TISE}}{Lampiran C: Detail Metrik dan Model Kinerja TISE}}\label{lampiran-c-detail-metrik-dan-model-kinerja-tise}

Tentu, berikut adalah Lampiran C untuk buku Anda, yang berisi detail
metrik dan model kinerja Triune-Intelligence Smart-Engineering (TISE):

\begin{center}\rule{0.5\linewidth}{0.5pt}\end{center}

Lampiran ini menyediakan detail teknis dan model kinerja yang mendalam
untuk konsep-konsep inti TISE yang dibahas di bagian utama buku. Ini
termasuk pengukuran kapasitas Triune Intelligence (TI), model kinerja
Product-Service-Knowledge-Value-Environmental System (PSKVE), dan konsep
terkait lainnya yang memperkuat kerangka TISE.

\section{\texorpdfstring{\textbf{C.1: Metrik Kapasitas Triune (Triune
Capacity Index -
TCI)}}{C.1: Metrik Kapasitas Triune (Triune Capacity Index - TCI)}}\label{c.1-metrik-kapasitas-triune-triune-capacity-index---tci}

Sebagaimana dibahas di Bab 5 (Siklus Kognitif PUDAL Engine), pengukuran
efektivitas dan kapasitas Triune-PUDAL Engine membutuhkan metrik yang
melampaui kinerja teknis AI semata, mencakup dimensi Natural
Intelligence (NI) dan Cultural Intelligence (CI). \textbf{Triune
Capacity Index (TCI)} adalah metrik komposit yang menggabungkan berbagai
indikator untuk memberikan gambaran holistik tentang kapasitas sistem
TISE.

TCI dihitung sebagai rata-rata tertimbang dari beberapa klaster metrik.
Jika Anda membutuhkan satu KPI utama, TCI dapat dihitung menggunakan
formula berikut:
\(\text{TCI}=w_A \cdot \overline{A}+w_B \cdot \overline{B}+w_C \cdot \overline{C}+w_D \cdot \overline{D}+w_E \cdot \overline{E},\)
di mana
\(\overline{A}, \overline{B}, \overline{C}, \overline{D}, \overline{E}\)
adalah skor rata-rata untuk klaster metrik yang berbeda, dan
\(w_A, w_B, w_C, w_D, w_E\) adalah bobot yang sesuai.

Beberapa klaster metrik yang relevan untuk TCI meliputi:

\begin{enumerate}
\def\labelenumi{\arabic{enumi}.}
\tightlist
\item
  \textbf{Natural-Intelligence Fitness}: Mengukur kesehatan dan
  kewaspadaan ``lengkung refleks Id + limbik'' manusia.

  \begin{itemize}
  \tightlist
  \item
    \textbf{Bio-Responsiveness RT (Waktu Reaksi Bio)}: Waktu reaksi
    median dari stimulus fisik hingga aktuasi aman.
  \item
    \textbf{Cognitive-Load Index (Indeks Beban Kognitif)}: Usaha mental
    operator saat mengawasi AI.
  \item
    \textbf{Trust Calibration Error (Kesalahan Kalibrasi Kepercayaan)}:
    Kesenjangan antara kepercayaan manusia dan akurasi sebenarnya dari
    AI.
  \end{itemize}
\item
  \textbf{Collective Intelligence (Kecerdasan Kolektif)}: Mengukur
  kemampuan sistem untuk berkolaborasi dan belajar secara sinergis.

  \begin{itemize}
  \tightlist
  \item
    \textbf{Decision Alignment Index (Indeks Penyelarasan Keputusan)}:
    Porsi keputusan akhir yang mengutip masukan dari ketiga pilar
    kecerdasan (NI, CI, AI).
  \item
    \textbf{Resolution-Latency (Latensi Resolusi)}: Waktu rata-rata
    untuk menyelesaikan konflik NI-AI-CI yang ditandai oleh lapisan tata
    kelola.
  \end{itemize}
\item
  \textbf{Societal \& Ethical Impact (Dampak Sosial \& Etika)}: Mengukur
  nilai ``Rekayasa untuk Kemanusiaan'' tertinggi.

  \begin{itemize}
  \tightlist
  \item
    \textbf{Safety Incident Rate (Tingkat Insiden Keamanan)}: Kejadian
    kritis per 10 ribu jam operasi.
  \item
    \textbf{Inclusivity Spread (Penyebaran Inklusivitas)}: Ukuran
    distribusi nilai di berbagai kelompok pemangku kepentingan. Ini
    dapat menggunakan ukuran seperti Gini-like measure.
  \item
    \textbf{Sustainable-Benefit ROI (ROI Manfaat Berkelanjutan)}:
    Eksternalitas positif bersih jangka panjang (energi yang dihemat,
    kehidupan yang ditingkatkan) per unit biaya. Ini dapat dievaluasi
    melalui analisis siklus hidup (\emph{lifecycle analysis}) dan
    pemetaan SDG (\emph{SDG mappings}).
  \end{itemize}
\end{enumerate}

Metrik-metrik ini, dikelompokkan dalam Indeks Kapasitas Triune (TCI),
memungkinkan kita untuk melampaui pertanyaan ``Apakah modelnya
berfungsi?'' menjadi ``Apakah seluruh sistem Triune aman, cerdas, adil,
dan terus belajar?''---bukti nyata bahwa TI memberikan janjinya untuk
kemanusiaan.

\section{\texorpdfstring{\textbf{C.2: Model Kinerja dan Prototipe
Virtual
PSKV-S}}{C.2: Model Kinerja dan Prototipe Virtual PSKV-S}}\label{c.2-model-kinerja-dan-prototipe-virtual-pskv-s}

Konsep Product-Service-Knowledge-Value (PSKV-S) adalah abstraksi kunci
dalam TISE untuk menciptakan nilai holistik, sebagaimana dibahas dalam
Bab 6. PSKV-S dirancang untuk beroperasi sebagai sebuah ``mesin'' yang
mendorong kekuatan dan pertukaran nilai.

\section{\texorpdfstring{\textbf{C.2.1: Model Konversi Nilai dalam
PSKV-S}}{C.2.1: Model Konversi Nilai dalam PSKV-S}}\label{c.2.1-model-konversi-nilai-dalam-pskv-s}

Prinsip penciptaan nilai sebuah PSKV dapat diilustrasikan melalui kurva
penawaran-permintaan (lihat Gambar C.1). Pada awalnya, ongkos untuk
membuat sebuah PSKV mungkin lebih tinggi daripada nilai gunanya. Namun,
melalui proses rekayasa, nilai guna dapat dinaikkan dan ongkos
diturunkan, menciptakan apa yang disebut ``berlian PSKV'' di mana nilai
guna melebihi ongkos. Nilai yang tercipta ini kemudian dibagi dua oleh
harga; bagian atas diambil oleh pembeli, dan bagian bawah diambil oleh
penjual.

\bookmarksetup{startatroot}

\chapter{\texorpdfstring{\textbf{Lampiran D: Peta Jalan Riset
Triune-Intelligence Smart-Engineering
(TISE)}}{Lampiran D: Peta Jalan Riset Triune-Intelligence Smart-Engineering (TISE)}}\label{lampiran-d-peta-jalan-riset-triune-intelligence-smart-engineering-tise}

Tentu, berdasarkan sumber yang diberikan dan riwayat percakapan kita,
ada materi yang sangat relevan untuk Lampiran D. Lampiran ini akan
berfokus pada \textbf{Peta Jalan Riset (Research Road Map) TISE} yang
diringkas dalam dokumen sumber, memberikan gambaran strategis tentang
pengembangan ilmu rekayasa cerdas.

\begin{center}\rule{0.5\linewidth}{0.5pt}\end{center}

Lampiran ini menyajikan Peta Jalan Riset (Research Road Map) untuk
pengembangan Triune-Intelligence Smart-Engineering (TISE), sebagaimana
diringkas dalam Tabel 3 dari sumber. Peta jalan ini berfungsi sebagai
panduan strategis untuk memperluas tubuh pengetahuan ilmu rekayasa ke
dalam Platform Rekayasa Cerdas (PRC), dengan fokus pada permasalahan
pengembangan lingkungan hidup yang sehat, kreatif, aman, dan
berkelanjutan.

Peta jalan ini mengidentifikasi lima lapisan riset dan pengembangan
utama: \textbf{Sains, Teknologi, Produk, Pengguna, dan Pasar}. Di setiap
lapisan ini, terdapat tema-tema pengembangan yang meliputi
\textbf{Perangkat Keras (Hardware), Perangkat Lunak (Software),
Perangkat Cair (Liquidware), Perangkat Gas (Vapourware), Ruang Cerdas
(Smart Space), dan Ruang Spiritual (Spiritual Space)}.

Peta jalan riset ini secara eksplisit menguraikan evolusi riset, mulai
dari pemrosesan fundamental hingga rekayasa ruang hidup spiritual.

\section{\texorpdfstring{\textbf{D.1: Ringkasan Peta Jalan Riset
TISE}}{D.1: Ringkasan Peta Jalan Riset TISE}}\label{d.1-ringkasan-peta-jalan-riset-tise}

Berikut adalah ringkasan Peta Jalan Riset TISE (Versi 0.2) yang
menggambarkan bagaimana berbagai tema rekayasa berkembang melintasi
lapisan riset dan pengembangan selama periode tahun 2021 hingga 2026.

\textbf{Tabel D.1: Ringkasan Peta Jalan Riset TISE (Versi 0.2)}

\begin{longtable}[]{@{}
  >{\raggedright\arraybackslash}p{(\linewidth - 12\tabcolsep) * \real{0.0499}}
  >{\raggedright\arraybackslash}p{(\linewidth - 12\tabcolsep) * \real{0.1525}}
  >{\raggedright\arraybackslash}p{(\linewidth - 12\tabcolsep) * \real{0.1525}}
  >{\raggedright\arraybackslash}p{(\linewidth - 12\tabcolsep) * \real{0.1584}}
  >{\raggedright\arraybackslash}p{(\linewidth - 12\tabcolsep) * \real{0.1525}}
  >{\raggedright\arraybackslash}p{(\linewidth - 12\tabcolsep) * \real{0.1672}}
  >{\raggedright\arraybackslash}p{(\linewidth - 12\tabcolsep) * \real{0.1672}}@{}}
\toprule\noalign{}
\begin{minipage}[b]{\linewidth}\raggedright
\textbf{TEMA}
\end{minipage} & \begin{minipage}[b]{\linewidth}\raggedright
\textbf{Perangkat Keras}
\end{minipage} & \begin{minipage}[b]{\linewidth}\raggedright
\textbf{Perangkat Lunak}
\end{minipage} & \begin{minipage}[b]{\linewidth}\raggedright
\textbf{Perangkat Cair}
\end{minipage} & \begin{minipage}[b]{\linewidth}\raggedright
\textbf{Perangkat Gas}
\end{minipage} & \begin{minipage}[b]{\linewidth}\raggedright
\textbf{Ruang Cerdas}
\end{minipage} & \begin{minipage}[b]{\linewidth}\raggedright
\textbf{Ruang Spiritual}
\end{minipage} \\
\midrule\noalign{}
\endhead
\bottomrule\noalign{}
\endlastfoot
\textbf{MARKET} & Masyarakat Sadar Covid & \emph{Home Care Market} &
Pasar Rumah Sehat & Pasar \emph{Home Learning Productivity} & Pasar
Ekonomi Rumah & Pasar Kreatif Spiritual \\
\textbf{USER} & Pasien Covid & Pasien Umum & Masyarakat Umum &
Masyarakat Profesional & Masyarakat Ekonomi & Masyarakat Spiritual \\
\textbf{PRODUK} & \emph{Coughing Pre Screening} & \emph{Home Based
Recovery Room} & \emph{Smart Healthy Home, Robots} & \emph{Productive
Learning Home} & \emph{Home Based Virtual Economy} & \emph{Home Based
Virtual World} \\
\textbf{TEKNOLOGI} & \emph{Correlation Fractal Dimension; DSP System},
Platform Modelica & Platform Modelica untuk \emph{Taken ED; Grassberger
Algorithm} & Platform Modelica untuk \emph{SPS Smart Engineering} &
Platform Modelica untuk \emph{PSKV Smart Engineering} & Platform
Modelica untuk \emph{Living Space Smart Engineering} & Platform Modelica
untuk \emph{Spiritual Space Smart Engineering} \\
\textbf{SAINS} & \emph{Fraktal Processing} & \emph{Multifraktal
Processing} & \emph{SPS Robotic Engineering} & \emph{PSKV Engineering} &
\emph{Smart Living Engineering} & \emph{Smart Spiritual Engineering} \\
\textbf{TAHUN} & \textbf{2021} & \textbf{2022} & \textbf{2023} &
\textbf{2024} & \textbf{2025} & \textbf{2026} \\
\end{longtable}

\section{\texorpdfstring{\textbf{D.2: Penjelasan Lapisan dan Tema dalam
Peta Jalan
Riset}}{D.2: Penjelasan Lapisan dan Tema dalam Peta Jalan Riset}}\label{d.2-penjelasan-lapisan-dan-tema-dalam-peta-jalan-riset}

\section{\texorpdfstring{\textbf{D.2.1: Lapisan Riset dan
Pengembangan}}{D.2.1: Lapisan Riset dan Pengembangan}}\label{d.2.1-lapisan-riset-dan-pengembangan}

\begin{enumerate}
\def\labelenumi{\arabic{enumi}.}
\tightlist
\item
  \textbf{Sains (Science)}: Ini adalah lapisan terdalam, yang berfokus
  pada penemuan pengetahuan fundamental dan teori-teori baru. Dalam
  konteks TISE, ini mencakup eksplorasi prinsip-prinsip yang mendasari
  berbagai jenis kecerdasan dan sistem. Contohnya adalah \emph{Fraktal
  Processing} dan \emph{Multifraktal Processing}.
\item
  \textbf{Teknologi (Technology)}: Lapisan ini berfokus pada
  pengembangan atau adaptasi teknologi kunci yang memungkinkan solusi
  inovatif. Ini adalah jembatan antara sains fundamental dan aplikasi
  praktis. Contohnya penggunaan Platform Modelica untuk berbagai domain
  rekayasa cerdas.
\item
  \textbf{Produk (Product)}: Lapisan ini berurusan dengan artefak
  berwujud atau tidak berwujud yang dihasilkan. Ini adalah hasil
  langsung dari upaya rekayasa dan apa yang ditawarkan kepada pengguna.
  Contohnya, \emph{Smart Healthy Home, Robots}.
\item
  \textbf{Pengguna (User)}: Lapisan ini berpusat pada siapa yang akan
  menggunakan atau berinteraksi dengan produk dan layanan yang
  dikembangkan. Pemahaman mendalam tentang kebutuhan, perilaku, dan
  konteks pengguna adalah krusial dalam TISE. Contohnya, \emph{Pasien
  Covid} dan \emph{Masyarakat Spiritual}.
\item
  \textbf{Pasar (Market)}: Lapisan terluar ini mengidentifikasi segmen
  pasar yang akan dilayani oleh produk dan layanan yang dikembangkan.
  Ini melibatkan pemahaman tentang permintaan, kompetisi, dan
  keberlanjutan ekonomi. Contohnya adalah \emph{Masyarakat Sadar Covid}
  dan \emph{Pasar Kreatif Spiritual}.
\end{enumerate}

\section{\texorpdfstring{\textbf{D.2.2: Tema
Pengembangan}}{D.2.2: Tema Pengembangan}}\label{d.2.2-tema-pengembangan}

Peta jalan ini mengintegrasikan berbagai tema rekayasa yang mencerminkan
sifat multidisiplin dari TISE:

\begin{itemize}
\tightlist
\item
  \textbf{Perangkat Keras (Hardware)}: Berkaitan dengan komponen fisik
  dan infrastruktur. Ini mencakup energi, materi, dan informasi.
\item
  \textbf{Perangkat Lunak (Software)}: Berkaitan dengan kode, bahasa
  pemrograman, dan algoritma.
\item
  \textbf{Perangkat Cair (Liquidware)}: Merujuk pada konsep yang lebih
  abstrak seperti algoritma dan pengetahuan.
\item
  \textbf{Perangkat Gas (Vapourware)}: Mengacu pada nilai, makna, dan
  relasi, yang lebih \emph{intangible}.
\item
  \textbf{Ruang Cerdas (Smart Space)}: Berkaitan dengan lingkungan fisik
  dan virtual yang adaptif dan responsif terhadap kebutuhan manusia. Ini
  termasuk lingkungan hidup yang cerdas, kreatif, aman, sehat, dan
  berkelanjutan.
\item
  \textbf{Ruang Spiritual (Spiritual Space)}: Mengacu pada dimensi
  kesadaran, realitas, dan imersi untuk lingkungan spiritual, seperti
  kegiatan peribadatan digital.
\end{itemize}

\section{\texorpdfstring{\textbf{D.3: Implikasi Peta Jalan Riset
TISE}}{D.3: Implikasi Peta Jalan Riset TISE}}\label{d.3-implikasi-peta-jalan-riset-tise}

Peta jalan ini memiliki beberapa implikasi penting untuk riset dan
pengembangan dalam kerangka TISE:

\begin{itemize}
\tightlist
\item
  \textbf{Panduan Holistik}: Ini menggarisbawahi pendekatan TISE yang
  holistik, di mana penelitian di satu lapisan (misalnya, \emph{Fraktal
  Processing} di lapisan Sains) dapat mendukung pengembangan teknologi
  di lapisan atasnya (\emph{Correlation Fractal Dimension} di lapisan
  Teknologi), yang kemudian berujung pada produk dan layanan yang
  relevan untuk pasar tertentu (misalnya, \emph{Coughing Pre Screening}
  untuk \emph{Masyarakat Sadar Covid}).
\item
  \textbf{Integrasi Triune Intelligence}: Setiap tema dan lapisan dalam
  peta jalan ini dapat dianalisis melalui lensa Triune Intelligence.
  Misalnya, pengembangan \emph{Smart Spiritual Engineering} (Sains)
  untuk \emph{Home Based Virtual World} (Produk) bagi \emph{Masyarakat
  Spiritual} (Pengguna) akan sangat mengandalkan Kecerdasan Kultural
  untuk mendefinisikan nilai-nilai spiritual, Kecerdasan Alami untuk
  memahami pengalaman manusia, dan Kecerdasan Buatan untuk menciptakan
  lingkungan virtual yang imersif.
\item
  \textbf{Fokus pada Kehidupan Abad ke-21}: Peta jalan ini secara
  langsung menjawab permasalahan kemanusiaan Abad XXI, yaitu bagaimana
  manusia menjalani kehidupan yang istimewa melalui kontribusi maksimal
  berdasarkan pertumbuhan potensi sepenuhnya dalam ruang hidup yang
  cerdas, kreatif, aman, sehat, dan berkelanjutan. Ini mencakup
  kebutuhan transformasi energi, material, informasi, otomasi,
  kecerdasan berbasis pengetahuan, dan ruang imersif.
\item
  \textbf{Inovasi dan Disrupsi}: Peta jalan ini mendorong inovasi yang
  mampu menghadapi berbagai disrupsi di lima dimensi kehidupan (Fisik,
  Biologis, Psikologis, Sosial, Spiritual), memungkinkan pencarian
  ``kebenaran tentang esensi disrupsi dan strategi adaptasi''.
\end{itemize}

Dengan menyediakan peta jalan yang terstruktur ini, TISE tidak hanya
membimbing penelitian ke arah inovasi teknis, tetapi juga memastikan
bahwa inovasi tersebut selaras dengan kebutuhan manusia yang mendalam
dan berkontribusi pada penciptaan masa depan yang lebih baik dan lebih
bermakna.

\begin{center}\rule{0.5\linewidth}{0.5pt}\end{center}

\bookmarksetup{startatroot}

\chapter{\texorpdfstring{\textbf{Lampiran D: Peta Jalan Riset
Triune-Intelligence Smart-Engineering
(TISE)}}{Lampiran D: Peta Jalan Riset Triune-Intelligence Smart-Engineering (TISE)}}\label{lampiran-d-peta-jalan-riset-triune-intelligence-smart-engineering-tise-1}

Tentu, berdasarkan sumber yang diberikan dan riwayat percakapan kita,
ada materi yang sangat relevan untuk Lampiran D. Lampiran ini akan
berfokus pada \textbf{Peta Jalan Riset (Research Road Map) TISE} yang
diringkas dalam dokumen sumber, memberikan gambaran strategis tentang
pengembangan ilmu rekayasa cerdas.

\begin{center}\rule{0.5\linewidth}{0.5pt}\end{center}

Lampiran ini menyajikan Peta Jalan Riset (Research Road Map) untuk
pengembangan Triune-Intelligence Smart-Engineering (TISE), sebagaimana
diringkas dalam Tabel 3 dari sumber. Peta jalan ini berfungsi sebagai
panduan strategis untuk memperluas tubuh pengetahuan ilmu rekayasa ke
dalam Platform Rekayasa Cerdas (PRC), dengan fokus pada permasalahan
pengembangan lingkungan hidup yang sehat, kreatif, aman, dan
berkelanjutan.

Peta jalan ini mengidentifikasi lima lapisan riset dan pengembangan
utama: \textbf{Sains, Teknologi, Produk, Pengguna, dan Pasar}. Di setiap
lapisan ini, terdapat tema-tema pengembangan yang meliputi
\textbf{Perangkat Keras (Hardware), Perangkat Lunak (Software),
Perangkat Cair (Liquidware), Perangkat Gas (Vapourware), Ruang Cerdas
(Smart Space), dan Ruang Spiritual (Spiritual Space)}.

Peta jalan riset ini secara eksplisit menguraikan evolusi riset, mulai
dari pemrosesan fundamental hingga rekayasa ruang hidup spiritual.

\section{\texorpdfstring{\textbf{D.1: Ringkasan Peta Jalan Riset
TISE}}{D.1: Ringkasan Peta Jalan Riset TISE}}\label{d.1-ringkasan-peta-jalan-riset-tise-1}

Berikut adalah ringkasan Peta Jalan Riset TISE (Versi 0.2) yang
menggambarkan bagaimana berbagai tema rekayasa berkembang melintasi
lapisan riset dan pengembangan selama periode tahun 2021 hingga 2026.

\textbf{Tabel D.1: Ringkasan Peta Jalan Riset TISE (Versi 0.2)}

\begin{longtable}[]{@{}
  >{\raggedright\arraybackslash}p{(\linewidth - 12\tabcolsep) * \real{0.0499}}
  >{\raggedright\arraybackslash}p{(\linewidth - 12\tabcolsep) * \real{0.1525}}
  >{\raggedright\arraybackslash}p{(\linewidth - 12\tabcolsep) * \real{0.1525}}
  >{\raggedright\arraybackslash}p{(\linewidth - 12\tabcolsep) * \real{0.1584}}
  >{\raggedright\arraybackslash}p{(\linewidth - 12\tabcolsep) * \real{0.1525}}
  >{\raggedright\arraybackslash}p{(\linewidth - 12\tabcolsep) * \real{0.1672}}
  >{\raggedright\arraybackslash}p{(\linewidth - 12\tabcolsep) * \real{0.1672}}@{}}
\toprule\noalign{}
\begin{minipage}[b]{\linewidth}\raggedright
\textbf{TEMA}
\end{minipage} & \begin{minipage}[b]{\linewidth}\raggedright
\textbf{Perangkat Keras}
\end{minipage} & \begin{minipage}[b]{\linewidth}\raggedright
\textbf{Perangkat Lunak}
\end{minipage} & \begin{minipage}[b]{\linewidth}\raggedright
\textbf{Perangkat Cair}
\end{minipage} & \begin{minipage}[b]{\linewidth}\raggedright
\textbf{Perangkat Gas}
\end{minipage} & \begin{minipage}[b]{\linewidth}\raggedright
\textbf{Ruang Cerdas}
\end{minipage} & \begin{minipage}[b]{\linewidth}\raggedright
\textbf{Ruang Spiritual}
\end{minipage} \\
\midrule\noalign{}
\endhead
\bottomrule\noalign{}
\endlastfoot
\textbf{MARKET} & Masyarakat Sadar Covid & \emph{Home Care Market} &
Pasar Rumah Sehat & Pasar \emph{Home Learning Productivity} & Pasar
Ekonomi Rumah & Pasar Kreatif Spiritual \\
\textbf{USER} & Pasien Covid & Pasien Umum & Masyarakat Umum &
Masyarakat Profesional & Masyarakat Ekonomi & Masyarakat Spiritual \\
\textbf{PRODUK} & \emph{Coughing Pre Screening} & \emph{Home Based
Recovery Room} & \emph{Smart Healthy Home, Robots} & \emph{Productive
Learning Home} & \emph{Home Based Virtual Economy} & \emph{Home Based
Virtual World} \\
\textbf{TEKNOLOGI} & \emph{Correlation Fractal Dimension; DSP System},
Platform Modelica & Platform Modelica untuk \emph{Taken ED; Grassberger
Algorithm} & Platform Modelica untuk \emph{SPS Smart Engineering} &
Platform Modelica untuk \emph{PSKV Smart Engineering} & Platform
Modelica untuk \emph{Living Space Smart Engineering} & Platform Modelica
untuk \emph{Spiritual Space Smart Engineering} \\
\textbf{SAINS} & \emph{Fraktal Processing} & \emph{Multifraktal
Processing} & \emph{SPS Robotic Engineering} & \emph{PSKV Engineering} &
\emph{Smart Living Engineering} & \emph{Smart Spiritual Engineering} \\
\textbf{TAHUN} & \textbf{2021} & \textbf{2022} & \textbf{2023} &
\textbf{2024} & \textbf{2025} & \textbf{2026} \\
\end{longtable}

\section{\texorpdfstring{\textbf{D.2: Penjelasan Lapisan dan Tema dalam
Peta Jalan
Riset}}{D.2: Penjelasan Lapisan dan Tema dalam Peta Jalan Riset}}\label{d.2-penjelasan-lapisan-dan-tema-dalam-peta-jalan-riset-1}

\section{\texorpdfstring{\textbf{D.2.1: Lapisan Riset dan
Pengembangan}}{D.2.1: Lapisan Riset dan Pengembangan}}\label{d.2.1-lapisan-riset-dan-pengembangan-1}

\begin{enumerate}
\def\labelenumi{\arabic{enumi}.}
\tightlist
\item
  \textbf{Sains (Science)}: Ini adalah lapisan terdalam, yang berfokus
  pada penemuan pengetahuan fundamental dan teori-teori baru. Dalam
  konteks TISE, ini mencakup eksplorasi prinsip-prinsip yang mendasari
  berbagai jenis kecerdasan dan sistem. Contohnya adalah \emph{Fraktal
  Processing} dan \emph{Multifraktal Processing}.
\item
  \textbf{Teknologi (Technology)}: Lapisan ini berfokus pada
  pengembangan atau adaptasi teknologi kunci yang memungkinkan solusi
  inovatif. Ini adalah jembatan antara sains fundamental dan aplikasi
  praktis. Contohnya penggunaan Platform Modelica untuk berbagai domain
  rekayasa cerdas.
\item
  \textbf{Produk (Product)}: Lapisan ini berurusan dengan artefak
  berwujud atau tidak berwujud yang dihasilkan. Ini adalah hasil
  langsung dari upaya rekayasa dan apa yang ditawarkan kepada pengguna.
  Contohnya, \emph{Smart Healthy Home, Robots}.
\item
  \textbf{Pengguna (User)}: Lapisan ini berpusat pada siapa yang akan
  menggunakan atau berinteraksi dengan produk dan layanan yang
  dikembangkan. Pemahaman mendalam tentang kebutuhan, perilaku, dan
  konteks pengguna adalah krusial dalam TISE. Contohnya, \emph{Pasien
  Covid} dan \emph{Masyarakat Spiritual}.
\item
  \textbf{Pasar (Market)}: Lapisan terluar ini mengidentifikasi segmen
  pasar yang akan dilayani oleh produk dan layanan yang dikembangkan.
  Ini melibatkan pemahaman tentang permintaan, kompetisi, dan
  keberlanjutan ekonomi. Contohnya adalah \emph{Masyarakat Sadar Covid}
  dan \emph{Pasar Kreatif Spiritual}.
\end{enumerate}

\section{\texorpdfstring{\textbf{D.2.2: Tema
Pengembangan}}{D.2.2: Tema Pengembangan}}\label{d.2.2-tema-pengembangan-1}

Peta jalan ini mengintegrasikan berbagai tema rekayasa yang mencerminkan
sifat multidisiplin dari TISE:

\begin{itemize}
\tightlist
\item
  \textbf{Perangkat Keras (Hardware)}: Berkaitan dengan komponen fisik
  dan infrastruktur. Ini mencakup energi, materi, dan informasi.
\item
  \textbf{Perangkat Lunak (Software)}: Berkaitan dengan kode, bahasa
  pemrograman, dan algoritma.
\item
  \textbf{Perangkat Cair (Liquidware)}: Merujuk pada konsep yang lebih
  abstrak seperti algoritma dan pengetahuan.
\item
  \textbf{Perangkat Gas (Vapourware)}: Mengacu pada nilai, makna, dan
  relasi, yang lebih \emph{intangible}.
\item
  \textbf{Ruang Cerdas (Smart Space)}: Berkaitan dengan lingkungan fisik
  dan virtual yang adaptif dan responsif terhadap kebutuhan manusia. Ini
  termasuk lingkungan hidup yang cerdas, kreatif, aman, sehat, dan
  berkelanjutan.
\item
  \textbf{Ruang Spiritual (Spiritual Space)}: Mengacu pada dimensi
  kesadaran, realitas, dan imersi untuk lingkungan spiritual, seperti
  kegiatan peribadatan digital.
\end{itemize}

\section{\texorpdfstring{\textbf{D.3: Implikasi Peta Jalan Riset
TISE}}{D.3: Implikasi Peta Jalan Riset TISE}}\label{d.3-implikasi-peta-jalan-riset-tise-1}

Peta jalan ini memiliki beberapa implikasi penting untuk riset dan
pengembangan dalam kerangka TISE:

\begin{itemize}
\tightlist
\item
  \textbf{Panduan Holistik}: Ini menggarisbawahi pendekatan TISE yang
  holistik, di mana penelitian di satu lapisan (misalnya, \emph{Fraktal
  Processing} di lapisan Sains) dapat mendukung pengembangan teknologi
  di lapisan atasnya (\emph{Correlation Fractal Dimension} di lapisan
  Teknologi), yang kemudian berujung pada produk dan layanan yang
  relevan untuk pasar tertentu (misalnya, \emph{Coughing Pre Screening}
  untuk \emph{Masyarakat Sadar Covid}).
\item
  \textbf{Integrasi Triune Intelligence}: Setiap tema dan lapisan dalam
  peta jalan ini dapat dianalisis melalui lensa Triune Intelligence.
  Misalnya, pengembangan \emph{Smart Spiritual Engineering} (Sains)
  untuk \emph{Home Based Virtual World} (Produk) bagi \emph{Masyarakat
  Spiritual} (Pengguna) akan sangat mengandalkan Kecerdasan Kultural
  untuk mendefinisikan nilai-nilai spiritual, Kecerdasan Alami untuk
  memahami pengalaman manusia, dan Kecerdasan Buatan untuk menciptakan
  lingkungan virtual yang imersif.
\item
  \textbf{Fokus pada Kehidupan Abad ke-21}: Peta jalan ini secara
  langsung menjawab permasalahan kemanusiaan Abad XXI, yaitu bagaimana
  manusia menjalani kehidupan yang istimewa melalui kontribusi maksimal
  berdasarkan pertumbuhan potensi sepenuhnya dalam ruang hidup yang
  cerdas, kreatif, aman, sehat, dan berkelanjutan. Ini mencakup
  kebutuhan transformasi energi, material, informasi, otomasi,
  kecerdasan berbasis pengetahuan, dan ruang imersif.
\item
  \textbf{Inovasi dan Disrupsi}: Peta jalan ini mendorong inovasi yang
  mampu menghadapi berbagai disrupsi di lima dimensi kehidupan (Fisik,
  Biologis, Psikologis, Sosial, Spiritual), memungkinkan pencarian
  ``kebenaran tentang esensi disrupsi dan strategi adaptasi''.
\end{itemize}

Dengan menyediakan peta jalan yang terstruktur ini, TISE tidak hanya
membimbing penelitian ke arah inovasi teknis, tetapi juga memastikan
bahwa inovasi tersebut selaras dengan kebutuhan manusia yang mendalam
dan berkontribusi pada penciptaan masa depan yang lebih baik dan lebih
bermakna.

\begin{center}\rule{0.5\linewidth}{0.5pt}\end{center}

\bookmarksetup{startatroot}

\chapter{\texorpdfstring{\textbf{Lampiran F: Konsep Fondasi Lanjutan dan
Model
Teknis}}{Lampiran F: Konsep Fondasi Lanjutan dan Model Teknis}}\label{lampiran-f-konsep-fondasi-lanjutan-dan-model-teknis}

Tentu, berdasarkan sumber yang diberikan dan riwayat percakapan kita,
ada materi yang sangat relevan untuk Lampiran F. Lampiran ini akan
berfokus pada konsep-konsep fondasi lanjutan dan model teknis yang
mendasari paradigma TISE, memberikan detail yang melengkapi pembahasan
di bab-bab utama.

\begin{center}\rule{0.5\linewidth}{0.5pt}\end{center}

Lampiran ini menyediakan detail teknis dan elaborasi konseptual yang
lebih mendalam untuk aspek-aspek kunci dari Triune-Intelligence
Smart-Engineering (TISE). Pembahasan di sini bertujuan untuk memberikan
pemahaman yang lebih kaya tentang definisi rekayasa, model berpikir yang
mendasarinya, dimensi realitas yang relevan, abstraksi mesin, serta
aspek formal dari sistem otonom dan pemodelan pengetahuan.

\section{\texorpdfstring{\textbf{F.1: Definisi Rekayasa dan
Fungsi-fungsi Intinya dalam
TISE}}{F.1: Definisi Rekayasa dan Fungsi-fungsi Intinya dalam TISE}}\label{f.1-definisi-rekayasa-dan-fungsi-fungsi-intinya-dalam-tise}

Dalam kerangka TISE, \textbf{rekayasa (keinsinyuran, atau
\emph{engineering})} didefinisikan sebagai: 1. \textbf{Aplikasi kreatif}
dari pengetahuan ilmiah. 2. Untuk menghasilkan \textbf{Rancang-Bangun
(RB)}. 3. Yang mampu mengerahkan \textbf{kekuatan alam secara aman dan
terkendali}. 4. Untuk \textbf{memecahkan masalah manusia yang penting
dan berharga}.

RB ini adalah ``mesin yang kuat untuk memecahkan masalah manusia yang
berat dan sukar''. Kekuatan RB berasal dari kemampuannya untuk
mengerahkan kekuatan alam. Ilmu rekayasa membekali rekayasawan
(insinyur, \emph{engineers}) untuk menjalankan keempat proses tersebut.
Seorang rekayasawan adalah \emph{problem-solver} profesional dengan
kompetensi yang lengkap: berilmu, kreatif, \emph{power sensitive},
\emph{safety oriented}, penting, dan bernilai tinggi.

Rekayasa tradisional seringkali berfokus pada optimasi parameter teknis
(misalnya, efisiensi, kecepatan, biaya). Namun, TISE memperluasnya ke
penciptaan nilai holistik dan peningkatan kesejahteraan manusia secara
berkelanjutan, bergeser dari \emph{problem-solving} murni menjadi
\emph{capability-building} dan ``tindakan penciptaan dunia yang lebih
baik''.

\section{\texorpdfstring{\textbf{F.1.1: Tiga Fase Rekayasa
Monodisiplin}}{F.1.1: Tiga Fase Rekayasa Monodisiplin}}\label{f.1.1-tiga-fase-rekayasa-monodisiplin}

Secara tradisional, seorang rekayasawan (monodisiplin) bekerja dalam
tiga fase: a. \textbf{Analisis Kebutuhan Manusia}: Menganalisis
kebutuhan manusia dan merumuskan fungsi (transformasi materi, energi,
informasi) serta spesifikasi (kinerja) RB yang dapat mengatasinya. b.
\textbf{Pencarian Prinsip Ilmu Pengetahuan}: Mencari prinsip-prinsip
ilmu pengetahuan serta instrumen yang terkait dengan transformasi
materi, energi, informasi yang berguna untuk menyusun RB.
c.~\textbf{Inovasi dan Aplikasi RB}: Menginovasi RB dengan merangkai
prinsip-prinsip ilmu pengetahuan secara kreatif sehingga RB mampu
memenuhi fungsi dan spesifikasi tersebut, serta menerapkannya sebagai
solusi.

\section{\texorpdfstring{\textbf{F.1.2: Tujuh Fungsi Rekayasa dalam
TISE}}{F.1.2: Tujuh Fungsi Rekayasa dalam TISE}}\label{f.1.2-tujuh-fungsi-rekayasa-dalam-tise}

Tuntutan terhadap rekayasawan telah berkembang untuk menghasilkan RB
yang kompleks, sehingga fungsi rekayasa jarang bisa dilakukan oleh
seorang insinyur saja; diperlukan pembagian tugas. Fungsi rekayasa dapat
diurai dan dibagi ke dalam tujuh fungsi: a. \textbf{Riset}: Menemukan
Pengetahuan Baru. b. \textbf{Pengembangan}: Menginvensi Instrumen,
Mesin, dan Struktur. c.~\textbf{Desain}: Menginovasi Solusi.
d.~\textbf{Konstruksi}: Mewujudkan Solusi. e. \textbf{Produksi}:
Mereplikasi Solusi Secara Massal. f.~\textbf{Operasi}: Mempekerjakan
Solusi. g. \textbf{Manajemen}: Menghasilkan Nilai Pemangku Kepentingan.

\section{\texorpdfstring{\textbf{F.2: Siklus Proses Berpikir
Rekayasa}}{F.2: Siklus Proses Berpikir Rekayasa}}\label{f.2-siklus-proses-berpikir-rekayasa}

Proses berpikir rekayasa itu sendiri terdiri dari empat tahap besar: 1.
\textbf{Logika}: Memahami masalah dan kebutuhan akan kekuatan RB, yang
melibatkan studi dan survei kepada pengguna. 2. \textbf{Rasional}:
Memahami realitas alam di lingkungan, kekuatan apa yang ada, dan
instrumen/komponen yang tersedia. Ini dilakukan melalui studi literatur,
pembuatan hipotesis, dan eksperimen laboratorium. 3. \textbf{Kreatif}:
Mendesain logika kerja (Logos) RB yang bisa memenuhi kebutuhan dan
membuat modelnya, melalui studi skema \emph{engines}, rekombinasi,
pemodelan, simulasi, dan uji \emph{testbed}. 4. \textbf{Konstruktif}:
Mewujudkan model itu ke dalam RB yang realistis, yang melibatkan studi
situs, rekonfigurasi, \emph{blueprint/site-plan}, instalasi, uji
kelaikan, \emph{soft launching}, operasional, pemeliharaan, dan akhir
siklus hidup.

\section{\texorpdfstring{\textbf{F.3: Realitas, Dimensi, dan Tingkat
Kesadaran dalam
TISE}}{F.3: Realitas, Dimensi, dan Tingkat Kesadaran dalam TISE}}\label{f.3-realitas-dimensi-dan-tingkat-kesadaran-dalam-tise}

TISE mengakui bahwa desain lingkungan cerdas membutuhkan deskripsi
realitas yang relevan. Realitas digambarkan sebagai interaksi antara
entitas dan gaya (\emph{forces}).

\section{\texorpdfstring{\textbf{F.3.1: Empat Pandangan Realitas (Four
Views of
Reality)}}{F.3.1: Empat Pandangan Realitas (Four Views of Reality)}}\label{f.3.1-empat-pandangan-realitas-four-views-of-reality}

Teori yang mendasari TISE mengkonstruksi empat pandangan realitas yang
sesuai dengan tingkat kesadaran: 1. \textbf{Realitas Fisik (Physical
Reality)}: Diatur oleh hukum termodinamika dan empat gaya primer
(gravitasi, elektromagnetik, gaya nuklir kuat, dan gaya nuklir lemah).
Semua entitas fisik dipengaruhi oleh gaya-gaya ini. 2. \textbf{Realitas
Biologis (Biological Reality)}: Diatur oleh gaya evolusi Darwinian,
algoritma genetik, fungsi \emph{fitness}, dan hukum \emph{survival of
the fittest}. Entitas biologis berusaha mempertahankan integritasnya dan
melestarikan spesiesnya. 3. \textbf{Realitas Kultural (Cultural
Reality)}: Tiga gaya (sosial, legal, dan ekonomi) berinteraksi dengan
hukum hierarki sosial, keagungan, dukungan; keamanan hukum,
perlindungan, dan penghormatan; serta kepemilikan dan akses aset
ekonomi, sumber daya, dan keuangan. Semua entitas budaya didorong oleh
gaya-gaya ini, yang pada dasarnya bersifat informasional. 4.
\textbf{Realitas Hiper (Hyper Reality)}: Medan gaya mencakup perasaan
kejutan kreatif, kegembiraan, pencapaian, harmoni, keindahan, dan cinta.
Pada manusia, ini memengaruhi perasaan dan emosi.

Desain sebuah RB biasanya melibatkan tiga pandangan realitas: domain
desain (tempat hasil desain dan entitas utama berada), pandangan yang
lebih rendah (sebagai sumber komponen dan gaya), dan pandangan yang
lebih tinggi (tempat hasil desain dievaluasi berdasarkan nilainya).

\section{\texorpdfstring{\textbf{F.3.2: Dimensi Realitas dan Tingkat
Kesadaran}}{F.3.2: Dimensi Realitas dan Tingkat Kesadaran}}\label{f.3.2-dimensi-realitas-dan-tingkat-kesadaran}

Manusia memiliki tingkat kesadaran yang berbeda terhadap dimensi-dimensi
realitas, yang dapat diilustrasikan sebagai berikut:

\begin{itemize}
\tightlist
\item
  \textbf{Naturalisme}: Kesadaran terhadap medan fisik, energi, dan
  teknologi. Tumbuhan dan hewan menyadarinya.
\item
  \textbf{Spiritisme}: Pasokan kebutuhan hidup, makanan, air bersih, dan
  kesehatan. Hewan menyadarinya.
\item
  \textbf{Simbolisme}: Efektivitas berbahasa dan penguasaan diri.
  Manusia biasa menyadarinya.
\item
  \textbf{Panteisme}: Kekuatan karakter dalam membangun organisasi dan
  komunitas. Manusia \emph{homosapiens} (terampil) menguasai tiga
  dimensi pertama dan menyadari panteisme. \emph{Homologos} (cerdas)
  menguasai empat dimensi ini. Komputer juga mampu menguasai empat
  dimensi ini.
\item
  \textbf{Teisme}: Kreativitas, penciptaan nilai, spiritualitas.
  \emph{Pra-Homocordium} menyadari teisme, sedangkan \emph{Homocordium}
  menguasainya.
\end{itemize}

Disrupsi teknologi memaksa manusia untuk meningkatkan kesadaran dirinya
melampaui komputer. Pilihan yang lebih realistis bagi kebanyakan manusia
adalah meningkatkan kesadaran sampai ke dimensi kelima (teisme dan
panenteisme).

\section{\texorpdfstring{\textbf{F.3.3: Evolusi Manusia dalam Konteks
TISE}}{F.3.3: Evolusi Manusia dalam Konteks TISE}}\label{f.3.3-evolusi-manusia-dalam-konteks-tise}

Dalam konteks ini, TISE mengidentifikasi evolusi manusia: *
\textbf{Homosapiens}: Manusia yang terampil secara fisik, dengan
kemampuan \emph{perception, recall, plan, act} yang instinktif, reaktif,
responsif, dan apresiatif. * \textbf{Homologos}: Manusia yang cerdas
secara logika, mampu berpikir dan mengingat dengan lebih kompleks,
memanfaatkan pengetahuan. * \textbf{Homodeus}: Manusia yang diberdayakan
oleh komputasi, mencapai kemampuan super. * \textbf{Homocordium}:
Manusia hati, yang berpusat pada nilai, etika, moralitas, empati,
kreativitas, intuisi, dan tujuan spiritual. Ini adalah esensi dari
``Vokator''.

Adanya kecerdasan buatan meniscayakan manusia beradaptasi dari manusia
otot \& otak menjadi manusia hati (\emph{homocordium}).

\section{\texorpdfstring{\textbf{F.4: Abstraksi Mesin (Engine
Abstraction) dan Siklus Empat
Langkahnya}}{F.4: Abstraksi Mesin (Engine Abstraction) dan Siklus Empat Langkahnya}}\label{f.4-abstraksi-mesin-engine-abstraction-dan-siklus-empat-langkahnya}

Sebagaimana dibahas di Bab 3 dan 2.1, setiap artefak rekayasa adalah
sebuah ``mesin''. Mesin diabstraksikan sebagai entitas yang memanfaatkan
dan mengubah gaya yang tersedia di lingkungan untuk melakukan kerja yang
diinginkan. Kunci dari sebuah mesin adalah operasi siklusnya, yang
memungkinkannya untuk melakukan kerja secara terus-menerus.

Siklus ini dapat dianalogikan dengan \textbf{roda gila
(\emph{flywheel})} yang berputar, menyimpan, dan melepaskan energi
kerja. Siklus empat langkah yang khas dalam abstraksi ini meliputi: 1.
\textbf{Pengumpulan Energi (Energy Collection/Intake)}: Mesin
mengumpulkan energi atau \emph{input} dari sumber eksternal. 2.
\textbf{Kompresi Energi (Energy Compression/Encoding)}: Energi yang
terkumpul diproses dan diubah menjadi bentuk yang lebih kuat atau dapat
digunakan, disebut ``energi MESIN'' (\emph{ENGINE energy}). 3.
\textbf{Dekompresi Energi menjadi Energi Roda Gila (Energy Decompression
into Flywheel Energy)}: Energi yang tersimpan kemudian dikonversi
menjadi ``energi RODA GILA'' (\emph{FLYWHEEL energy}), yang
merepresentasikan energi kerja yang siap untuk dimanfaatkan. 4.
\textbf{Pembersihan/Reset Mesin (Engine Cleaning/Reset)}: Sebagian dari
energi yang dihasilkan digunakan untuk mengatur ulang mesin,
membersihkan ``limbah'' atau sisa proses, dan menyiapkannya untuk siklus
berikutnya.

Setiap siklus yang selesai menghasilkan porsi energi kerja yang tetap.
Untuk beban kerja yang lebih besar, mesin harus beroperasi pada
frekuensi yang lebih tinggi.

\section{\texorpdfstring{\textbf{F.5: Formalisasi Penutupan Sistem
Informasional}}{F.5: Formalisasi Penutupan Sistem Informasional}}\label{f.5-formalisasi-penutupan-sistem-informasional}

Dalam MBSE, terdapat kebutuhan akan kerangka kerja formal untuk
penutupan sistem. Penelitian telah dilakukan untuk formalisasi teoretis
sistem tertutup, khususnya dalam hal \textbf{penutupan informasional}.

\begin{itemize}
\tightlist
\item
  Sebuah teorema menunjukkan hubungan untuk tingkat informasi bersama
  (\emph{mutual information}) yang disajikan di batas sistem tertutup
  secara informasional.
\item
  Untuk menjaga penutupan pada keadaan \emph{n} + 1, keluaran sistem
  tertutup ke lingkungan pada keadaan \emph{n} (jumlah informasi yang
  ditransmisikan dari sistem ke lingkungannya) harus mengikuti teorema
  ini.
\end{itemize}

Ini relevan dalam TISE untuk memastikan stabilitas dan prediktabilitas
sistem cerdas yang berinteraksi dengan lingkungannya.

\section{\texorpdfstring{\textbf{F.6: Taksonomi Sistem Otonom (SoAS) dan
Level Otonomi
(LoA)}}{F.6: Taksonomi Sistem Otonom (SoAS) dan Level Otonomi (LoA)}}\label{f.6-taksonomi-sistem-otonom-soas-dan-level-otonomi-loa}

Integrasi otonomi ke dalam sistem yang lebih besar (\emph{System of
Systems} - SoS) menciptakan kompleksitas baru. Diperlukan pengembangan
taksonomi yang komprehensif untuk \emph{System of Autonomous Systems}
(SoAS).

\begin{itemize}
\tightlist
\item
  \textbf{Taksonomi LoA}: Taksonomi berdasarkan \emph{Level of Autonomy}
  (LoA) memberikan gambaran yang lebih jelas tentang berbagai tingkat
  kemampuan otonom sistem, menghasilkan bahasa umum untuk berbagai
  disiplin ilmu rekayasa.
\item
  \textbf{Peringkat SoAS}: Sebuah SoAS dapat diberi peringkat lebih
  tinggi dalam LoA jika sistemnya mampu melakukan lebih banyak tugas
  misi secara mandiri.
\item
  \textbf{Peran AI dalam Kooperasi SoAS}: SoAS menggunakan algoritma
  kerja sama AI, membuat sistem berkolaborasi dalam melakukan tugas
  misi. Contohnya adalah dalam perencanaan multi-UAV (\emph{multi-UAV
  planning}).
\item
  \textbf{Framework ALFUS}: \emph{Autonomy Levels for Unmanned Systems
  (ALFUS)} adalah kerangka kerja yang dikembangkan oleh NIST untuk
  menentukan tingkat otonomi sistem tak berawak.
\end{itemize}

Pemahaman tentang taksonomi dan LoA ini sangat penting untuk merancang
sistem TISE yang menyeimbangkan kemampuan otonom dengan pengawasan dan
intervensi manusia, sesuai dengan filosofi ``Engineers empower humans.''

\section{\texorpdfstring{\textbf{F.7: Kerangka Pemodelan yang Dapat
Disusun (\emph{Composable Modeling
Frameworks})}}{F.7: Kerangka Pemodelan yang Dapat Disusun (Composable Modeling Frameworks)}}\label{f.7-kerangka-pemodelan-yang-dapat-disusun-composable-modeling-frameworks}

Konsep kerangka pemodelan yang dapat disusun (\emph{composable modeling
frameworks}) sangat penting untuk mengelola sistem yang kompleks dan
mempromosikan penggunaan ulang.

\begin{itemize}
\tightlist
\item
  \textbf{MBSE dan Komposabilitas}: Komposabilitas sintaksis mengacu
  pada kemampuan untuk menggabungkan model atau komponen dari sumber
  yang berbeda. MBSE berupaya meningkatkan efisiensi implementasi MBSE
  melalui penggunaan pola (\emph{patterns}) dan kerangka kerja pemodelan
  yang dapat disusun.
\item
  \textbf{Simulasi Efektivitas Tempur}: Penelitian telah berfokus pada
  kerangka pemodelan yang dapat disusun untuk sistem yang kompleks,
  seperti sistem pertahanan udara dan rudal berjaringan, yang disebut
  \emph{Composable Modeling Frameworks for Networked Air \& Missile
  Defense Systems}. TISE juga membahas penggunaan model untuk simulasi
  dinamika penawaran-permintaan dan berbagai sumber.
\item
  \textbf{Metode Ontologis untuk Pemodelan}: Metode pemodelan ontologis
  (\emph{ontological metamodeling method}), seperti ekstensi UML dengan
  tag, dapat digunakan untuk membangun \emph{ontoCMFs} untuk
  representasi taktik. Ini membantu dalam menciptakan konseptualisasi
  bersama (\emph{conceptual alignment}) yang penting untuk
  interoperabilitas simulasi.
\item
  \textbf{Pola Desain (\emph{Design Patterns})}: Dalam MBSE, pola desain
  digunakan untuk reusabilitas pengetahuan. Misalnya, Gasser menjelaskan
  pola konstruksi perilaku (\emph{behavioral construct patterns}) untuk
  memfasilitasi pemodelan perilaku sistem, sehingga insinyur dapat
  berfokus pada perilaku yang diharapkan daripada estetika diagram.
\end{itemize}

Integrasi kerangka pemodelan yang dapat disusun ini memungkinkan TISE
untuk merancang sistem yang lebih modular, fleksibel, dan efisien,
selaras dengan karakteristik ``Dapat Dilaksanakan'' (Doable) dan
``Metodis'' (Methodic).

\begin{center}\rule{0.5\linewidth}{0.5pt}\end{center}

\bookmarksetup{startatroot}

\chapter{Summary}\label{summary}

In summary, this book has no content whatsoever.

\bookmarksetup{startatroot}

\chapter*{References}\label{references}
\addcontentsline{toc}{chapter}{References}

\markboth{References}{References}

\phantomsection\label{refs}
\begin{CSLReferences}{1}{0}
\bibitem[\citeproctext]{ref-knuth84}
Knuth, Donald E. 1984. {``Literate Programming.''} \emph{Comput. J.} 27
(2): 97--111. \url{https://doi.org/10.1093/comjnl/27.2.97}.

\end{CSLReferences}




\end{document}
